\documentclass{llncs}

\usepackage{standalone}
\documentclass{llncs}

\usepackage{unicode-math}
\usepackage{newcomputermodern}
\usepackage{microtype}

\usepackage{xcolor-material}
\usepackage{hyperref}
\usepackage{ebproof}
\usepackage{array}

\usepackage{jbw-boxfigure}
\usepackage{jbw-fix-hyperref-autoref}

\title{%
	Automatically Translating Proof Systems
	for SMT Solvers to the λΠ-calculus}
\author{Ciarán Dunne, Guillaume Burel}
\institute{INRIA, ENS Paris-Saclay}

\usepackage[backend=biber]{biblatex}
\addbibresource{inria.bib}

% font faces
\newcommand{\msf}{\mathsf}
\newcommand{\mbf}{\mathbf}
\newcommand{\mscr}{\mathscr}
\newcommand{\mtt}[1]{{\color{MaterialPink500}{\texttt{#1}}}}
\newcommand{\ptn}[1]{{\color{MaterialPurple500}{\texttt{#1}}}}
\newcommand{\mcomment}[1]{{\color{MaterialIndigo500}{\text{#1}}}}
\newcommand{\wrapp}[4]{{\color{#1}{#2}}{#4}{\color{#1}{#3}}}
\newcommand{\prn}[1]{\wrapp{MaterialGrey600}{\lparen}{\rparen}{#1}}

\newcommand{\cvc}[1]{\textsc{cvc{#1}}}
\newcommand{\verit}{\textsc{veriT}}

% ----- abstract syntax ----------
% plurality, binding
\newcommand{\plur}[2]{{#1}_{1}\ldots{#1}_{#2}}
\newcommand{\plurcomma}[2]{{#1}_{1}, \ldots, {#1}_{#2}}
\newcommand{\maybe}[1]{\wrapp{MaterialGrey600}{\langle}{\rangle_?}{#1}}
\newcommand{\many}[1]{\wrapp{MaterialGrey600}{\langle}{\rangle_+}{#1}}
\newcommand{\any}[1]{\wrapp{MaterialGrey600}{\langle}{\rangle_∗}{#1}}
\newcommand{\some}[1]{\wrapp{MaterialGrey600}{\langle}{\rangle}{#1}}
\newcommand{\binddot}[3]{{#1\,#2.\,#3}}

% binders
\newcommand{\lam}{\binddot{λ}}
\newcommand{\pii}{\binddot{Π}}
\newcommand{\alam}[1]{\binddot{\msf{λ}^{#1}}}
\newcommand{\apii}[1]{\binddot{\msf{Π}^{#1}}}
\newcommand{\lett}[2]{\msf{let}\ \prn{#1}\,\msf{in}\ #2}

% applications
\newcommand{\appSym}{\cdot}
\newcommand{\app}[2]{{#1}\appSym{#2}}
\newcommand{\apptwo}[3]{#1\,#2\,#3}
\newcommand{\appthree}[4]{#1\,#2\,#3\,#4}
\newcommand{\appfour}[5]{#1\,#2\,#3\,#4\,#5}
\newcommand{\appldots}[3]{((\app{#1}{#2})\,\ldots\,{#3})}

% universes
\newcommand{\PROP}{\mtt{prop}}
\newcommand{\TYPE}{\mtt{type}}
\newcommand{\KIND}{\mtt{kind}}

% sets for syntactic categories
\newcommand{\Term}{\msf{Term}}



% -------------- Eunoia ------------------------------------
% eunoia symbols
\newcommand{\epar}[1]{\wrapp{MaterialGrey600}{\texttt{(}}{\texttt{)}}{#1}}
\newcommand{\eapp}[2]{\epar{#1\ #2}}
\newcommand{\earr}[1]{\epar{\mtt{->}\ #1}}

\newcommand{\eo}{\msf{eo}}
\newcommand{\sym}[1]{\msf{s}_{\msf{#1}}}
% \newcommand{\eoapp}[2]{\prn{#1\ #2}}
% \newcommand{\earr}[1]{\mtt{#1}}
\newcommand{\ctyp}{\mbf{ctyp}}
\newcommand{\att}{\mbf{att}}

\newcommand{\hyp}{\mbf{hyp}}
\newcommand{\step}{\mbf{step}}

% --- figures for `eoas.tex` ------------------------------
% s & {⦂}\ \sym 0 ∣ \sym 1 ∣ \ldots
% 		& \mcomment{(symbols)}         &
%
\newcommand{\eoTermSyntax}{
	\begin{array}[t]{r@{\ }l@{\quad}l@{\qquad}r@{\ }l@{\quad}l}
		%!----------------------------------------!%
		t & {⦂}\ s \mid \eapp{s}{\vec t} & \mcomment{(terms)}      &
		ρ & {⦂}\ \epar{s\ t\ \maybe{ν}}  & \mcomment{(parameters)}   \\[1mm]
	\end{array}
}

\newcommand{\eoAttrSyntax}{
	\begin{array}[t]{r@{\ }c@{\ }l@{\quad}l}
		ν & {⦂} & \mtt{:implicit} \mid \mtt{:list}                       & \mcomment{(var. attributes)}   \\[1mm]
		α & {⦂} & \mtt{:right-assoc} ∣ \mtt{:right-assoc-nil}\,\some{t}  & \mcomment{(const. attributes)} \\
		  & {∣} & \mtt{:left-assoc} ∣ \mtt{:left-assoc-nil}\,\some{t}                                     \\
		  & {∣} & \mtt{:chainable}\,\some{s} ∣ \mtt{:pairwise}\,\some{s}
	\end{array}
}

\newcommand{\eoCommandSyntax}{
	\begin{array}[t]{r@{\ }l@{\quad}l}
		r & \begin{array}[t]{c@{\ }l}
			    {⦂} & \epar{t\ t'}
		    \end{array}
		  & \mcomment{(term pairs)}
		\\
		δ & \begin{array}[t]{c@{\ }l}
			    {⦂} & \epar{\mtt{declare-const}\ s\ t\ \maybe{α}}                               \\
			    {∣} & \epar{\mtt{declare-parameterized-const}\ s\ \epar{\vec{ρ}}\ t\ \maybe{α}} \\
			    {∣} & \epar{\mtt{define}\ {s} \ \epar{\vec ρ} \ t\ \ \maybe{\mtt{:type}\,t'}}   \\
			    {∣} & \epar{\mtt{program}\ {s} \ \epar{\vec ρ}\
			    \mtt{:signature}\,\epar{\vec{t}}\,t'\
			    \epar{\vec{r}}
			    }                                                                               \\
			    {∣} & {\color{MaterialGrey600}{\texttt{(}}}
			    \mtt{declare-rule}\ s\ \epar{\vec ρ}\
			    \\ & \quad \maybe{\mtt{:premises}\ \epar{\vec{t}_{\text{prem}}}}\
			    \\ & \quad \maybe{\mtt{:args}\ \epar{\vec{t}_{\text{args}}}}\
			    \\ & \quad \maybe{\mtt{:requires}\ \epar{\vec{r}}}\
			    \\ & \quad \mtt{:conclusion}\ {t_{\text{conc}}}
			    {\color{MaterialGrey600}{\texttt{)}}}                                           \\
			    {∣} & \epar{\mtt{include}\ μ}                                                   \\
		    \end{array}
		  & \mcomment{(std. commands)}
	\end{array}
}

\newcommand{\eoPrfCommandSyntax}{
	\begin{array}[t]{r@{\ }l@{\quad}l}
		π & \begin{array}[t]{c@{\ }l}{⦂}
			        & \epar{\mtt{assume}\ s\ φ}       \\
			    % {∣} & \prn{\mtt{assume-push}\ s\ e}      \\
			    {∣} & \epar{\mtt{step}\ s\ φ\
			    \some{\mtt{:rule}\ s'}\
			    \maybe{\mtt{:premises}\ {\vec ψ}}\
			    \maybe{\mtt{:args}\ {\vec t}}}\\
			    % {∣} & \prn{\mtt{step-pop}\ s\ \maybe{e}\
			    % \prn{\mtt{:rule}\ {s_r}}\
			    % \maybe{\mtt{:premises}\ {\vec φ}}\
			    % \maybe{\mtt{:args}\ {\vec e}} }
		    \end{array}
		  & \mcomment{(proof commands)}
	\end{array}}

% --- macros for `lp.tex` ------------------------------
\newcommand{\lp}{λΠ}
\newcommand{\var}[1]{\msf{v}_{#1}}
\newcommand{\lpTermSyntax}{
	\begin{array}[t]{r@{\ }l@{\quad}r}
		% ------------------------------------- %
		u & {⦂}\ \TYPE ∣ \KIND
		  & \mcomment{(universes)}                                                             \\
		% ------------------------------------- %
		t & {⦂}\ x ∣ κ ∣ u ∣ \prn{\app{t}{t'}} ∣ \prn{\lam{x : t}{t'}} ∣ \prn{\pii{x : t}{t'}}
		  & \mcomment{(terms)}
		% ------------------------------------- %
	\end{array}
}

\newcommand{\lpCtxSyntax}{
	\begin{array}[t]{r@{\ }l@{\quad}r}
		% ------------------------------------- %
		γ & {⦂}\ \prn{x : t} \mid \prn{t ↪ t'} \
		  & \mcomment{(context elements)}        \\
		% ------------------------------------- %
		Γ & {⦂}\ \any{γ}
		  & \mcomment{(contexts)}
		% ------------------------------------- %
	\end{array}
}

\newcommand{\judge}[3]{{#1} ⊢_{Σ} {#2} : {#3}}
\newcommand{\wf}[1]{\msf{wf}\,{#1}}
% typing rules
\newcommand{\typeRule}{
	\begin{prooftree}
		\hypo{\wf Γ}
		\infer1{ \judge Γ \TYPE \KIND }
	\end{prooftree}
}

\newcommand{\varRule}{
	\begin{prooftree}
		\hypo{ \wf Γ}
		\infer1[$\prn{x : t}∈ Γ$]{ \judge Γ x t }
	\end{prooftree}
}

\newcommand{\constRule}{
	\begin{prooftree}
		\hypo{\wf Γ}
		\hypo{\judge Γ t u}
		\infer2[$\prn{κ : t}∈ Σ$]{ \judge Γ κ t }
	\end{prooftree}
}

\newcommand{\prodRule}{
	\begin{prooftree}
		\hypo{\judge Γ t {\TYPE}}
		\hypo{\judge{Γ, \prn{x : t}}{t'}{u'}}
		\infer2{\judge Γ {\prn{\pii {x : t} {t'}}} {u'} }
	\end{prooftree}
}

\newcommand{\absRule}{
	\begin{prooftree}
		\hypo{\judge{Γ, \prn{x : t}} e {t'}}
		\hypo{\judge Γ {\prn{\pii {x : t} {t'}}} u}
		\infer2{\judge Γ {\prn{\lam {x : t} {e}}} {\prn{\pii {x : t} {t'}}} }
	\end{prooftree}
}

\newcommand{\appRule}{
	\begin{prooftree}
		\hypo{\judge Γ e {\prn{\pii{x : t}{t'}}}}
		\hypo{\judge Γ {e'} t}
		\infer2{\judge Γ {\prn{\app{e}{e'}}} {t'[x ↦ {e'}]}}
	\end{prooftree}
}

\newcommand{\convRule}{
	\begin{prooftree}
		\hypo{\judge Γ e t}
		\hypo{\judge Γ {t'} {u}}
		\infer2[$\prn{t ≡_{Σ} t'}$]{\judge Γ {e} {t'}}
	\end{prooftree}
}

\newcommand{\wfRuleEmp}{
	\begin{prooftree}
		\infer0{ \wf{\msf{∅}} }
	\end{prooftree}
}

\newcommand{\wfRuleVar}{
	\begin{prooftree}
		\hypo{ \judge Γ t u }
		\infer1{ \wf{(Γ, \prn{x : t})}}
	\end{prooftree}
}

\newcommand{\rname}[1]{\mcomment{($\textsc{#1}$)}}

\newcommand{\lpTypingRules}{
	\begin{array}[t]{c}
		\rname{typ} \ {⦂}\ \  \typeRule  \\[4mm]
		\rname{con} \ {⦂}\ \  \constRule \\[4mm]
		\rname{var} \ {⦂}\ \  \varRule   \\[4mm]
		\rname{prod} \ {⦂}\ \  \prodRule \\[4mm]
		\rname{fun} \ {⦂}\ \  \absRule   \\[4mm]
		\rname{app} \ {⦂}\ \  \appRule   \\[4mm]
		\rname{conv} \ {⦂}\ \  \convRule \\[4mm]
	\end{array}
}

\newcommand{\lpWfRules}{
	\begin{array}[t]{c@{\qquad}c}
		\rname{wf0} \ \ {⦂}\ \  \wfRuleEmp
		 & \rname{wf+} \ \ {⦂}\ \  \wfRuleVar
	\end{array}
}


% --- concrete syntax for lambdapi
\newcommand{\lambdapi}{\textsc{LambdaPi}}
\newcommand{\llam}{\binddot{\msf{λ}}}
\newcommand{\ppii}{\binddot{\msf{Π}}}

\newcommand{\prm}[1]{\mbf{#1}}
\newcommand{\xsym}[1]{\mtt{@}{#1}}
\newcommand{\meta}[1]{\ptn{?{#1}}}

\newcommand{\ptrn}[1]{\ptn{\$#1}}
\newcommand{\pv}{\mbf{pvars}}

\newcommand{\imp}[1]{[#1]}
\newcommand{\xpl}[1]{\left(#1\right)}

\newcommand{\lpTermsConcrete}{
	\begin{array}[t]{r@{\ }l@{\quad}r}
		ρ & {⦂}\,\xpl{s : t} ∣ \imp{s : t}
		  & \mcomment{(parameters)}
		\\[1mm]
		t & {⦂}\ s ∣ \imp{t} ∣ \prn{\app{t}{t'}}
		∣ \prn{\llam{ρ}{t}} ∣ \prn{\ppii{ρ}{t}}
		  & \mcomment{(terms)}
	\end{array}
}


\newcommand{\lpPatternSyntax}{
	\begin{array}[t]{r@{\ }l@{\quad}r}
		% ------------------------------------- %
		θ & {⦂}\ \ptrn{x} ∣ s\,\any{θ}
		  & \mcomment{(patterns)}
		\\[1mm]
		r & {⦂}\ \prn{{s\,\any{θ}} ↪ {θ'}}
		  & \mcomment{(rewrite rules)}
		% ------------------------------------- %
	\end{array}
}

\newcommand{\lpConcreteSyntax}{
	\begin{array}[t]{r@{\ }c@{\ }l@{\quad}r}
		m & {⦂}
		  & \mtt{constant} ∣ \mtt{sequential} ∣ \mtt{injective}
		  & \mcomment{(modifiers)}
		\\[1mm]
		c & {⦂}
		  & \maybe{m}\, \mtt{symbol}\ s \ \any{ρ} \,{: t} \ \maybe{≔ t'};
		  & \mcomment{(commands)}
		\\
		  & {∣}
		  & \mtt{rule}\ r\ \any{\mtt{with}\ r'};
		\\
		  & {∣}
		  & \mtt{require}\,\mtt{open}\,\many{μ};
	\end{array}}

% \newcommand{\symb}[1]{\mtt{symbol}\ {#1};}

% \newcommand{\symb}[1]{\mtt{rule}\ {#1};}

\newcommand{\Set}{\msf{Set}}
\newcommand{\El}[1]{\msf{El}\,{#1}}
\newcommand{\Ell}[1]{\msf{El}\,\prn{#1}}


\title{%
	Automatically Translating Proof Systems
	for SMT Solvers to the λΠ-calculus}
\author{Ciarán Dunne, Guillaume Burel\inst{2,3}}
\institute{INRIA, ENS Paris-Saclay
  \and ensIIE
\and SAMOVAR, Télécom SudParis, Institut Polytechnique de Paris, 91120
Palaiseau, France}

\begin{document}
\maketitle
% we're not just translating proofs, but also the proof systems embedded in Eunoia.

\begin{abstract}
	\noindent
	Eunoia is a logical framework used for
	theories, inference rules, and derivations of SMT solvers with
	proof production facilities.
	%
	We present a tool \texttt{eo2lp}, for automatically translating Eunoia
	specifications and proofs into code for \texttt{lambdapi} ---
	an interactive theorem prover based on the $\lambda\Pi$-calculus.
	%
	Our approach is demonstrated by
	(a) translating the portion of \texttt{cvc5}'s
	\textit{co-operating proof calculus} (CPC) corresponding to the QF-UF
	fragment of SMT-LIB; and
	%
	(b) translating proofs produced by running \texttt{cvc5} on a set of
	QF-UF problems from the SMT-LIB benchmark library.
\end{abstract}

\section{Background}
\documentclass[class=llncs, crop=false]{standalone}
\documentclass{llncs}

\usepackage{unicode-math}
\usepackage{newcomputermodern}
\usepackage{microtype}

\usepackage{xcolor-material}
\usepackage{hyperref}
\usepackage{ebproof}
\usepackage{array}

\usepackage{jbw-boxfigure}
\usepackage{jbw-fix-hyperref-autoref}

\title{%
	Automatically Translating Proof Systems
	for SMT Solvers to the λΠ-calculus}
\author{Ciarán Dunne, Guillaume Burel}
\institute{INRIA, ENS Paris-Saclay}

\usepackage[backend=biber]{biblatex}
\addbibresource{inria.bib}

% font faces
\newcommand{\msf}{\mathsf}
\newcommand{\mbf}{\mathbf}
\newcommand{\mscr}{\mathscr}
\newcommand{\mtt}[1]{{\color{MaterialPink500}{\texttt{#1}}}}
\newcommand{\ptn}[1]{{\color{MaterialPurple500}{\texttt{#1}}}}
\newcommand{\mcomment}[1]{{\color{MaterialIndigo500}{\text{#1}}}}
\newcommand{\wrapp}[4]{{\color{#1}{#2}}{#4}{\color{#1}{#3}}}
\newcommand{\prn}[1]{\wrapp{MaterialGrey600}{\lparen}{\rparen}{#1}}

\newcommand{\cvc}[1]{\textsc{cvc{#1}}}
\newcommand{\verit}{\textsc{veriT}}

% ----- abstract syntax ----------
% plurality, binding
\newcommand{\plur}[2]{{#1}_{1}\ldots{#1}_{#2}}
\newcommand{\plurcomma}[2]{{#1}_{1}, \ldots, {#1}_{#2}}
\newcommand{\maybe}[1]{\wrapp{MaterialGrey600}{\langle}{\rangle_?}{#1}}
\newcommand{\many}[1]{\wrapp{MaterialGrey600}{\langle}{\rangle_+}{#1}}
\newcommand{\any}[1]{\wrapp{MaterialGrey600}{\langle}{\rangle_∗}{#1}}
\newcommand{\some}[1]{\wrapp{MaterialGrey600}{\langle}{\rangle}{#1}}
\newcommand{\binddot}[3]{{#1\,#2.\,#3}}

% binders
\newcommand{\lam}{\binddot{λ}}
\newcommand{\pii}{\binddot{Π}}
\newcommand{\alam}[1]{\binddot{\msf{λ}^{#1}}}
\newcommand{\apii}[1]{\binddot{\msf{Π}^{#1}}}
\newcommand{\lett}[2]{\msf{let}\ \prn{#1}\,\msf{in}\ #2}

% applications
\newcommand{\appSym}{\cdot}
\newcommand{\app}[2]{{#1}\appSym{#2}}
\newcommand{\apptwo}[3]{#1\,#2\,#3}
\newcommand{\appthree}[4]{#1\,#2\,#3\,#4}
\newcommand{\appfour}[5]{#1\,#2\,#3\,#4\,#5}
\newcommand{\appldots}[3]{((\app{#1}{#2})\,\ldots\,{#3})}

% universes
\newcommand{\PROP}{\mtt{prop}}
\newcommand{\TYPE}{\mtt{type}}
\newcommand{\KIND}{\mtt{kind}}

% sets for syntactic categories
\newcommand{\Term}{\msf{Term}}



% -------------- Eunoia ------------------------------------
% eunoia symbols
\newcommand{\epar}[1]{\wrapp{MaterialGrey600}{\texttt{(}}{\texttt{)}}{#1}}
\newcommand{\eapp}[2]{\epar{#1\ #2}}
\newcommand{\earr}[1]{\epar{\mtt{->}\ #1}}

\newcommand{\eo}{\msf{eo}}
\newcommand{\sym}[1]{\msf{s}_{\msf{#1}}}
% \newcommand{\eoapp}[2]{\prn{#1\ #2}}
% \newcommand{\earr}[1]{\mtt{#1}}
\newcommand{\ctyp}{\mbf{ctyp}}
\newcommand{\att}{\mbf{att}}

\newcommand{\hyp}{\mbf{hyp}}
\newcommand{\step}{\mbf{step}}

% --- figures for `eoas.tex` ------------------------------
% s & {⦂}\ \sym 0 ∣ \sym 1 ∣ \ldots
% 		& \mcomment{(symbols)}         &
%
\newcommand{\eoTermSyntax}{
	\begin{array}[t]{r@{\ }l@{\quad}l@{\qquad}r@{\ }l@{\quad}l}
		%!----------------------------------------!%
		t & {⦂}\ s \mid \eapp{s}{\vec t} & \mcomment{(terms)}      &
		ρ & {⦂}\ \epar{s\ t\ \maybe{ν}}  & \mcomment{(parameters)}   \\[1mm]
	\end{array}
}

\newcommand{\eoAttrSyntax}{
	\begin{array}[t]{r@{\ }c@{\ }l@{\quad}l}
		ν & {⦂} & \mtt{:implicit} \mid \mtt{:list}                       & \mcomment{(var. attributes)}   \\[1mm]
		α & {⦂} & \mtt{:right-assoc} ∣ \mtt{:right-assoc-nil}\,\some{t}  & \mcomment{(const. attributes)} \\
		  & {∣} & \mtt{:left-assoc} ∣ \mtt{:left-assoc-nil}\,\some{t}                                     \\
		  & {∣} & \mtt{:chainable}\,\some{s} ∣ \mtt{:pairwise}\,\some{s}
	\end{array}
}

\newcommand{\eoCommandSyntax}{
	\begin{array}[t]{r@{\ }l@{\quad}l}
		r & \begin{array}[t]{c@{\ }l}
			    {⦂} & \epar{t\ t'}
		    \end{array}
		  & \mcomment{(term pairs)}
		\\
		δ & \begin{array}[t]{c@{\ }l}
			    {⦂} & \epar{\mtt{declare-const}\ s\ t\ \maybe{α}}                               \\
			    {∣} & \epar{\mtt{declare-parameterized-const}\ s\ \epar{\vec{ρ}}\ t\ \maybe{α}} \\
			    {∣} & \epar{\mtt{define}\ {s} \ \epar{\vec ρ} \ t\ \ \maybe{\mtt{:type}\,t'}}   \\
			    {∣} & \epar{\mtt{program}\ {s} \ \epar{\vec ρ}\
			    \mtt{:signature}\,\epar{\vec{t}}\,t'\
			    \epar{\vec{r}}
			    }                                                                               \\
			    {∣} & {\color{MaterialGrey600}{\texttt{(}}}
			    \mtt{declare-rule}\ s\ \epar{\vec ρ}\
			    \\ & \quad \maybe{\mtt{:premises}\ \epar{\vec{t}_{\text{prem}}}}\
			    \\ & \quad \maybe{\mtt{:args}\ \epar{\vec{t}_{\text{args}}}}\
			    \\ & \quad \maybe{\mtt{:requires}\ \epar{\vec{r}}}\
			    \\ & \quad \mtt{:conclusion}\ {t_{\text{conc}}}
			    {\color{MaterialGrey600}{\texttt{)}}}                                           \\
			    {∣} & \epar{\mtt{include}\ μ}                                                   \\
		    \end{array}
		  & \mcomment{(std. commands)}
	\end{array}
}

\newcommand{\eoPrfCommandSyntax}{
	\begin{array}[t]{r@{\ }l@{\quad}l}
		π & \begin{array}[t]{c@{\ }l}{⦂}
			        & \epar{\mtt{assume}\ s\ φ}       \\
			    % {∣} & \prn{\mtt{assume-push}\ s\ e}      \\
			    {∣} & \epar{\mtt{step}\ s\ φ\
			    \some{\mtt{:rule}\ s'}\
			    \maybe{\mtt{:premises}\ {\vec ψ}}\
			    \maybe{\mtt{:args}\ {\vec t}}}\\
			    % {∣} & \prn{\mtt{step-pop}\ s\ \maybe{e}\
			    % \prn{\mtt{:rule}\ {s_r}}\
			    % \maybe{\mtt{:premises}\ {\vec φ}}\
			    % \maybe{\mtt{:args}\ {\vec e}} }
		    \end{array}
		  & \mcomment{(proof commands)}
	\end{array}}

% --- macros for `lp.tex` ------------------------------
\newcommand{\lp}{λΠ}
\newcommand{\var}[1]{\msf{v}_{#1}}
\newcommand{\lpTermSyntax}{
	\begin{array}[t]{r@{\ }l@{\quad}r}
		% ------------------------------------- %
		u & {⦂}\ \TYPE ∣ \KIND
		  & \mcomment{(universes)}                                                             \\
		% ------------------------------------- %
		t & {⦂}\ x ∣ κ ∣ u ∣ \prn{\app{t}{t'}} ∣ \prn{\lam{x : t}{t'}} ∣ \prn{\pii{x : t}{t'}}
		  & \mcomment{(terms)}
		% ------------------------------------- %
	\end{array}
}

\newcommand{\lpCtxSyntax}{
	\begin{array}[t]{r@{\ }l@{\quad}r}
		% ------------------------------------- %
		γ & {⦂}\ \prn{x : t} \mid \prn{t ↪ t'} \
		  & \mcomment{(context elements)}        \\
		% ------------------------------------- %
		Γ & {⦂}\ \any{γ}
		  & \mcomment{(contexts)}
		% ------------------------------------- %
	\end{array}
}

\newcommand{\judge}[3]{{#1} ⊢_{Σ} {#2} : {#3}}
\newcommand{\wf}[1]{\msf{wf}\,{#1}}
% typing rules
\newcommand{\typeRule}{
	\begin{prooftree}
		\hypo{\wf Γ}
		\infer1{ \judge Γ \TYPE \KIND }
	\end{prooftree}
}

\newcommand{\varRule}{
	\begin{prooftree}
		\hypo{ \wf Γ}
		\infer1[$\prn{x : t}∈ Γ$]{ \judge Γ x t }
	\end{prooftree}
}

\newcommand{\constRule}{
	\begin{prooftree}
		\hypo{\wf Γ}
		\hypo{\judge Γ t u}
		\infer2[$\prn{κ : t}∈ Σ$]{ \judge Γ κ t }
	\end{prooftree}
}

\newcommand{\prodRule}{
	\begin{prooftree}
		\hypo{\judge Γ t {\TYPE}}
		\hypo{\judge{Γ, \prn{x : t}}{t'}{u'}}
		\infer2{\judge Γ {\prn{\pii {x : t} {t'}}} {u'} }
	\end{prooftree}
}

\newcommand{\absRule}{
	\begin{prooftree}
		\hypo{\judge{Γ, \prn{x : t}} e {t'}}
		\hypo{\judge Γ {\prn{\pii {x : t} {t'}}} u}
		\infer2{\judge Γ {\prn{\lam {x : t} {e}}} {\prn{\pii {x : t} {t'}}} }
	\end{prooftree}
}

\newcommand{\appRule}{
	\begin{prooftree}
		\hypo{\judge Γ e {\prn{\pii{x : t}{t'}}}}
		\hypo{\judge Γ {e'} t}
		\infer2{\judge Γ {\prn{\app{e}{e'}}} {t'[x ↦ {e'}]}}
	\end{prooftree}
}

\newcommand{\convRule}{
	\begin{prooftree}
		\hypo{\judge Γ e t}
		\hypo{\judge Γ {t'} {u}}
		\infer2[$\prn{t ≡_{Σ} t'}$]{\judge Γ {e} {t'}}
	\end{prooftree}
}

\newcommand{\wfRuleEmp}{
	\begin{prooftree}
		\infer0{ \wf{\msf{∅}} }
	\end{prooftree}
}

\newcommand{\wfRuleVar}{
	\begin{prooftree}
		\hypo{ \judge Γ t u }
		\infer1{ \wf{(Γ, \prn{x : t})}}
	\end{prooftree}
}

\newcommand{\rname}[1]{\mcomment{($\textsc{#1}$)}}

\newcommand{\lpTypingRules}{
	\begin{array}[t]{c}
		\rname{typ} \ {⦂}\ \  \typeRule  \\[4mm]
		\rname{con} \ {⦂}\ \  \constRule \\[4mm]
		\rname{var} \ {⦂}\ \  \varRule   \\[4mm]
		\rname{prod} \ {⦂}\ \  \prodRule \\[4mm]
		\rname{fun} \ {⦂}\ \  \absRule   \\[4mm]
		\rname{app} \ {⦂}\ \  \appRule   \\[4mm]
		\rname{conv} \ {⦂}\ \  \convRule \\[4mm]
	\end{array}
}

\newcommand{\lpWfRules}{
	\begin{array}[t]{c@{\qquad}c}
		\rname{wf0} \ \ {⦂}\ \  \wfRuleEmp
		 & \rname{wf+} \ \ {⦂}\ \  \wfRuleVar
	\end{array}
}


% --- concrete syntax for lambdapi
\newcommand{\lambdapi}{\textsc{LambdaPi}}
\newcommand{\llam}{\binddot{\msf{λ}}}
\newcommand{\ppii}{\binddot{\msf{Π}}}

\newcommand{\prm}[1]{\mbf{#1}}
\newcommand{\xsym}[1]{\mtt{@}{#1}}
\newcommand{\meta}[1]{\ptn{?{#1}}}

\newcommand{\ptrn}[1]{\ptn{\$#1}}
\newcommand{\pv}{\mbf{pvars}}

\newcommand{\imp}[1]{[#1]}
\newcommand{\xpl}[1]{\left(#1\right)}

\newcommand{\lpTermsConcrete}{
	\begin{array}[t]{r@{\ }l@{\quad}r}
		ρ & {⦂}\,\xpl{s : t} ∣ \imp{s : t}
		  & \mcomment{(parameters)}
		\\[1mm]
		t & {⦂}\ s ∣ \imp{t} ∣ \prn{\app{t}{t'}}
		∣ \prn{\llam{ρ}{t}} ∣ \prn{\ppii{ρ}{t}}
		  & \mcomment{(terms)}
	\end{array}
}


\newcommand{\lpPatternSyntax}{
	\begin{array}[t]{r@{\ }l@{\quad}r}
		% ------------------------------------- %
		θ & {⦂}\ \ptrn{x} ∣ s\,\any{θ}
		  & \mcomment{(patterns)}
		\\[1mm]
		r & {⦂}\ \prn{{s\,\any{θ}} ↪ {θ'}}
		  & \mcomment{(rewrite rules)}
		% ------------------------------------- %
	\end{array}
}

\newcommand{\lpConcreteSyntax}{
	\begin{array}[t]{r@{\ }c@{\ }l@{\quad}r}
		m & {⦂}
		  & \mtt{constant} ∣ \mtt{sequential} ∣ \mtt{injective}
		  & \mcomment{(modifiers)}
		\\[1mm]
		c & {⦂}
		  & \maybe{m}\, \mtt{symbol}\ s \ \any{ρ} \,{: t} \ \maybe{≔ t'};
		  & \mcomment{(commands)}
		\\
		  & {∣}
		  & \mtt{rule}\ r\ \any{\mtt{with}\ r'};
		\\
		  & {∣}
		  & \mtt{require}\,\mtt{open}\,\many{μ};
	\end{array}}

% \newcommand{\symb}[1]{\mtt{symbol}\ {#1};}

% \newcommand{\symb}[1]{\mtt{rule}\ {#1};}

\newcommand{\Set}{\msf{Set}}
\newcommand{\El}[1]{\msf{El}\,{#1}}
\newcommand{\Ell}[1]{\msf{El}\,\prn{#1}}


\begin{document}
%
The area of automated reasoning research known as
\emph{satisfiability modulo theories} (SMT)
aims to develop tools capable of deciding
the satisfiability of logical specifications
within a curated selection of mathematical theories~\cite{Barrett2021}.
%
% Whilst many  accomplish this task
% with similar strategies, the approach employed by
% a solver can vary, meaning that different solvers
% will excel in different theories.
%
Such tools (known as SMT \emph{solvers}) are increasingly used as
back-ends for various tasks, in particular for hardware verification
\cite{DBLP:conf/fmcad/MattareiMBDHH18}, for program verification
\cite{barnett06boogie,riant23debugging,kosmatov16frama-c}, for model
checking \cite{DBLP:conf/cav/ChampionMST16}, to increase automation in
proof assistants
\cite{DBLP:conf/cpp/ArmandFGKTW11,DBLP:journals/jar/BlanchetteBP13},
and to check the security of access policies \cite{Backes2018}. Most
of theses applications require a high level of confidence in the
answers produced by the solver. For this purpose, the specification of
the proof system of the solvers needs to be clearly formalized, and
one should be able to check their output using trusted tools.




Towards the aim of standardizing and benchmarking
 SMT solvers,
the \emph{SMT library initiative} oversee the development of
the \emph{SMT-LIB standard} \cite{Barrett2015-standard}.
%
The standard specifies a common language used
for interacting with solvers, including a detailed
description of the mathematical foundations of the SMT problem.

% Theories are typically first-order.
% Since SMT is an extension of the SAT problem,
% SMT goes beyond the traditional SAT problem, logical
% specifications are first-order logic.

% SMT solvers exist: cvc5, veriT, z3.
% SMT-LIB standard exists:
%   2.6 (many-sorted FOL)
%   2.7 (they added lambdas??)
%   3.0 proposal -
%     near to Eunoia but very different.
%     SMT3 actually allows higher-order SMT problems,
%     Eunoia just uses higher-order stuff for rules, programs, type definitions, etc.
%

% some mention of SMT solvers used by Isabelle, Coq, Lean, etc.

% proof production
For specifications deemed unsatisfiable, many solvers
can generate \emph{proof certificates} that demonstrate
the absurdity of the assertions made by the specification.
%
Proof generation is however not covered by the SMT-LIB
standard, which has led to the development of various proof
formats for SMT solvers.
%%
One of the first attempt to output proofs was done by the solver CVC3
and its successors. They can produce proofs in the LFSC format
\cite{DBLP:journals/fmsd/StumpORHT13}. LFSC extends the Edinburgh
Logical Framework (LF) with side conditions, programs that are
executed to restrict applications of some inference rules. However,
several proof rules of CVC5 are not supported by this back-end, and
such steps are given as axioms and need to be checked. Another
limitation is that the rules that are used are specific to CVC5, and
do not necessarily exists in this form in other solvers.
%
This need for a more interoperable proof format lead to design of the
\emph{Alethe} format \cite{schurrAletheGenericSMT2021}.  Alethe draws from a fixed set of rules designed
to reflect the reasoning mechanisms of SMT solvers.
%
The common proof interface provided by Alethe has
enabled interoperability between solvers and other
automated reasoning tools;
particularly \emph{proof assistants}, where it is used for the
reconstruction of
proofs obtain automatically
\cite{DBLP:conf/cpp/ArmandFGKTW11,DBLP:conf/cade/SchurrFD21}. Among solvers
that output proofs
in the Alethe format, one can cite VeriT
\cite{DBLP:conf/cade/BoutonODF09} and CVC5 \cite{Barbosa2022}.
%
Proofs in the Alethe format can be checked by the tool
\textit{Carcara} \cite{andreottiCarcaraEfficientProof2023}.
Unfortunately, long-term interoperability can be
challenging, as developers must rewrite aspects
of their tools to maintain parity with the evolving
Alethe specification. Furthermore, from the  point of view of the developer of a
solver, the fixed set of available inference rules of Alethe hinders
the design of new reasoning techniques.
%
\textit{Eunoia} was proposed to tackle these issues. \textit{Eunoia} is a logical framework that allows formalizing
the inference rules used by the proof production facilities
of an SMT solver. Eunoia shares aspects with the speculative proposal
for SMT-LIB 3~\cite{smt3-proposal}: it uses e.g.\ dependent types and binders.
%
Proofs in the
Eunoia format can be checked by the C++ tool
\textit{Ethos}~\cite{EthosUser_manualmdMain}. Eunoia and Ethos are
work in progress, and they are consequently still evolving.
CVC5 can output proofs in the Eunoia format. For this purpose, an
encoding of the proof calculus of CVC5, namely the Cooperating Proof
Calculus (CPC), has been implemented in Eunoia. Using this, proofs of
CVC5 can be checked by Ethos.



Carcara and Ethos are relatively small tools, so that their source can
be inspected to persuade oneself of their correction. However, one
could want to go a step further to gain another level of trust, by
translating proofs in Alethe or Eunoia format into formats that are
more mature, for instance the format of proof assistants such as Rocq
or Isabelle/HOL. Such a translation could also help with reusing the
proofs in another context, in an interoperability perspective. The
$\lambda\Pi$-calculus modulo rewriting~\cite{CousineauDowek}, as
implemented in the tools Dedukti~\cite{assaf2023dedukti} and
LambdaPi~\cite{hondet:hal-02981561}, was designed to offer a
trustworthy proof checker with an emphasis on interoperability of
proof systems. Proofs from various systems can be embedded into
Dedukti: proofs assistants such as Rocq, Isabelle, HOL Light, Matita
and Lean; automated theorem provers such as Zenon Modulo, ArchSAT and
Vampire, and more generally any prover outputting proofs in the TPTP
format; and even the programming language semantic framework
K. Conversely, some proofs in Dedukti can be exported back into Rocq,
Matita, or HOL/Light~\cite{DBLP:journals/corr/abs-1807-01873}. Such a formalism seems therefore a prime target
for translating proofs of SMT solvers.
Coltellacci~\cite{coltellacciReconstructionSMTProofs2024a} developed a
translation of Alethe proofs into LambdaPi, by adding a new back-end
in the Carcara tool. Roughly, to each inference rule of the Alethe
specification corresponds a lemma in LambdaPi, and lemmas are combined
to reconstruct the proof.

As Eunoia is a logical framework, the set of rules is not fixed and
cannot be implemented once for all in LambdaPi. To be
able to translate Eunoia proofs into Dedukti or LambdaPi, one need a
way to encode also how inference rules are defined in Eunoia, in order
to be as generic as it. This is the purpose of this paper, together
with the actual translation of SMT proofs from CVC5 into LambdaPi.

In the next section, we present Eunoia, Ethos and
CPC. Section~\ref{sec:lambdapi} defines the $\lambda\Pi$-calculus
modulo rewriting and its implementation in LambdaPi. The translation
from Eunoia to LambdaPi is given in Section~\ref{sec:results}, as well
as actual results. We conclude in Section~\ref{sec:conclu} and discuss
future work.


% % eunoia is (similar to)/(inspired by) by the speculative
% % proposal for SMT-LIB 3.
% SMT-LIB is a specification language used for interacting
% with SMT solvers. The syntax and semantics of SMT-LIB are
% standardized by the
% SMT-LIB Initiative~\autocite{Barrett2015-standard}).
% %
% Generally speaking, SMT-LIB scripts specify satisfiability
% problems for many-sorted first-order logic (MFOL) and
% provide instructions for solvers to interact with these
% specifications.
%
\end{document}


\section{Eunoia}\label{sec:eunoia}
\documentclass[class=llncs, crop=false]{standalone}

\documentclass{llncs}

\usepackage{unicode-math}
\usepackage{newcomputermodern}
\usepackage{microtype}

\usepackage{xcolor-material}
\usepackage{hyperref}
\usepackage{ebproof}
\usepackage{array}

\usepackage{jbw-boxfigure}
\usepackage{jbw-fix-hyperref-autoref}

\title{%
	Automatically Translating Proof Systems
	for SMT Solvers to the λΠ-calculus}
\author{Ciarán Dunne, Guillaume Burel}
\institute{INRIA, ENS Paris-Saclay}

\usepackage[backend=biber]{biblatex}
\addbibresource{inria.bib}

% font faces
\newcommand{\msf}{\mathsf}
\newcommand{\mbf}{\mathbf}
\newcommand{\mscr}{\mathscr}
\newcommand{\mtt}[1]{{\color{MaterialPink500}{\texttt{#1}}}}
\newcommand{\ptn}[1]{{\color{MaterialPurple500}{\texttt{#1}}}}
\newcommand{\mcomment}[1]{{\color{MaterialIndigo500}{\text{#1}}}}
\newcommand{\wrapp}[4]{{\color{#1}{#2}}{#4}{\color{#1}{#3}}}
\newcommand{\prn}[1]{\wrapp{MaterialGrey600}{\lparen}{\rparen}{#1}}

\newcommand{\cvc}[1]{\textsc{cvc{#1}}}
\newcommand{\verit}{\textsc{veriT}}

% ----- abstract syntax ----------
% plurality, binding
\newcommand{\plur}[2]{{#1}_{1}\ldots{#1}_{#2}}
\newcommand{\plurcomma}[2]{{#1}_{1}, \ldots, {#1}_{#2}}
\newcommand{\maybe}[1]{\wrapp{MaterialGrey600}{\langle}{\rangle_?}{#1}}
\newcommand{\many}[1]{\wrapp{MaterialGrey600}{\langle}{\rangle_+}{#1}}
\newcommand{\any}[1]{\wrapp{MaterialGrey600}{\langle}{\rangle_∗}{#1}}
\newcommand{\some}[1]{\wrapp{MaterialGrey600}{\langle}{\rangle}{#1}}
\newcommand{\binddot}[3]{{#1\,#2.\,#3}}

% binders
\newcommand{\lam}{\binddot{λ}}
\newcommand{\pii}{\binddot{Π}}
\newcommand{\alam}[1]{\binddot{\msf{λ}^{#1}}}
\newcommand{\apii}[1]{\binddot{\msf{Π}^{#1}}}
\newcommand{\lett}[2]{\msf{let}\ \prn{#1}\,\msf{in}\ #2}

% applications
\newcommand{\appSym}{\cdot}
\newcommand{\app}[2]{{#1}\appSym{#2}}
\newcommand{\apptwo}[3]{#1\,#2\,#3}
\newcommand{\appthree}[4]{#1\,#2\,#3\,#4}
\newcommand{\appfour}[5]{#1\,#2\,#3\,#4\,#5}
\newcommand{\appldots}[3]{((\app{#1}{#2})\,\ldots\,{#3})}

% universes
\newcommand{\PROP}{\mtt{prop}}
\newcommand{\TYPE}{\mtt{type}}
\newcommand{\KIND}{\mtt{kind}}

% sets for syntactic categories
\newcommand{\Term}{\msf{Term}}



% -------------- Eunoia ------------------------------------
% eunoia symbols
\newcommand{\epar}[1]{\wrapp{MaterialGrey600}{\texttt{(}}{\texttt{)}}{#1}}
\newcommand{\eapp}[2]{\epar{#1\ #2}}
\newcommand{\earr}[1]{\epar{\mtt{->}\ #1}}

\newcommand{\eo}{\msf{eo}}
\newcommand{\sym}[1]{\msf{s}_{\msf{#1}}}
% \newcommand{\eoapp}[2]{\prn{#1\ #2}}
% \newcommand{\earr}[1]{\mtt{#1}}
\newcommand{\ctyp}{\mbf{ctyp}}
\newcommand{\att}{\mbf{att}}

\newcommand{\hyp}{\mbf{hyp}}
\newcommand{\step}{\mbf{step}}

% --- figures for `eoas.tex` ------------------------------
% s & {⦂}\ \sym 0 ∣ \sym 1 ∣ \ldots
% 		& \mcomment{(symbols)}         &
%
\newcommand{\eoTermSyntax}{
	\begin{array}[t]{r@{\ }l@{\quad}l@{\qquad}r@{\ }l@{\quad}l}
		%!----------------------------------------!%
		t & {⦂}\ s \mid \eapp{s}{\vec t} & \mcomment{(terms)}      &
		ρ & {⦂}\ \epar{s\ t\ \maybe{ν}}  & \mcomment{(parameters)}   \\[1mm]
	\end{array}
}

\newcommand{\eoAttrSyntax}{
	\begin{array}[t]{r@{\ }c@{\ }l@{\quad}l}
		ν & {⦂} & \mtt{:implicit} \mid \mtt{:list}                       & \mcomment{(var. attributes)}   \\[1mm]
		α & {⦂} & \mtt{:right-assoc} ∣ \mtt{:right-assoc-nil}\,\some{t}  & \mcomment{(const. attributes)} \\
		  & {∣} & \mtt{:left-assoc} ∣ \mtt{:left-assoc-nil}\,\some{t}                                     \\
		  & {∣} & \mtt{:chainable}\,\some{s} ∣ \mtt{:pairwise}\,\some{s}
	\end{array}
}

\newcommand{\eoCommandSyntax}{
	\begin{array}[t]{r@{\ }l@{\quad}l}
		r & \begin{array}[t]{c@{\ }l}
			    {⦂} & \epar{t\ t'}
		    \end{array}
		  & \mcomment{(term pairs)}
		\\
		δ & \begin{array}[t]{c@{\ }l}
			    {⦂} & \epar{\mtt{declare-const}\ s\ t\ \maybe{α}}                               \\
			    {∣} & \epar{\mtt{declare-parameterized-const}\ s\ \epar{\vec{ρ}}\ t\ \maybe{α}} \\
			    {∣} & \epar{\mtt{define}\ {s} \ \epar{\vec ρ} \ t\ \ \maybe{\mtt{:type}\,t'}}   \\
			    {∣} & \epar{\mtt{program}\ {s} \ \epar{\vec ρ}\
			    \mtt{:signature}\,\epar{\vec{t}}\,t'\
			    \epar{\vec{r}}
			    }                                                                               \\
			    {∣} & {\color{MaterialGrey600}{\texttt{(}}}
			    \mtt{declare-rule}\ s\ \epar{\vec ρ}\
			    \\ & \quad \maybe{\mtt{:premises}\ \epar{\vec{t}_{\text{prem}}}}\
			    \\ & \quad \maybe{\mtt{:args}\ \epar{\vec{t}_{\text{args}}}}\
			    \\ & \quad \maybe{\mtt{:requires}\ \epar{\vec{r}}}\
			    \\ & \quad \mtt{:conclusion}\ {t_{\text{conc}}}
			    {\color{MaterialGrey600}{\texttt{)}}}                                           \\
			    {∣} & \epar{\mtt{include}\ μ}                                                   \\
		    \end{array}
		  & \mcomment{(std. commands)}
	\end{array}
}

\newcommand{\eoPrfCommandSyntax}{
	\begin{array}[t]{r@{\ }l@{\quad}l}
		π & \begin{array}[t]{c@{\ }l}{⦂}
			        & \epar{\mtt{assume}\ s\ φ}       \\
			    % {∣} & \prn{\mtt{assume-push}\ s\ e}      \\
			    {∣} & \epar{\mtt{step}\ s\ φ\
			    \some{\mtt{:rule}\ s'}\
			    \maybe{\mtt{:premises}\ {\vec ψ}}\
			    \maybe{\mtt{:args}\ {\vec t}}}\\
			    % {∣} & \prn{\mtt{step-pop}\ s\ \maybe{e}\
			    % \prn{\mtt{:rule}\ {s_r}}\
			    % \maybe{\mtt{:premises}\ {\vec φ}}\
			    % \maybe{\mtt{:args}\ {\vec e}} }
		    \end{array}
		  & \mcomment{(proof commands)}
	\end{array}}

% --- macros for `lp.tex` ------------------------------
\newcommand{\lp}{λΠ}
\newcommand{\var}[1]{\msf{v}_{#1}}
\newcommand{\lpTermSyntax}{
	\begin{array}[t]{r@{\ }l@{\quad}r}
		% ------------------------------------- %
		u & {⦂}\ \TYPE ∣ \KIND
		  & \mcomment{(universes)}                                                             \\
		% ------------------------------------- %
		t & {⦂}\ x ∣ κ ∣ u ∣ \prn{\app{t}{t'}} ∣ \prn{\lam{x : t}{t'}} ∣ \prn{\pii{x : t}{t'}}
		  & \mcomment{(terms)}
		% ------------------------------------- %
	\end{array}
}

\newcommand{\lpCtxSyntax}{
	\begin{array}[t]{r@{\ }l@{\quad}r}
		% ------------------------------------- %
		γ & {⦂}\ \prn{x : t} \mid \prn{t ↪ t'} \
		  & \mcomment{(context elements)}        \\
		% ------------------------------------- %
		Γ & {⦂}\ \any{γ}
		  & \mcomment{(contexts)}
		% ------------------------------------- %
	\end{array}
}

\newcommand{\judge}[3]{{#1} ⊢_{Σ} {#2} : {#3}}
\newcommand{\wf}[1]{\msf{wf}\,{#1}}
% typing rules
\newcommand{\typeRule}{
	\begin{prooftree}
		\hypo{\wf Γ}
		\infer1{ \judge Γ \TYPE \KIND }
	\end{prooftree}
}

\newcommand{\varRule}{
	\begin{prooftree}
		\hypo{ \wf Γ}
		\infer1[$\prn{x : t}∈ Γ$]{ \judge Γ x t }
	\end{prooftree}
}

\newcommand{\constRule}{
	\begin{prooftree}
		\hypo{\wf Γ}
		\hypo{\judge Γ t u}
		\infer2[$\prn{κ : t}∈ Σ$]{ \judge Γ κ t }
	\end{prooftree}
}

\newcommand{\prodRule}{
	\begin{prooftree}
		\hypo{\judge Γ t {\TYPE}}
		\hypo{\judge{Γ, \prn{x : t}}{t'}{u'}}
		\infer2{\judge Γ {\prn{\pii {x : t} {t'}}} {u'} }
	\end{prooftree}
}

\newcommand{\absRule}{
	\begin{prooftree}
		\hypo{\judge{Γ, \prn{x : t}} e {t'}}
		\hypo{\judge Γ {\prn{\pii {x : t} {t'}}} u}
		\infer2{\judge Γ {\prn{\lam {x : t} {e}}} {\prn{\pii {x : t} {t'}}} }
	\end{prooftree}
}

\newcommand{\appRule}{
	\begin{prooftree}
		\hypo{\judge Γ e {\prn{\pii{x : t}{t'}}}}
		\hypo{\judge Γ {e'} t}
		\infer2{\judge Γ {\prn{\app{e}{e'}}} {t'[x ↦ {e'}]}}
	\end{prooftree}
}

\newcommand{\convRule}{
	\begin{prooftree}
		\hypo{\judge Γ e t}
		\hypo{\judge Γ {t'} {u}}
		\infer2[$\prn{t ≡_{Σ} t'}$]{\judge Γ {e} {t'}}
	\end{prooftree}
}

\newcommand{\wfRuleEmp}{
	\begin{prooftree}
		\infer0{ \wf{\msf{∅}} }
	\end{prooftree}
}

\newcommand{\wfRuleVar}{
	\begin{prooftree}
		\hypo{ \judge Γ t u }
		\infer1{ \wf{(Γ, \prn{x : t})}}
	\end{prooftree}
}

\newcommand{\rname}[1]{\mcomment{($\textsc{#1}$)}}

\newcommand{\lpTypingRules}{
	\begin{array}[t]{c}
		\rname{typ} \ {⦂}\ \  \typeRule  \\[4mm]
		\rname{con} \ {⦂}\ \  \constRule \\[4mm]
		\rname{var} \ {⦂}\ \  \varRule   \\[4mm]
		\rname{prod} \ {⦂}\ \  \prodRule \\[4mm]
		\rname{fun} \ {⦂}\ \  \absRule   \\[4mm]
		\rname{app} \ {⦂}\ \  \appRule   \\[4mm]
		\rname{conv} \ {⦂}\ \  \convRule \\[4mm]
	\end{array}
}

\newcommand{\lpWfRules}{
	\begin{array}[t]{c@{\qquad}c}
		\rname{wf0} \ \ {⦂}\ \  \wfRuleEmp
		 & \rname{wf+} \ \ {⦂}\ \  \wfRuleVar
	\end{array}
}


% --- concrete syntax for lambdapi
\newcommand{\lambdapi}{\textsc{LambdaPi}}
\newcommand{\llam}{\binddot{\msf{λ}}}
\newcommand{\ppii}{\binddot{\msf{Π}}}

\newcommand{\prm}[1]{\mbf{#1}}
\newcommand{\xsym}[1]{\mtt{@}{#1}}
\newcommand{\meta}[1]{\ptn{?{#1}}}

\newcommand{\ptrn}[1]{\ptn{\$#1}}
\newcommand{\pv}{\mbf{pvars}}

\newcommand{\imp}[1]{[#1]}
\newcommand{\xpl}[1]{\left(#1\right)}

\newcommand{\lpTermsConcrete}{
	\begin{array}[t]{r@{\ }l@{\quad}r}
		ρ & {⦂}\,\xpl{s : t} ∣ \imp{s : t}
		  & \mcomment{(parameters)}
		\\[1mm]
		t & {⦂}\ s ∣ \imp{t} ∣ \prn{\app{t}{t'}}
		∣ \prn{\llam{ρ}{t}} ∣ \prn{\ppii{ρ}{t}}
		  & \mcomment{(terms)}
	\end{array}
}


\newcommand{\lpPatternSyntax}{
	\begin{array}[t]{r@{\ }l@{\quad}r}
		% ------------------------------------- %
		θ & {⦂}\ \ptrn{x} ∣ s\,\any{θ}
		  & \mcomment{(patterns)}
		\\[1mm]
		r & {⦂}\ \prn{{s\,\any{θ}} ↪ {θ'}}
		  & \mcomment{(rewrite rules)}
		% ------------------------------------- %
	\end{array}
}

\newcommand{\lpConcreteSyntax}{
	\begin{array}[t]{r@{\ }c@{\ }l@{\quad}r}
		m & {⦂}
		  & \mtt{constant} ∣ \mtt{sequential} ∣ \mtt{injective}
		  & \mcomment{(modifiers)}
		\\[1mm]
		c & {⦂}
		  & \maybe{m}\, \mtt{symbol}\ s \ \any{ρ} \,{: t} \ \maybe{≔ t'};
		  & \mcomment{(commands)}
		\\
		  & {∣}
		  & \mtt{rule}\ r\ \any{\mtt{with}\ r'};
		\\
		  & {∣}
		  & \mtt{require}\,\mtt{open}\,\many{μ};
	\end{array}}

% \newcommand{\symb}[1]{\mtt{symbol}\ {#1};}

% \newcommand{\symb}[1]{\mtt{rule}\ {#1};}

\newcommand{\Set}{\msf{Set}}
\newcommand{\El}[1]{\msf{El}\,{#1}}
\newcommand{\Ell}[1]{\msf{El}\,\prn{#1}}



\begin{document}
% ---------------------------------------------------------
\begin{boxfigure}[t!]{fig:eo-syntax}
	{Syntax for Eunoia}
	%
	$$ \eoTermSyntax $$
	\hrule
	$$ \eoAttrSyntax $$
	\hrule
	$$ \eoCommandSyntax $$
	% \hrule
	% $$ \eoPrfCommandSyntax $$
\end{boxfigure}
% ---------------------------------------------------------
With respect to a fixed set $𝒮_{\mbf{eo}}$ of \emph{symbols},
the rules in \autoref{fig:eo-syntax} define syntax for a
fragment of Eunoia.
%
In particular, we define sets of expressions
for \emph{terms}, \emph{parameters}, and \emph{attributes}.
%
Each term is either a symbol $s$ or an \emph{application}
$\eapp{s}{\vec t}$ for some list of terms $\vec t$.
%
% TODO. ref to explanation of variable attributes.
Let $ν$ and $α$ range over \emph{variable attributes}
and \emph{constant attributes} respectively.
%
Then, each \emph{parameter} $ρ$ consists of a symbol $s$,
a term $t$, and possibly a variable attribute $ν$.
%

Finally, the rules of \autoref{fig:eo-syntax}
define syntax for a subset of Eunoia's top-level commands,
which we will call \emph{standard commands}.
%
Hereinafter, a Eunoia \emph{signature} is a list $Δ$
of standard commands. We define some
`abstract syntax' for working with signatures below.

\subsubsection{Constant Declaration.}
%
Let $δ$ be a \emph{(parameterized) constant declaration}
with symbol $s$, parameters $\vec ρ$, and term $t$.
%
If $δ ∈ Δ$, then $\prn{\vec ρ.\, t}$ is called the
\emph{type} of $s$;
written $\prn{Δ ⊢ s : \prn{\vec ρ.\, t}}$.
%
If $δ$ provides a constant attribute $α$,
we write $\prn{Δ ⊢ s ∷ α}$.
%

\subsubsection{Macro Definition.}
%
Let $δ$ be a \emph{macro definition} with
symbol $s$, parameters $\vec ρ$, and term $t$.
%
If $δ ∈ Δ$, we call $\prn{\vec ρ.\, t}$
the \emph{definiens} of $s$;
written $\prn{Δ ⊢ s ≔ \prn{\vec ρ.\, t}}$.
%
If the attribute $\mtt{:type}\ t'$ is given by $δ$,
then $\prn{Δ ⊢ s : \prn{\vec ρ.\, t'}}$ also holds.

\subsubsection{Program Declaration.}
%
Let $δ$ be a \emph{program declaration} with
symbol $s$, parameters $\vec ρ$,
attribute $\mtt{:signature}\,\prn{\vec t}\,t'$,
and term-pairs $\vec r$.
%
If $δ ∈ Δ$ and $\vec t = \plur t n$,
we say that $s$ has \emph{domain} $(\plurcomma t n)$,
\emph{range} $t'$ and \emph{cases} $\vec r$
with respect to the parameters $\vec ρ$.
%
More formally, we write
  $\prn{Δ ⊢ s : \prn{\vec ρ.\,\earr{\plur t n\, t'}}}$
for the type of $s$,
and
  $\prn{Δ ⊢ s ≔ \mbf{cases}(\vec ρ.\ r_1, \ldots, r_n)}$
for the cases.
%
\subsubsection{Rule Declaration.}
%
Let $δ$ be a \emph{rule declaration} with symbol $s$,
parameters $\vec ρ$, and \emph{conclusion} $ψ$.
%
Suppose also that $δ$ provides attributes
$\mtt{:premises}\ \epar{\vec φ}$,
$\mtt{:args}\ \epar{\vec t}$,
and $\mtt{:requires}\ \epar{\vec r}$.
%
If $δ ∈ Δ$, then we may write:
  $$Δ ⊢ s : \prn{\vec ρ.\ \mbf{infer}(ψ ∣ \vec φ, \vec t, \vec r)}$$


\subsubsection{Signature Inclusion.}
%
We assume a \emph{(global) environment}
$Θ$ which maps filepaths to Eunoia signatures.
%
Thus, whenever $\epar{\mtt{include}\ μ} ∈ Δ$
for some path $μ$, the signature $Δ$ inherits
all of the declarations made by $Θ(μ)$.


\subsection{Elaboration of Eunoia Terms}

\subsection{Proof Scripts}
%
\begin{boxfigure}[t!]{fig:eo-prf-commands}
 {Syntax for Eunoia proof commands.}
 $$ \eoPrfCommandSyntax $$
\end{boxfigure}
%
\Autoref{fig:eo-prf-commands} provides syntax for two
basic forms of Eunoia \emph{proof commands}, which are
called  \emph{assumption} and \emph{step}.
%
Each \emph{proof script} $Π$ is a list
$\prn{\vec δ; \vec π}$ where
$\vec δ$ is a list of standard commands called
the \emph{preamble} and $\vec π$ are proof commands.
%

Given $\epar{\mtt{assume}\ s\ t} ∈ Π$,
we write $\prn{Π ⊢ s ≔ \hyp(t)}$.
%
Furthermore, let $π$ be a proof step with symbol $s$,
\emph{conclusion} $ψ$, and attributes $\mtt{:premises}\ \vec{φ}$
and $\mtt{:args}\ \vec{t}$.
%
Then we may write $\prn{Π ⊢ s ≔ \step(s', φ, \vec ψ, \vec t)}$
iff $π ∈ Π$.
% In practice, an Eunoia-friendly SMT solver will give
% its proofs with respect to some fixed signature $Δ$.
%
% We define We say that a proof scripts is \emph{valid} with respect
% to $Δ$ iff a list of
% commands of the form $\vec δ, \vec π$ proof commands.



\end{document}


% \subsection{Terms, Types and Attributes}
% %
% The \emph{built-in constants} of Eunoia are given by a fixed
% subset of symbols and denoted by the following expressions:
% $\mtt{Type}$ (the kind of all types),
% $\mtt{->}$ (arrow-type constructor),
% $\mtt{\_}$ (binary function application),
% $\mtt{Bool}$ (the type of Boolean values),
% $\mtt{true}$ and $\mtt{false}$ (terms of type \mtt{Bool}).


\section{λΠ-calculus modulo rewriting}\label{sec:lambdapi}
\documentclass[class=llncs, crop=false]{standalone}
\documentclass{llncs}

\usepackage{unicode-math}
\usepackage{newcomputermodern}
\usepackage{microtype}

\usepackage{xcolor-material}
\usepackage{hyperref}
\usepackage{ebproof}
\usepackage{array}

\usepackage{jbw-boxfigure}
\usepackage{jbw-fix-hyperref-autoref}

\title{%
	Automatically Translating Proof Systems
	for SMT Solvers to the λΠ-calculus}
\author{Ciarán Dunne, Guillaume Burel}
\institute{INRIA, ENS Paris-Saclay}

\usepackage[backend=biber]{biblatex}
\addbibresource{inria.bib}

% font faces
\newcommand{\msf}{\mathsf}
\newcommand{\mbf}{\mathbf}
\newcommand{\mscr}{\mathscr}
\newcommand{\mtt}[1]{{\color{MaterialPink500}{\texttt{#1}}}}
\newcommand{\ptn}[1]{{\color{MaterialPurple500}{\texttt{#1}}}}
\newcommand{\mcomment}[1]{{\color{MaterialIndigo500}{\text{#1}}}}
\newcommand{\wrapp}[4]{{\color{#1}{#2}}{#4}{\color{#1}{#3}}}
\newcommand{\prn}[1]{\wrapp{MaterialGrey600}{\lparen}{\rparen}{#1}}

\newcommand{\cvc}[1]{\textsc{cvc{#1}}}
\newcommand{\verit}{\textsc{veriT}}

% ----- abstract syntax ----------
% plurality, binding
\newcommand{\plur}[2]{{#1}_{1}\ldots{#1}_{#2}}
\newcommand{\plurcomma}[2]{{#1}_{1}, \ldots, {#1}_{#2}}
\newcommand{\maybe}[1]{\wrapp{MaterialGrey600}{\langle}{\rangle_?}{#1}}
\newcommand{\many}[1]{\wrapp{MaterialGrey600}{\langle}{\rangle_+}{#1}}
\newcommand{\any}[1]{\wrapp{MaterialGrey600}{\langle}{\rangle_∗}{#1}}
\newcommand{\some}[1]{\wrapp{MaterialGrey600}{\langle}{\rangle}{#1}}
\newcommand{\binddot}[3]{{#1\,#2.\,#3}}

% binders
\newcommand{\lam}{\binddot{λ}}
\newcommand{\pii}{\binddot{Π}}
\newcommand{\alam}[1]{\binddot{\msf{λ}^{#1}}}
\newcommand{\apii}[1]{\binddot{\msf{Π}^{#1}}}
\newcommand{\lett}[2]{\msf{let}\ \prn{#1}\,\msf{in}\ #2}

% applications
\newcommand{\appSym}{\cdot}
\newcommand{\app}[2]{{#1}\appSym{#2}}
\newcommand{\apptwo}[3]{#1\,#2\,#3}
\newcommand{\appthree}[4]{#1\,#2\,#3\,#4}
\newcommand{\appfour}[5]{#1\,#2\,#3\,#4\,#5}
\newcommand{\appldots}[3]{((\app{#1}{#2})\,\ldots\,{#3})}

% universes
\newcommand{\PROP}{\mtt{prop}}
\newcommand{\TYPE}{\mtt{type}}
\newcommand{\KIND}{\mtt{kind}}

% sets for syntactic categories
\newcommand{\Term}{\msf{Term}}



% -------------- Eunoia ------------------------------------
% eunoia symbols
\newcommand{\epar}[1]{\wrapp{MaterialGrey600}{\texttt{(}}{\texttt{)}}{#1}}
\newcommand{\eapp}[2]{\epar{#1\ #2}}
\newcommand{\earr}[1]{\epar{\mtt{->}\ #1}}

\newcommand{\eo}{\msf{eo}}
\newcommand{\sym}[1]{\msf{s}_{\msf{#1}}}
% \newcommand{\eoapp}[2]{\prn{#1\ #2}}
% \newcommand{\earr}[1]{\mtt{#1}}
\newcommand{\ctyp}{\mbf{ctyp}}
\newcommand{\att}{\mbf{att}}

\newcommand{\hyp}{\mbf{hyp}}
\newcommand{\step}{\mbf{step}}

% --- figures for `eoas.tex` ------------------------------
% s & {⦂}\ \sym 0 ∣ \sym 1 ∣ \ldots
% 		& \mcomment{(symbols)}         &
%
\newcommand{\eoTermSyntax}{
	\begin{array}[t]{r@{\ }l@{\quad}l@{\qquad}r@{\ }l@{\quad}l}
		%!----------------------------------------!%
		t & {⦂}\ s \mid \eapp{s}{\vec t} & \mcomment{(terms)}      &
		ρ & {⦂}\ \epar{s\ t\ \maybe{ν}}  & \mcomment{(parameters)}   \\[1mm]
	\end{array}
}

\newcommand{\eoAttrSyntax}{
	\begin{array}[t]{r@{\ }c@{\ }l@{\quad}l}
		ν & {⦂} & \mtt{:implicit} \mid \mtt{:list}                       & \mcomment{(var. attributes)}   \\[1mm]
		α & {⦂} & \mtt{:right-assoc} ∣ \mtt{:right-assoc-nil}\,\some{t}  & \mcomment{(const. attributes)} \\
		  & {∣} & \mtt{:left-assoc} ∣ \mtt{:left-assoc-nil}\,\some{t}                                     \\
		  & {∣} & \mtt{:chainable}\,\some{s} ∣ \mtt{:pairwise}\,\some{s}
	\end{array}
}

\newcommand{\eoCommandSyntax}{
	\begin{array}[t]{r@{\ }l@{\quad}l}
		r & \begin{array}[t]{c@{\ }l}
			    {⦂} & \epar{t\ t'}
		    \end{array}
		  & \mcomment{(term pairs)}
		\\
		δ & \begin{array}[t]{c@{\ }l}
			    {⦂} & \epar{\mtt{declare-const}\ s\ t\ \maybe{α}}                               \\
			    {∣} & \epar{\mtt{declare-parameterized-const}\ s\ \epar{\vec{ρ}}\ t\ \maybe{α}} \\
			    {∣} & \epar{\mtt{define}\ {s} \ \epar{\vec ρ} \ t\ \ \maybe{\mtt{:type}\,t'}}   \\
			    {∣} & \epar{\mtt{program}\ {s} \ \epar{\vec ρ}\
			    \mtt{:signature}\,\epar{\vec{t}}\,t'\
			    \epar{\vec{r}}
			    }                                                                               \\
			    {∣} & {\color{MaterialGrey600}{\texttt{(}}}
			    \mtt{declare-rule}\ s\ \epar{\vec ρ}\
			    \\ & \quad \maybe{\mtt{:premises}\ \epar{\vec{t}_{\text{prem}}}}\
			    \\ & \quad \maybe{\mtt{:args}\ \epar{\vec{t}_{\text{args}}}}\
			    \\ & \quad \maybe{\mtt{:requires}\ \epar{\vec{r}}}\
			    \\ & \quad \mtt{:conclusion}\ {t_{\text{conc}}}
			    {\color{MaterialGrey600}{\texttt{)}}}                                           \\
			    {∣} & \epar{\mtt{include}\ μ}                                                   \\
		    \end{array}
		  & \mcomment{(std. commands)}
	\end{array}
}

\newcommand{\eoPrfCommandSyntax}{
	\begin{array}[t]{r@{\ }l@{\quad}l}
		π & \begin{array}[t]{c@{\ }l}{⦂}
			        & \epar{\mtt{assume}\ s\ φ}       \\
			    % {∣} & \prn{\mtt{assume-push}\ s\ e}      \\
			    {∣} & \epar{\mtt{step}\ s\ φ\
			    \some{\mtt{:rule}\ s'}\
			    \maybe{\mtt{:premises}\ {\vec ψ}}\
			    \maybe{\mtt{:args}\ {\vec t}}}\\
			    % {∣} & \prn{\mtt{step-pop}\ s\ \maybe{e}\
			    % \prn{\mtt{:rule}\ {s_r}}\
			    % \maybe{\mtt{:premises}\ {\vec φ}}\
			    % \maybe{\mtt{:args}\ {\vec e}} }
		    \end{array}
		  & \mcomment{(proof commands)}
	\end{array}}

% --- macros for `lp.tex` ------------------------------
\newcommand{\lp}{λΠ}
\newcommand{\var}[1]{\msf{v}_{#1}}
\newcommand{\lpTermSyntax}{
	\begin{array}[t]{r@{\ }l@{\quad}r}
		% ------------------------------------- %
		u & {⦂}\ \TYPE ∣ \KIND
		  & \mcomment{(universes)}                                                             \\
		% ------------------------------------- %
		t & {⦂}\ x ∣ κ ∣ u ∣ \prn{\app{t}{t'}} ∣ \prn{\lam{x : t}{t'}} ∣ \prn{\pii{x : t}{t'}}
		  & \mcomment{(terms)}
		% ------------------------------------- %
	\end{array}
}

\newcommand{\lpCtxSyntax}{
	\begin{array}[t]{r@{\ }l@{\quad}r}
		% ------------------------------------- %
		γ & {⦂}\ \prn{x : t} \mid \prn{t ↪ t'} \
		  & \mcomment{(context elements)}        \\
		% ------------------------------------- %
		Γ & {⦂}\ \any{γ}
		  & \mcomment{(contexts)}
		% ------------------------------------- %
	\end{array}
}

\newcommand{\judge}[3]{{#1} ⊢_{Σ} {#2} : {#3}}
\newcommand{\wf}[1]{\msf{wf}\,{#1}}
% typing rules
\newcommand{\typeRule}{
	\begin{prooftree}
		\hypo{\wf Γ}
		\infer1{ \judge Γ \TYPE \KIND }
	\end{prooftree}
}

\newcommand{\varRule}{
	\begin{prooftree}
		\hypo{ \wf Γ}
		\infer1[$\prn{x : t}∈ Γ$]{ \judge Γ x t }
	\end{prooftree}
}

\newcommand{\constRule}{
	\begin{prooftree}
		\hypo{\wf Γ}
		\hypo{\judge Γ t u}
		\infer2[$\prn{κ : t}∈ Σ$]{ \judge Γ κ t }
	\end{prooftree}
}

\newcommand{\prodRule}{
	\begin{prooftree}
		\hypo{\judge Γ t {\TYPE}}
		\hypo{\judge{Γ, \prn{x : t}}{t'}{u'}}
		\infer2{\judge Γ {\prn{\pii {x : t} {t'}}} {u'} }
	\end{prooftree}
}

\newcommand{\absRule}{
	\begin{prooftree}
		\hypo{\judge{Γ, \prn{x : t}} e {t'}}
		\hypo{\judge Γ {\prn{\pii {x : t} {t'}}} u}
		\infer2{\judge Γ {\prn{\lam {x : t} {e}}} {\prn{\pii {x : t} {t'}}} }
	\end{prooftree}
}

\newcommand{\appRule}{
	\begin{prooftree}
		\hypo{\judge Γ e {\prn{\pii{x : t}{t'}}}}
		\hypo{\judge Γ {e'} t}
		\infer2{\judge Γ {\prn{\app{e}{e'}}} {t'[x ↦ {e'}]}}
	\end{prooftree}
}

\newcommand{\convRule}{
	\begin{prooftree}
		\hypo{\judge Γ e t}
		\hypo{\judge Γ {t'} {u}}
		\infer2[$\prn{t ≡_{Σ} t'}$]{\judge Γ {e} {t'}}
	\end{prooftree}
}

\newcommand{\wfRuleEmp}{
	\begin{prooftree}
		\infer0{ \wf{\msf{∅}} }
	\end{prooftree}
}

\newcommand{\wfRuleVar}{
	\begin{prooftree}
		\hypo{ \judge Γ t u }
		\infer1{ \wf{(Γ, \prn{x : t})}}
	\end{prooftree}
}

\newcommand{\rname}[1]{\mcomment{($\textsc{#1}$)}}

\newcommand{\lpTypingRules}{
	\begin{array}[t]{c}
		\rname{typ} \ {⦂}\ \  \typeRule  \\[4mm]
		\rname{con} \ {⦂}\ \  \constRule \\[4mm]
		\rname{var} \ {⦂}\ \  \varRule   \\[4mm]
		\rname{prod} \ {⦂}\ \  \prodRule \\[4mm]
		\rname{fun} \ {⦂}\ \  \absRule   \\[4mm]
		\rname{app} \ {⦂}\ \  \appRule   \\[4mm]
		\rname{conv} \ {⦂}\ \  \convRule \\[4mm]
	\end{array}
}

\newcommand{\lpWfRules}{
	\begin{array}[t]{c@{\qquad}c}
		\rname{wf0} \ \ {⦂}\ \  \wfRuleEmp
		 & \rname{wf+} \ \ {⦂}\ \  \wfRuleVar
	\end{array}
}


% --- concrete syntax for lambdapi
\newcommand{\lambdapi}{\textsc{LambdaPi}}
\newcommand{\llam}{\binddot{\msf{λ}}}
\newcommand{\ppii}{\binddot{\msf{Π}}}

\newcommand{\prm}[1]{\mbf{#1}}
\newcommand{\xsym}[1]{\mtt{@}{#1}}
\newcommand{\meta}[1]{\ptn{?{#1}}}

\newcommand{\ptrn}[1]{\ptn{\$#1}}
\newcommand{\pv}{\mbf{pvars}}

\newcommand{\imp}[1]{[#1]}
\newcommand{\xpl}[1]{\left(#1\right)}

\newcommand{\lpTermsConcrete}{
	\begin{array}[t]{r@{\ }l@{\quad}r}
		ρ & {⦂}\,\xpl{s : t} ∣ \imp{s : t}
		  & \mcomment{(parameters)}
		\\[1mm]
		t & {⦂}\ s ∣ \imp{t} ∣ \prn{\app{t}{t'}}
		∣ \prn{\llam{ρ}{t}} ∣ \prn{\ppii{ρ}{t}}
		  & \mcomment{(terms)}
	\end{array}
}


\newcommand{\lpPatternSyntax}{
	\begin{array}[t]{r@{\ }l@{\quad}r}
		% ------------------------------------- %
		θ & {⦂}\ \ptrn{x} ∣ s\,\any{θ}
		  & \mcomment{(patterns)}
		\\[1mm]
		r & {⦂}\ \prn{{s\,\any{θ}} ↪ {θ'}}
		  & \mcomment{(rewrite rules)}
		% ------------------------------------- %
	\end{array}
}

\newcommand{\lpConcreteSyntax}{
	\begin{array}[t]{r@{\ }c@{\ }l@{\quad}r}
		m & {⦂}
		  & \mtt{constant} ∣ \mtt{sequential} ∣ \mtt{injective}
		  & \mcomment{(modifiers)}
		\\[1mm]
		c & {⦂}
		  & \maybe{m}\, \mtt{symbol}\ s \ \any{ρ} \,{: t} \ \maybe{≔ t'};
		  & \mcomment{(commands)}
		\\
		  & {∣}
		  & \mtt{rule}\ r\ \any{\mtt{with}\ r'};
		\\
		  & {∣}
		  & \mtt{require}\,\mtt{open}\,\many{μ};
	\end{array}}

% \newcommand{\symb}[1]{\mtt{symbol}\ {#1};}

% \newcommand{\symb}[1]{\mtt{rule}\ {#1};}

\newcommand{\Set}{\msf{Set}}
\newcommand{\El}[1]{\msf{El}\,{#1}}
\newcommand{\Ell}[1]{\msf{El}\,\prn{#1}}


\begin{document}

\begin{boxfigure}[t!]{fig:lpTermSyntax}
	{Abstract syntax for $\lp$-terms.}
	%
	$$\lpTermSyntax$$
	%
\end{boxfigure}
%
\Autoref{fig:lpTermSyntax}
% , \autoref{fig:lpCtxSyntax}
and \autoref{fig:lpTypingRules} provide abstract syntax
for the $\lp$-calculus.
%
In particular, the rules in \autoref{fig:lpTermSyntax}
define the \emph{terms} of the $\lp$-calculus.
%
Each term is either a \emph{variable} $x$,
a \emph{constant} $κ$,
a \emph{universe} from $\{ \TYPE, \KIND \}$,
an \emph{application} of two terms $\prn{\app{t}{t'}}$,
or an \emph{abstraction}
$\prn{\binddot{\mscr{B}}{x : t}{t'}}$
where $\mscr{B}$ is a \emph{binder} from $\{ λ, Π \}$.
% TODO. define/declare α-conversion and β-reduction.
Terms are identified up to $α$-conversion
(i.e., renaming of bound variables), and we assume
the usual definitions for \emph{substitution} $\prn{t[x ↦ t']}$
and \emph{$β$-reduction} $\prn{t ⇝_{β} t'}$.

% --------------------------------------------------------%
% \begin{boxfigure}[t!]{fig:lpCtxSyntax}
% 	{Abstract syntax for $\lp$-contexts.}
% 	%
% 	$$\lpCtxSyntax$$
% 	%
% \end{boxfigure}
A \emph{typing} is an expression of the form $\prn{t : t'}$
for some terms $t$, $t'$, and a \emph{context} is a set of
typings of the form $\prn{x : t}$ for some variable $x$ and
term $t$.
%
Hereinafter, given a context $Γ$ and a typing $\prn{x : t}$,
let $\prn{Γ, \prn{x : t}}$ denote the extension $Γ ∪ \{x : t\}$.
%
A \emph{rewrite rule} is an expression of the form
$\prn{ℓ ↪ r}$, where $ℓ$ and $r$ are terms such that
$ℓ$ has the form
$\prn{\app{\prn{\app{κ}{t_1}} ⋅ \ldots}{t_n}}$
for some terms $\plur t n$.
% -------------------------------------------------------- %
\begin{boxfigure}[t!]{fig:lpTypingRules}
	{Typing rules for the $\lp$-calculus.}
	%
	$$ \lpTypingRules $$
	\hrule
	$$ \lpWfRules $$
\end{boxfigure}
% ----
A \emph{signature} is a set whose elements are either
typings or rewrite rules, where each typing
is of the form $\prn{κ : t}$ for some constant $κ$ and
term $t$.
%
Given some signature $Σ$, let $R_{Σ}$ be the smallest
binary relation such that:
%
\begin{enumerate}
	\item $\prn{ℓ ↪ r} ∈ Σ$ implies $(ℓ, r) ∈ R_{Σ}$, and
	\item $R_{Σ}$ is closed under application, abstraction,
	      and substitution.
\end{enumerate}
%
%, abstraction and substitution.  $R$
Then, \emph{equality modulo rewriting} $\prn{{≡_{Σ}}}$ is
defined as the least equivalence relation containing
$R_Σ$ and the $β$-reduction relation.
Furthermore, the rules in \autoref{fig:lpTypingRules}
provide a (mutually inductive) definition of a
\emph{well-formedness} relation $\prn{\msf{wf}}$ on
contexts and a \emph{typing relation}
$\prn{⊢_{Σ}}$ between contexts and typings,
where $\prn{Γ ⊢_{Σ} e : t}$ may be read as
``$Γ$ proves $e$ has type $t$ with respect to $Σ$''.

% The rules in \autoref{fig:lpCtxSyntax} define
% a \emph{context element} $γ$ as either a \emph{typing}
% $\prn{x : t}$, or a \emph{rewrite rule} .
% %
% Furthermore, a \emph{context} is defined as a list $Γ$ of
% context elements.



% -------------------------------------------------------- %
\subsection{The \textsc{LambdaPi} Proof Assistant}
%
\begin{boxfigure}[t!]{fig:lpSyntax}
	{Concrete syntax for \textsc{LambdaPi} signatures.}
	%
	$$ \lpConcreteSyntax $$
\end{boxfigure}
%
Concrete syntax for a fragment of the $\lambdapi$ proof
assistant is given in \autoref{fig:lpSyntax}.
%
A $\lambdapi$ \emph{file} is a list of \emph{commands},
where each command is either a \emph{symbol declaration}
(\mtt{symbol}), \emph{rewrite rule declaration} (\mtt{rule}),
or an \emph{import} (\mtt{require}).
%
Each $\lambdapi$ file can be seen a representation of a
signature in the $\lp$-calculus thus:
%
\begin{enumerate}
	\item The most primitive form of symbol declaration is
	      $\prn{\mtt{symbol}\ s : t;}$, which has the effect
	      of adding $\prn{s : t}$ to the signature.

	\item A rewrite rule $\prn{l ↪ r}$ may be introduced by
	      $\prn{\mtt{rule}\ ℓ ↪ r;}$.
	      For user convenience, a list of rewrite rules
	      may be introduced using the \mtt{with}-syntax.

	\item Given an \emph{environment} $Θ$,
	      the contents of files named by $\plur μ n$
	      may be imported using the $\mtt{require}$ command.
	      %
	      By default, imported symbols are qualified by
	      the name of the file they were defined by.
	      That is, if $μ$ declares the symbol $s$
	      and $μ'$ imports $μ$, then the name of $s$ in $μ'$
	      is $\prn{μ.s}$.


\end{enumerate}


\end{document}


\section{Translation and Results}\label{sec:results}
\documentclass[class=llncs, crop=false]{standalone}

\documentclass{llncs}

\usepackage{unicode-math}
\usepackage{newcomputermodern}
\usepackage{microtype}

\usepackage{xcolor-material}
\usepackage{hyperref}
\usepackage{ebproof}
\usepackage{array}

\usepackage{jbw-boxfigure}
\usepackage{jbw-fix-hyperref-autoref}

\title{%
	Automatically Translating Proof Systems
	for SMT Solvers to the λΠ-calculus}
\author{Ciarán Dunne, Guillaume Burel}
\institute{INRIA, ENS Paris-Saclay}

\usepackage[backend=biber]{biblatex}
\addbibresource{inria.bib}

% font faces
\newcommand{\msf}{\mathsf}
\newcommand{\mbf}{\mathbf}
\newcommand{\mscr}{\mathscr}
\newcommand{\mtt}[1]{{\color{MaterialPink500}{\texttt{#1}}}}
\newcommand{\ptn}[1]{{\color{MaterialPurple500}{\texttt{#1}}}}
\newcommand{\mcomment}[1]{{\color{MaterialIndigo500}{\text{#1}}}}
\newcommand{\wrapp}[4]{{\color{#1}{#2}}{#4}{\color{#1}{#3}}}
\newcommand{\prn}[1]{\wrapp{MaterialGrey600}{\lparen}{\rparen}{#1}}

\newcommand{\cvc}[1]{\textsc{cvc{#1}}}
\newcommand{\verit}{\textsc{veriT}}

% ----- abstract syntax ----------
% plurality, binding
\newcommand{\plur}[2]{{#1}_{1}\ldots{#1}_{#2}}
\newcommand{\plurcomma}[2]{{#1}_{1}, \ldots, {#1}_{#2}}
\newcommand{\maybe}[1]{\wrapp{MaterialGrey600}{\langle}{\rangle_?}{#1}}
\newcommand{\many}[1]{\wrapp{MaterialGrey600}{\langle}{\rangle_+}{#1}}
\newcommand{\any}[1]{\wrapp{MaterialGrey600}{\langle}{\rangle_∗}{#1}}
\newcommand{\some}[1]{\wrapp{MaterialGrey600}{\langle}{\rangle}{#1}}
\newcommand{\binddot}[3]{{#1\,#2.\,#3}}

% binders
\newcommand{\lam}{\binddot{λ}}
\newcommand{\pii}{\binddot{Π}}
\newcommand{\alam}[1]{\binddot{\msf{λ}^{#1}}}
\newcommand{\apii}[1]{\binddot{\msf{Π}^{#1}}}
\newcommand{\lett}[2]{\msf{let}\ \prn{#1}\,\msf{in}\ #2}

% applications
\newcommand{\appSym}{\cdot}
\newcommand{\app}[2]{{#1}\appSym{#2}}
\newcommand{\apptwo}[3]{#1\,#2\,#3}
\newcommand{\appthree}[4]{#1\,#2\,#3\,#4}
\newcommand{\appfour}[5]{#1\,#2\,#3\,#4\,#5}
\newcommand{\appldots}[3]{((\app{#1}{#2})\,\ldots\,{#3})}

% universes
\newcommand{\PROP}{\mtt{prop}}
\newcommand{\TYPE}{\mtt{type}}
\newcommand{\KIND}{\mtt{kind}}

% sets for syntactic categories
\newcommand{\Term}{\msf{Term}}



% -------------- Eunoia ------------------------------------
% eunoia symbols
\newcommand{\epar}[1]{\wrapp{MaterialGrey600}{\texttt{(}}{\texttt{)}}{#1}}
\newcommand{\eapp}[2]{\epar{#1\ #2}}
\newcommand{\earr}[1]{\epar{\mtt{->}\ #1}}

\newcommand{\eo}{\msf{eo}}
\newcommand{\sym}[1]{\msf{s}_{\msf{#1}}}
% \newcommand{\eoapp}[2]{\prn{#1\ #2}}
% \newcommand{\earr}[1]{\mtt{#1}}
\newcommand{\ctyp}{\mbf{ctyp}}
\newcommand{\att}{\mbf{att}}

\newcommand{\hyp}{\mbf{hyp}}
\newcommand{\step}{\mbf{step}}

% --- figures for `eoas.tex` ------------------------------
% s & {⦂}\ \sym 0 ∣ \sym 1 ∣ \ldots
% 		& \mcomment{(symbols)}         &
%
\newcommand{\eoTermSyntax}{
	\begin{array}[t]{r@{\ }l@{\quad}l@{\qquad}r@{\ }l@{\quad}l}
		%!----------------------------------------!%
		t & {⦂}\ s \mid \eapp{s}{\vec t} & \mcomment{(terms)}      &
		ρ & {⦂}\ \epar{s\ t\ \maybe{ν}}  & \mcomment{(parameters)}   \\[1mm]
	\end{array}
}

\newcommand{\eoAttrSyntax}{
	\begin{array}[t]{r@{\ }c@{\ }l@{\quad}l}
		ν & {⦂} & \mtt{:implicit} \mid \mtt{:list}                       & \mcomment{(var. attributes)}   \\[1mm]
		α & {⦂} & \mtt{:right-assoc} ∣ \mtt{:right-assoc-nil}\,\some{t}  & \mcomment{(const. attributes)} \\
		  & {∣} & \mtt{:left-assoc} ∣ \mtt{:left-assoc-nil}\,\some{t}                                     \\
		  & {∣} & \mtt{:chainable}\,\some{s} ∣ \mtt{:pairwise}\,\some{s}
	\end{array}
}

\newcommand{\eoCommandSyntax}{
	\begin{array}[t]{r@{\ }l@{\quad}l}
		r & \begin{array}[t]{c@{\ }l}
			    {⦂} & \epar{t\ t'}
		    \end{array}
		  & \mcomment{(term pairs)}
		\\
		δ & \begin{array}[t]{c@{\ }l}
			    {⦂} & \epar{\mtt{declare-const}\ s\ t\ \maybe{α}}                               \\
			    {∣} & \epar{\mtt{declare-parameterized-const}\ s\ \epar{\vec{ρ}}\ t\ \maybe{α}} \\
			    {∣} & \epar{\mtt{define}\ {s} \ \epar{\vec ρ} \ t\ \ \maybe{\mtt{:type}\,t'}}   \\
			    {∣} & \epar{\mtt{program}\ {s} \ \epar{\vec ρ}\
			    \mtt{:signature}\,\epar{\vec{t}}\,t'\
			    \epar{\vec{r}}
			    }                                                                               \\
			    {∣} & {\color{MaterialGrey600}{\texttt{(}}}
			    \mtt{declare-rule}\ s\ \epar{\vec ρ}\
			    \\ & \quad \maybe{\mtt{:premises}\ \epar{\vec{t}_{\text{prem}}}}\
			    \\ & \quad \maybe{\mtt{:args}\ \epar{\vec{t}_{\text{args}}}}\
			    \\ & \quad \maybe{\mtt{:requires}\ \epar{\vec{r}}}\
			    \\ & \quad \mtt{:conclusion}\ {t_{\text{conc}}}
			    {\color{MaterialGrey600}{\texttt{)}}}                                           \\
			    {∣} & \epar{\mtt{include}\ μ}                                                   \\
		    \end{array}
		  & \mcomment{(std. commands)}
	\end{array}
}

\newcommand{\eoPrfCommandSyntax}{
	\begin{array}[t]{r@{\ }l@{\quad}l}
		π & \begin{array}[t]{c@{\ }l}{⦂}
			        & \epar{\mtt{assume}\ s\ φ}       \\
			    % {∣} & \prn{\mtt{assume-push}\ s\ e}      \\
			    {∣} & \epar{\mtt{step}\ s\ φ\
			    \some{\mtt{:rule}\ s'}\
			    \maybe{\mtt{:premises}\ {\vec ψ}}\
			    \maybe{\mtt{:args}\ {\vec t}}}\\
			    % {∣} & \prn{\mtt{step-pop}\ s\ \maybe{e}\
			    % \prn{\mtt{:rule}\ {s_r}}\
			    % \maybe{\mtt{:premises}\ {\vec φ}}\
			    % \maybe{\mtt{:args}\ {\vec e}} }
		    \end{array}
		  & \mcomment{(proof commands)}
	\end{array}}

% --- macros for `lp.tex` ------------------------------
\newcommand{\lp}{λΠ}
\newcommand{\var}[1]{\msf{v}_{#1}}
\newcommand{\lpTermSyntax}{
	\begin{array}[t]{r@{\ }l@{\quad}r}
		% ------------------------------------- %
		u & {⦂}\ \TYPE ∣ \KIND
		  & \mcomment{(universes)}                                                             \\
		% ------------------------------------- %
		t & {⦂}\ x ∣ κ ∣ u ∣ \prn{\app{t}{t'}} ∣ \prn{\lam{x : t}{t'}} ∣ \prn{\pii{x : t}{t'}}
		  & \mcomment{(terms)}
		% ------------------------------------- %
	\end{array}
}

\newcommand{\lpCtxSyntax}{
	\begin{array}[t]{r@{\ }l@{\quad}r}
		% ------------------------------------- %
		γ & {⦂}\ \prn{x : t} \mid \prn{t ↪ t'} \
		  & \mcomment{(context elements)}        \\
		% ------------------------------------- %
		Γ & {⦂}\ \any{γ}
		  & \mcomment{(contexts)}
		% ------------------------------------- %
	\end{array}
}

\newcommand{\judge}[3]{{#1} ⊢_{Σ} {#2} : {#3}}
\newcommand{\wf}[1]{\msf{wf}\,{#1}}
% typing rules
\newcommand{\typeRule}{
	\begin{prooftree}
		\hypo{\wf Γ}
		\infer1{ \judge Γ \TYPE \KIND }
	\end{prooftree}
}

\newcommand{\varRule}{
	\begin{prooftree}
		\hypo{ \wf Γ}
		\infer1[$\prn{x : t}∈ Γ$]{ \judge Γ x t }
	\end{prooftree}
}

\newcommand{\constRule}{
	\begin{prooftree}
		\hypo{\wf Γ}
		\hypo{\judge Γ t u}
		\infer2[$\prn{κ : t}∈ Σ$]{ \judge Γ κ t }
	\end{prooftree}
}

\newcommand{\prodRule}{
	\begin{prooftree}
		\hypo{\judge Γ t {\TYPE}}
		\hypo{\judge{Γ, \prn{x : t}}{t'}{u'}}
		\infer2{\judge Γ {\prn{\pii {x : t} {t'}}} {u'} }
	\end{prooftree}
}

\newcommand{\absRule}{
	\begin{prooftree}
		\hypo{\judge{Γ, \prn{x : t}} e {t'}}
		\hypo{\judge Γ {\prn{\pii {x : t} {t'}}} u}
		\infer2{\judge Γ {\prn{\lam {x : t} {e}}} {\prn{\pii {x : t} {t'}}} }
	\end{prooftree}
}

\newcommand{\appRule}{
	\begin{prooftree}
		\hypo{\judge Γ e {\prn{\pii{x : t}{t'}}}}
		\hypo{\judge Γ {e'} t}
		\infer2{\judge Γ {\prn{\app{e}{e'}}} {t'[x ↦ {e'}]}}
	\end{prooftree}
}

\newcommand{\convRule}{
	\begin{prooftree}
		\hypo{\judge Γ e t}
		\hypo{\judge Γ {t'} {u}}
		\infer2[$\prn{t ≡_{Σ} t'}$]{\judge Γ {e} {t'}}
	\end{prooftree}
}

\newcommand{\wfRuleEmp}{
	\begin{prooftree}
		\infer0{ \wf{\msf{∅}} }
	\end{prooftree}
}

\newcommand{\wfRuleVar}{
	\begin{prooftree}
		\hypo{ \judge Γ t u }
		\infer1{ \wf{(Γ, \prn{x : t})}}
	\end{prooftree}
}

\newcommand{\rname}[1]{\mcomment{($\textsc{#1}$)}}

\newcommand{\lpTypingRules}{
	\begin{array}[t]{c}
		\rname{typ} \ {⦂}\ \  \typeRule  \\[4mm]
		\rname{con} \ {⦂}\ \  \constRule \\[4mm]
		\rname{var} \ {⦂}\ \  \varRule   \\[4mm]
		\rname{prod} \ {⦂}\ \  \prodRule \\[4mm]
		\rname{fun} \ {⦂}\ \  \absRule   \\[4mm]
		\rname{app} \ {⦂}\ \  \appRule   \\[4mm]
		\rname{conv} \ {⦂}\ \  \convRule \\[4mm]
	\end{array}
}

\newcommand{\lpWfRules}{
	\begin{array}[t]{c@{\qquad}c}
		\rname{wf0} \ \ {⦂}\ \  \wfRuleEmp
		 & \rname{wf+} \ \ {⦂}\ \  \wfRuleVar
	\end{array}
}


% --- concrete syntax for lambdapi
\newcommand{\lambdapi}{\textsc{LambdaPi}}
\newcommand{\llam}{\binddot{\msf{λ}}}
\newcommand{\ppii}{\binddot{\msf{Π}}}

\newcommand{\prm}[1]{\mbf{#1}}
\newcommand{\xsym}[1]{\mtt{@}{#1}}
\newcommand{\meta}[1]{\ptn{?{#1}}}

\newcommand{\ptrn}[1]{\ptn{\$#1}}
\newcommand{\pv}{\mbf{pvars}}

\newcommand{\imp}[1]{[#1]}
\newcommand{\xpl}[1]{\left(#1\right)}

\newcommand{\lpTermsConcrete}{
	\begin{array}[t]{r@{\ }l@{\quad}r}
		ρ & {⦂}\,\xpl{s : t} ∣ \imp{s : t}
		  & \mcomment{(parameters)}
		\\[1mm]
		t & {⦂}\ s ∣ \imp{t} ∣ \prn{\app{t}{t'}}
		∣ \prn{\llam{ρ}{t}} ∣ \prn{\ppii{ρ}{t}}
		  & \mcomment{(terms)}
	\end{array}
}


\newcommand{\lpPatternSyntax}{
	\begin{array}[t]{r@{\ }l@{\quad}r}
		% ------------------------------------- %
		θ & {⦂}\ \ptrn{x} ∣ s\,\any{θ}
		  & \mcomment{(patterns)}
		\\[1mm]
		r & {⦂}\ \prn{{s\,\any{θ}} ↪ {θ'}}
		  & \mcomment{(rewrite rules)}
		% ------------------------------------- %
	\end{array}
}

\newcommand{\lpConcreteSyntax}{
	\begin{array}[t]{r@{\ }c@{\ }l@{\quad}r}
		m & {⦂}
		  & \mtt{constant} ∣ \mtt{sequential} ∣ \mtt{injective}
		  & \mcomment{(modifiers)}
		\\[1mm]
		c & {⦂}
		  & \maybe{m}\, \mtt{symbol}\ s \ \any{ρ} \,{: t} \ \maybe{≔ t'};
		  & \mcomment{(commands)}
		\\
		  & {∣}
		  & \mtt{rule}\ r\ \any{\mtt{with}\ r'};
		\\
		  & {∣}
		  & \mtt{require}\,\mtt{open}\,\many{μ};
	\end{array}}

% \newcommand{\symb}[1]{\mtt{symbol}\ {#1};}

% \newcommand{\symb}[1]{\mtt{rule}\ {#1};}

\newcommand{\Set}{\msf{Set}}
\newcommand{\El}[1]{\msf{El}\,{#1}}
\newcommand{\Ell}[1]{\msf{El}\,\prn{#1}}


\begin{document}
% ------------------------
Recall the definition of $\eunoia$ terms given by
\autoref{sec:eunoia}, and similarly the definition
for $\lambdapi$ terms from \autoref{sec:lambdapi}.
%
We define a pair of \emph{translation} operators below.
%
\begin{definition}
For any $\eunoia$ symbol $s ∈ 𝒮_\eo$, define $\bar s$ thus:
$$
  \bar s =
\begin{cases}
  \esc s
  & \text{if $s$ contains any of
    $\{ \mtt{\$}, \mtt{@}, \ldots \}$,}
  \\
  {\ s}
  & \text{otherwise.}
\end{cases}
$$
\end{definition}
%
\begin{definition}
%
Let $\tm{⋅}$ be the least operator such that
$\tm{s} = \bar s$ and:
%
$$
\begin{array}[t]{r@{\ =\ }l}
  \tm{\eapp{s}{\vec t}}
  &
  \begin{cases}
    \tm{t_1} ⤳ \ldots ⤳ \ty{t_n}
    &
    \text{if $s = \prn{\mtt{->}}$,}
  \\
    \appldots{\bar s}{\tm{t_1}}{\tm{t_n}}
    &
    \text{otherwise.}
\end{cases}
\end{array}
$$
%
Furthermore, let $\ty{⋅}$ be the least operator
such that:
$$
\begin{array}[t]{r@{\ =\ }l}
  \ty{s} &
  \begin{cases}
    \Set & \text{if $s = \mtt{Type}$,}
  \\
    \El{\tm{s}} & \text{otherwise.}
  \end{cases}
\\[5mm]
  \ty{\eapp{s}{\vec t}}
  &
  \begin{cases}
    \ty{t_1} → \ldots → \ty{t_n}
    &
    \text{if $s = \prn{\mtt{->}}$,}
  \\
    \El{\tm{\eapp{s}{\vec t}}}
    &
    \text{otherwise.}
\end{cases}
\end{array}
$$
%
\end{definition}
%
Now, recall the set of $\eunoia$ commands,
and also those of $\lambdapi$.
%
\begin{definition}
Let $\cmd{⋅}$ be the least operator such that:
%
$$
\begin{array}[t]{l@{\ =\ }l}
\cmd{\epar{\mtt{declare-const}\ s\ t\ \maybe{α}}}
& \mtt{symbol}\ {\bar s} : {\ty{t}};
\\
%-----------
\cmd{\epar{\mtt{declare-parameterized-const}\ s\ \epar{\vec{ρ}}\ t\ \maybe{α}}}
& \mtt{symbol}\ {\bar s}\ {\ctx{\vec ρ}}: {\ty{t}};\
\\
%-----------
\cmd{\epar{\mtt{define}\ {s}\ \epar{\vec ρ} \ t\ \maybe{\mtt{:type}\ t'}}}
& \mtt{symbol}\ {\bar s}\ {\ctx{\vec ρ}} : {\ty{t'}}
  ≔ \tm{t};
\\
%-----------
\cmd{
\epar{\mtt{program}\ {s} \ \epar{\vec ρ}\
			    \mtt{:signature}\,\epar{\vec{t}}\,t'
				\
			    \epar{\vec{r}}
			}
}
& \mtt{symbol}\ {\bar s}\ {\ctx{\vec ρ}} : {\ty{\earr{\vec t\ t'}}};

\end{array}
$$
$$
\begin{array}[t]{c@{\ }l}
			    {⦂} &                                \\
			    {∣} &  \\
			    {∣} &    \\
			    {∣} &                                                                           \\
			    {∣} & {\color{MaterialGrey600}{\texttt{(}}}
			    \mtt{declare-rule}\ s\ \epar{\vec ρ}\
			    \\ & \quad \maybe{\mtt{:premises}\ \epar{\vec{t}_{\text{prem}}}}\
			    \\ & \quad \maybe{\mtt{:args}\ \epar{\vec{t}_{\text{args}}}}\
			    \\ & \quad \maybe{\mtt{:requires}\ \epar{\vec{r}}}\
			    \\ & \quad \mtt{:conclusion}\ {t_{\text{conc}}}
			    {\color{MaterialGrey600}{\texttt{)}}}                                           \\
			    {∣} & \epar{\mtt{include}\ μ}
		    \end{array}
$$

\end{definition}

\end{document}


\section{Conclusion and Future Work}\label{sec:conclu}


\printbibliography

\end{document}


% \hrule
% \vspace{2mm}

% 	\hrule
% $$\begin{array}{r@{\ }l@{\quad}l}
% 		δ &
% 		\begin{array}[t]{c@{\ }l}
% 			{⦂} & \prn{\decl{x}{\prn{\vec ρ}}{\prn{{:}\ t}}{\prn{{∷}\ α}}}
% 			∣ \prn{\defn{x}{\paren{\vec ρ}}{\paren{{≔}\ t}}{\paren{{:}\ t'}}}    \\
% 			{∣} & \paren{\prog{x}
% 			{\paren{\vec ρ}}
% 			{\paren{{:}\ {\paren{t_1 \ldots t_n} → t'}}}
% 			{\paren{r_1 \ldots r_m}}
% 			}                                                                      \\
% 			{∣} & \paren{\irule{x}
% 				{\paren{\vec ρ}}
% 				{\paren{φ_1 \ldots φ_n}}
% 				{\paren{t_1 \ldots t_m}}
% 				{\paren{ψ}}
% 			}
% 		\end{array}
% 		  & \mcomment{(declarations)}
% 	\end{array}
% $$
