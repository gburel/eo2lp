\documentclass[class=llncs, crop=false]{standalone}

\documentclass{llncs}

\usepackage{unicode-math}
\usepackage{newcomputermodern}
\usepackage{microtype}

\usepackage{xcolor-material}
\usepackage{hyperref}
\usepackage{ebproof}
\usepackage{array}

\usepackage{jbw-boxfigure}
\usepackage{jbw-fix-hyperref-autoref}

\title{%
	Automatically Translating Proof Systems
	for SMT Solvers to the λΠ-calculus}
\author{Ciarán Dunne, Guillaume Burel}
\institute{INRIA, ENS Paris-Saclay}

\usepackage[backend=biber]{biblatex}
\addbibresource{inria.bib}

% font faces
\newcommand{\msf}{\mathsf}
\newcommand{\mbf}{\mathbf}
\newcommand{\mscr}{\mathscr}
\newcommand{\mtt}[1]{{\color{MaterialPink500}{\texttt{#1}}}}
\newcommand{\ptn}[1]{{\color{MaterialPurple500}{\texttt{#1}}}}
\newcommand{\mcomment}[1]{{\color{MaterialIndigo500}{\text{#1}}}}
\newcommand{\wrapp}[4]{ {\color{#1}{#2}}{#4}{\color{#1}{#3}}}
\newcommand{\prn}[1]{\wrapp{MaterialGrey600}{\lparen}{\rparen}{#1}}

% ----- abstract syntax ----------
% plurality, binding
\newcommand{\plur}[2]{{#1}_{1}\ldots{#1}_{#2}}
\newcommand{\plurcomma}[2]{{#1}_{1}, \ldots, {#1}_{#2}}
\newcommand{\maybe}[1]{\wrapp{MaterialGrey600}{\langle}{\rangle_?}{#1}}
\newcommand{\many}[1]{\wrapp{MaterialGrey600}{\langle}{\rangle_+}{#1}}
\newcommand{\any}[1]{\wrapp{MaterialGrey600}{\langle}{\rangle_∗}{#1}}
\newcommand{\some}[1]{\wrapp{MaterialGrey600}{\langle}{\rangle}{#1}}
\newcommand{\binddot}[3]{{#1\,#2.\,#3}}

% binders
\newcommand{\lam}{\binddot{λ}}
\newcommand{\pii}{\binddot{Π}}
\newcommand{\alam}[1]{\binddot{\msf{λ}^{#1}}}
\newcommand{\apii}[1]{\binddot{\msf{Π}^{#1}}}
\newcommand{\lett}[2]{\msf{let}\ \prn{#1}\,\msf{in}\ #2}

% applications
\newcommand{\appSym}{\cdot}
\newcommand{\app}[2]{{#1}\appSym{#2}}
\newcommand{\apptwo}[3]{#1\,#2\,#3}
\newcommand{\appthree}[4]{#1\,#2\,#3\,#4}
\newcommand{\appfour}[5]{#1\,#2\,#3\,#4\,#5}
\newcommand{\appldots}[3]{((\app{#1}{#2})\,\ldots\,{#3})}

% universes
\newcommand{\PROP}{\mtt{prop}}
\newcommand{\TYPE}{\mtt{type}}
\newcommand{\KIND}{\mtt{kind}}

% sets for syntactic categories
\newcommand{\Term}{\msf{Term}}

% eunoia symbols
\newcommand{\eo}{\msf{eo}}
\newcommand{\sym}[1]{\msf{s}_{\msf{#1}}}
\newcommand{\eoapp}[2]{\prn{#1\ #2}}

% --- figures for `eoas.tex` ------------------------------
\newcommand{\eoTermSyntax}{
	\begin{array}[t]{r@{\ }l@{\quad}l@{\qquad}r@{\ }l@{\quad}l}
		%!----------------------------------------!%
		s & {⦂}\ \sym 0 ∣ \sym 1 ∣ \ldots                                                                                                                                                                                                                                                     & \mcomment{(symbols)}           & e & {⦂}\ s \mid \prn{{s}\,\many{e}}  & \mcomment{(terms)}      \\[1mm]
		ν & {⦂}\ \mtt{:implicit} \mid \mtt{:list}                                                                                                                                                                                                                                             & \mcomment{(var. attributes)}   & ρ & {⦂}\ \prn{s\ e\ \maybe{ν}}\qquad & \mcomment{(parameters)} \\[1mm]
		α & \multicolumn{4}{@{}l}{ \begin{array}[t]{@{}l@{\ }l@{\quad}r}{⦂} & \mtt{:right-assoc} ∣ \mtt{:right-assoc-nil}\some{t} \\ {∣} & \mtt{:left-assoc} ∣ \mtt{:left-assoc-nil}\some{t} \\ {∣} & \mtt{:chainable}\some{s} ∣ \mtt{:pairwise}\some{s} ∣ \mtt{:binder}\some{s}\end{array} } & \mcomment{(const. attributes)}
	\end{array}
}

\newcommand{\eoCommandSyntax}{
	\begin{array}[t]{r@{\ }l}
		δ & \begin{array}[t]{c@{\ }l}{⦂} & \prn{\mtt{declare-const}\ s\ e\ α} \\ {∣} & \prn{\mtt{declare-consts}\ ℓ\ e} \\ {∣} & \prn{\mtt{declare-parameterized-const}\ s\ {\vec ρ}\ e\ α} \\ {∣} & {\color{MaterialGrey600}{\big\lparen}} \mtt{declare-rule}\ s\ \any{ρ}\
             \\ & \quad \maybe{\mtt{:assumption}\,e}\
             \\ & \quad \maybe{\mtt{:premises}\,\any{φ}}\
             \\ & \quad \maybe{\mtt{:args}\,\any{e}}\
             \\ & \quad \maybe{\mtt{:requires}\,{\many{e,e'}}}\
             \\ & \quad \mtt{:conclusion}\,{ψ} {\color{MaterialGrey600}{\big\rparen}} \\ {∣} & \prn{\mtt{declare-type}\ {s}\ \prn{\any e}} \\ {∣} & \prn{\mtt{define}\ {s} \ \maybe{\vec ρ} \ e\
             \ \maybe{\mtt{:type}\,e'}}                           \\ {∣} & \prn{\mtt{define-type}\ {s} \ \prn{\any e} \ e'} \\ {∣} & \prn{\mtt{include}\ \some{\msf{string}}} \\ {∣} & \prn{\mtt{program}\ x \ \any{ρ} \ \some{\mtt{:signature}\,\prn{\many{e}}\,e'} \ \prn{\many{e\ e'}} }\end{array}
	\end{array}}

\newcommand{\eoPrfCommandSyntax}{
	\begin{array}[t]{r@{\ }l}
		π & \begin{array}[t]{c@{\ }l}{⦂} & \prn{\mtt{assume}\ s\ e} \\ {∣} & \prn{\mtt{assume-push}\ s\ e} \\ {∣} & \prn{\mtt{step}\ s\ \maybe{e}\
             \prn{\mtt{:rule}\ s_r}\
             \maybe{\mtt{:premises}\ {\vec φ}}\
             \maybe{\mtt{:args}\ {\vec e}} }            \\ {∣} & \prn{\mtt{step-pop}\ s\ \maybe{e}\
             \prn{\mtt{:rule}\ {s_r}}\
             \maybe{\mtt{:premises}\ {\vec φ}}\
             \maybe{\mtt{:args}\ {\vec e}} }\end{array}
	\end{array}}

% --- macros for `lp.tex` ------------------------------
\newcommand{\lp}{λΠ}
\newcommand{\var}[1]{\msf{v}_{#1}}
\newcommand{\lpTermSyntax}{
	\begin{array}[t]{r@{\ }l@{\quad}r}
		% ------------------------------------- %
		u & {⦂}\ \TYPE ∣ \KIND
		  & \mcomment{(universes)}                                                             \\
		% ------------------------------------- %
		t & {⦂}\ x ∣ κ ∣ u ∣ \prn{\app{t}{t'}} ∣ \prn{\lam{x : t}{t'}} ∣ \prn{\pii{x : t}{t'}}
		  & \mcomment{(terms)}
		% ------------------------------------- %
	\end{array}
}

\newcommand{\lpCtxSyntax}{
	\begin{array}[t]{r@{\ }l@{\quad}r}
		% ------------------------------------- %
		γ & {⦂}\ \prn{x : t} \mid \prn{t ↪ t'} \
		  & \mcomment{(context elements)}        \\
		% ------------------------------------- %
		Γ & {⦂}\ \any{γ}
		  & \mcomment{(contexts)}
		% ------------------------------------- %
	\end{array}
}

\newcommand{\judge}[3]{{#1} ⊢_{Σ} {#2} : {#3}}
\newcommand{\wf}[1]{\msf{wf}\,{#1}}
% typing rules
\newcommand{\typeRule}{
	\begin{prooftree}
		\hypo{\wf Γ}
		\infer1{ \judge Γ \TYPE \KIND }
	\end{prooftree}
}

\newcommand{\varRule}{
	\begin{prooftree}
		\hypo{ \wf Γ}
		\infer1[$\prn{x : t}∈ Γ$]{ \judge Γ x t }
	\end{prooftree}
}

\newcommand{\constRule}{
	\begin{prooftree}
		\hypo{\wf Γ}
		\hypo{\judge Γ t u}
		\infer2[$\prn{κ : t}∈ Σ$]{ \judge Γ κ t }
	\end{prooftree}
}

\newcommand{\prodRule}{
	\begin{prooftree}
		\hypo{\judge Γ t u}
		\hypo{\judge{Γ, \prn{x : t}}{t'}{u'}}
		\infer2{\judge Γ {\prn{\pii {x : t} {t'}}} {u'} }
	\end{prooftree}
}

\newcommand{\absRule}{
	\begin{prooftree}
		\hypo{\judge{Γ, \prn{x : t}} e {t'}}
		\hypo{\judge Γ {\prn{\pii {x : t} {t'}}} u}
		\infer2{\judge Γ {\prn{\lam {x : t} {e}}} {\prn{\pii {x : t} {t'}}} }
	\end{prooftree}
}

\newcommand{\appRule}{
	\begin{prooftree}
		\hypo{\judge Γ e {\prn{\pii{x : t}{t'}}}}
		\hypo{\judge Γ {e'} t}
		\infer2{\judge Γ {\prn{\app{e}{e'}}} {t'[x ↦ {e'}]}}
	\end{prooftree}
}

\newcommand{\convRule}{
	\begin{prooftree}
		\hypo{\judge Γ e t}
		\hypo{\judge Γ {t'} {u}}
		\infer2[$\prn{t ≡_{Σ} t'}$]{\judge Γ {e} {t'}}
	\end{prooftree}
}

\newcommand{\wfRuleEmp}{
	\begin{prooftree}
		\infer0{ \wf{\msf{∅}} }
	\end{prooftree}
}

\newcommand{\wfRuleVar}{
	\begin{prooftree}
		\hypo{ \judge Γ t u }
		\infer1{ \wf{(Γ, \prn{x : t})}}
	\end{prooftree}
}

\newcommand{\rname}[1]{\mcomment{($\textsc{#1}$)}}

\newcommand{\lpTypingRules}{
	\begin{array}[t]{c}
		\rname{typ} \ {⦂}\ \  \typeRule  \\[4mm]
		\rname{con} \ {⦂}\ \  \constRule \\[4mm]
		\rname{var} \ {⦂}\ \  \varRule   \\[4mm]
		\rname{prod} \ {⦂}\ \  \prodRule \\[4mm]
		\rname{fun} \ {⦂}\ \  \absRule   \\[4mm]
		\rname{app} \ {⦂}\ \  \appRule   \\[4mm]
		\rname{conv} \ {⦂}\ \  \convRule \\[4mm]
	\end{array}
}

\newcommand{\lpWfRules}{
	\begin{array}[t]{c@{\qquad}c}
		\rname{wf0} \ \ {⦂}\ \  \wfRuleEmp
		 & \rname{wf+} \ \ {⦂}\ \  \wfRuleVar
	\end{array}
}


% --- concrete syntax for lambdapi
\newcommand{\lambdapi}{\textsc{LambdaPi}}
\newcommand{\llam}{\binddot{\msf{λ}}}
\newcommand{\ppii}{\binddot{\msf{Π}}}

\newcommand{\xsym}[1]{\mtt{@}{#1}}
\newcommand{\ptrn}[1]{\ptn{\$#1}}
\newcommand{\meta}[1]{\mtt{?}{#1}}

\newcommand{\imp}[1]{[#1]}
\newcommand{\xpl}[1]{\left(#1\right)}

\newcommand{\lpTermsConcrete}{
	\begin{array}[t]{r@{\ }l@{\quad}r}
		χ & {⦂}\,\xpl{s : t} ∣ \imp{s : t}
		  & \mcomment{(parameters)}
		\\[1mm]
		t & {⦂}\ s ∣ \imp{t} ∣ \prn{\app{t}{t'}}
		∣ \prn{\llam{χ}{t}} ∣ \prn{\ppii{χ}{t}}
		  & \mcomment{(terms)}
	\end{array}
}


\newcommand{\lpPatternSyntax}{
	\begin{array}[t]{r@{\ }l@{\quad}r}
		% ------------------------------------- %
		p & {⦂}\ \ptrn{x} ∣ s\,\any{p}
		  & \mcomment{(patterns)}
		\\[1mm]
		r & {⦂}\ \prn{{s\,\any{p}} ↪ {p'}}
		  & \mcomment{(rewrite rules)}
		% ------------------------------------- %
	\end{array}
}

\newcommand{\lpConcreteSyntax}{
	\begin{array}[t]{r@{\ }c@{\ }l@{\quad}r}
		m & {⦂}
		  & \mtt{constant} ∣ \mtt{sequential} ∣ \mtt{injective}
		  & \mcomment{(modifiers)}
		\\[1mm]
		c & {⦂}
		  & \maybe{m}\, \mtt{symbol}\ s \ \any{ρ} \,{: t} \ \maybe{≔ t'};
		  & \mcomment{(commands)}
		\\
		  & {∣}
		  & \mtt{rule}\ r\ \any{\mtt{with}\ r'};
		\\
		  & {∣}
		  & \mtt{require}\,\mtt{open}\,\many{μ};
	\end{array}}

% \newcommand{\symb}[1]{\mtt{symbol}\ {#1};}

% \newcommand{\symb}[1]{\mtt{rule}\ {#1};}


\begin{document}
% ---------------------------------------------------------
\begin{boxfigure}[t!]{fig:eo-syntax}
	{Syntax for Eunoia: terms, attributes, and commands.}
	%
	$$ \eoTermSyntax $$
	\hrule
	$$ \eoAttrSyntax $$
	\hrule
	$$ \eoCommandSyntax $$
	\hrule
	$$ \eoPrfCommandSyntax $$
\end{boxfigure}
% ---------------------------------------------------------
With respect to a fixed set $𝒮_{\mbf{eo}}$ of \emph{symbols},
the rules in \autoref{fig:eo-syntax} define syntax for a
fragment of Eunoia.
%
In particular, we define sets of expressions
for \emph{terms}, \emph{parameters}, and \emph{attributes}.
%
Each term is either a symbol $s$ or an \emph{application}
$\eapp{s}{\vec t}$ for some list of terms $\vec t$.
%
% TODO. ref to explanation of variable attributes.
Let $ν$ and $α$ range over \emph{variable attributes}
and \emph{constant attributes} respectively.
%
Then, each \emph{parameter} $ρ$ consists of a symbol $s$,
a term $t$, and possibly a variable attribute $ν$.
%

The rules of \autoref{fig:eo-syntax} define a subset
of Eunoia \emph{commands},
which are divided into \emph{standard commands}
and \emph{proof commands}.
%
Hereinafter, a Eunoia \emph{signature} is a list $Δ$
of standard commands, and a \emph{proof script} is a
list $Υ = \prn{\vec δ ⨾ \vec π}$
where $\vec δ$ is a list of standard commands called
the \emph{preamble} and $\vec π$ is a list of
proof commands called the \emph{body} of the script.
%
The preamble of $Υ$ should be understood as the encoding
of an `input problem', and the body understood as a
proof of the unsatisfiability of the problem.

In practice, a Eunoia-friendly solver should have
a trusted signature $Δ$ declaring a bespoke set of
constants and inference rules.
%
The correctness of a generated proof script $Υ$ may then
be checked with respect to $Δ$ using the Ethos tool.
%

\subsection{Commands and their Declarations}
%
We proceed to define an abstract interface for `reading'
information from signatures and proof scripts.
%
This interface is useful for
characterizing the \emph{elaboration}
and \emph{translation} operators
found in \autoref{sec:eo-elab} and
\autoref{sec:results} respectively.

\subsubsection{Constant Declaration.}
%
Let $δ$ be a \emph{(parameterized) constant declaration}
with symbol $s$, parameters $\vec ρ$, and term $t$.
%
We may write $\prn{δ ⊢ s(\vec ρ) : t}$ to express
that the command $δ$ declares $t$ as the \emph{type} of $s$
with respect to a parameters $\vec ρ$.
%
Furthermore, a constant attribute $α$ is given by $δ$,
we may write $\prn{δ ⊢ s(\vec ρ) ∷ α}$.
%

\subsubsection{Macro Definition.}
%
Let $δ$ be a \emph{macro definition} with
symbol $s$, parameters $\vec ρ$, and term $t$.
%
Then $δ$ declares $t$ as the \emph{definiens} of $s$
wrt. $\vec ρ$; written $\prn{δ ⊢ s(\vec ρ) ≔ t}$.
%
If the attribute $\mtt{:type}\ t'$ is given by $δ$,
then $\prn{δ ⊢ s(\vec ρ) : t'}$ also holds.

\subsubsection{Program Declaration.}
%
Let $δ$ be a \emph{program declaration} with
symbol $s$, parameters $\vec ρ$,
$\mtt{:signature}\,\prn{\plur t n}\,t'$,
and cases $\plur c m$.
%
Then, $δ$ declares the type and definiens of $s$
as follows,
where $\plur t n$ are the \emph{domain} types of $s$
and $t'$
is \emph{range}:
% Let $\vec t = \plur t n$ and $\vec r = \plur r m$,
% and $δ ∈ Δ$.
% $(\plurcomma t n)$, \emph{range} $t'$ and
% \emph{cases} $\vec r$ with respect to $\vec ρ$.
%
% More formally, we may write:
  $$
  δ ⊢ s(\vec ρ) : \earr{\plur t n\, t'}
\quad\text{and}\quad
  δ ⊢ s(\vec ρ) ≔ \mbf{cases}[c_1, \ldots, c_m]
$$
% for the type of $s$,
% and
%   $$
% for the cases.
%
\subsubsection{Inference Rule Declaration.}
%
Let $δ$ be a \emph{rule declaration} with symbol $s$,
parameters $\vec ρ$, and \emph{conclusion} $φ$.
%
Also, suppose $δ$ provides
$\mtt{:premises}\ \epar{\plur ψ n}$,
$\mtt{:args}\ \epar{\plur t m}$ with types $\plur τ m$,
and $\mtt{:requires}\
  \epar{\epar{x_1\ y_1}\ldots\epar{x_o\ y_o}}$.
%
First, let $δ$ declare the type and definiens
of an \emph{auxiliary symbol} for $s$ thus:
$$
\begin{array}[t]{l@{\ ⊢\ }l}
δ & s^⭑(\vec ρ) : \earr{\plur τ m\ \bool}
\\[1mm]
δ & s^⭑(\vec ρ) ≔ \mbf{cases}
  [\;
    \epar{{\eapp{s^⭑}{\plur t m}}\ φ}
    ∣
\epar{x_1\ y_1}\ldots\epar{x_k\ y_k}
  \;]
\end{array}
$$
%
Then, the type of $s$ is given by $δ$ with an extended list of parameters%
\footnote{The symbols chosen for $\plur α m$ must be \emph{fresh} with respect to $δ$.
  That is, each $α_i$ is distinct from any symbol occurring in any of the
  terms or parameters supplied by $δ$.}
thus:
  $$δ ⊢ s(\vec ρ, \epar{α_1\ τ_1} \ldots \epar{α_n\ τ_m}) :
  \earr{
    \epar{\mtt{Proof}\ ψ_1} \ldots
    \epar{\mtt{Proof}\ ψ_n}\
    \epar{\mtt{Proof}\ φ^⭑}
  }$$
where $φ^⭑ ≔ \eapp{s^⭑}{\plur α n}$
ensures that a term of type $\eapp{\mtt{Proof}}{φ}$
may only be obtained iff $\prn{α_i = t_i}$
for $1 ≤ i ≤ m$ and $\prn{x_j = y_j}$ for $1 ≤ j ≤ o$.
% and $$.

\subsubsection{Signature Inclusion.}
%
Hereinafter, let $Θ$ be a \emph{(global) environment}
mapping from filepaths to Eunoia signatures.
%
Let $δ$ be an inclusion with valid filepath $μ$.
Then, any judgement $J$ made by a command $δ'$ in $Θ_μ$
is also made by $δ$.
That is:
$$
\text{$∀\,δ' ∈ Θ_μ$,}\quadδ' ⊢ J \quad⟹\quad \epar{\mtt{include}\ μ} ⊢ J
$$

\subsubsection{Proof Scripts.}

Two basic forms of \emph{proof commands} are given,
which are called \emph{assumption} and \emph{step}
respectively.
%
Given $π = \epar{\mtt{assume}\ s\ φ}$ or
$π = \epar{\mtt{step}\ s\ φ\ \ldots}$, we call
$s$ the \emph{name} of $π$ and $φ$ the \emph{conclusion}.
%
In either case, we may write:
$$π ⊢ s : \eapp{\mtt{Proof}}{φ}$$
%
Furthermore, let $π$ be a step with
$\mtt{:rule}\ s'$,
$\mtt{:premises}\ \epar{\vec p}$, and
$\mtt{:args}\ \epar{\vec t}$.
%
Then, $π$ judges the definiens of $s$
as the application of $s'$ to the premises
$\plur p n$ and arguments $\plur t m$, i.e.;
$$π ⊢ s ≔ \eapp{s'}{\plur t m\ \plur p n}$$
%



% iff $π ∈ Π$.

%
% We define We say that a proof scripts is \emph{valid} with respect
% to $Δ$ iff a list of
% commands of the form $\vec δ, \vec π$ proof commands.



% ----------------------------------------------------------
\subsection{Elaboration of Terms}\label{sec:eo-elab}
%
In Eunoia, the `de-sugared' meaning of an application
$\eapp{s}{\vec t}$ depends on the constant attribute
assigned to $s$ within some signature $Δ$.
%
For example, consider a constant declaration $δ$ with:
$$
δ ⊢ \eor : \earr{\bool\ \bool\ \bool}
\quad\text{and}\quad
δ ⊢ \eor ∷ {\rcn \false}
$$
% and $δ ⊢ $.
% $$\dca{\eor}{\earr{\bool\ \bool\ \bool}}$$
%
From the type of $\eor$, we may expect its only valid
uses to be of the form $\eapp{\eor}{t_1\ t_2}$.
%
However, the assignment of the constant attribute means
that the $\mtt{or}$ symbol is treated as
\emph{right-associative} with \emph{nil-terminator} $\false$.
%
Thus, $n$-ary applications of $\eor$ are
\emph{elaborated} to a normal form, e.g.;
%
$$\eapp{\eor}{\ev x\ \ev y\ \ev z}\ ⇢\
  \eapp{\eor}{\ev x\ \eapp{\eor}{\ev y\ \eapp{\eor}{\ev z\ \false}}}$$
%
Furthermore, the elaboration of such applications may
also depend on the attributes of `locally bound' symbols.
For example, consider $\vec ρ$ containing
parameters $\epar{\ev x\ \bool}$,
$\epar{\ev y\ \bool\ \mtt{:list}}$,
and $\epar{\ev z\ \bool}$.
%
Then, observe:
$$
\eapp{\eor}{\ev x\ \ev y\ \ev z}
\ {⇢}
\ \econs\eor{\ev x}{
\econcat{\eor}{\ev y}{\eapp{\eor}{\ev z\ \false}}
}
$$
%
The $\mtt{:list}$ attribute alters the elaboration strategy
under the assumption that $\ev y$ will (eventually) be
substituted for some $\eor$-list.
%
It may be helpful for the reader
to consider the result of substituting
$\ev y ↦ \econs\eor{\ev w}{\false}$
thus:
$$
\begin{array}[t]{c}
\econs\eor{\ev x}{
\econcat{\eor}{\econs\eor{\ev w}{\false}}
  {\eapp{\eor}{\ev z\ \false}}}
\\[1mm] {⇣} \\[1mm]
  \econs\eor{\ev x}{
  \econs\eor{\ev w}{
    \econs\eor{\ev z}{\false}}
  }
\end{array}
$$
%
With the aim of supporting the elaboration strategies
corresponding to the constant attributes given in
\autoref{fig:eo-syntax},
we define an \emph{elaboration} operator below.
%
\begin{definition}
For any symbol $f$ and list of parameters $\vec ρ$,
let $\glue{\vec ρ,f}$ be the binary operator
such that for any terms $t_1$, $t_2$, the following holds:
%
$$
\glue{\vec ρ,f}[t_1,t_2] =
\begin{cases}
  \econcat{f}{t_1}{t_2} & \text{if $\vec ρ ⊢ t_1 ∷ \mtt{:list}$,}
  \\
  \econs{f}{t_1}{t_2} & \text{otherwise.}
\end{cases}
$$
%
Then for any signature $Δ$, let $\elab{Δ,\vec ρ}$
be the least (unary) operator such that for any
symbol $f$ and terms $\vec t = \plur t n$,
the following holds:
$$
\begin{array}[t]{r@{\ =\ }l}
  \elab{Δ,\vec ρ}[f] & f
  \\[1mm]
  \elab{Δ,\vec ρ}[\eapp{f}{\vec t}]
  &
  \begin{cases}
    \foldr(G, t_\nil, \vec t')
    & \tsm{if $Δ ⊢ f ∷ \rcn{t_\nil}$, }
    \\
    \foldl(G', t_\nil, \vec t')
    & \tsm{if $Δ ⊢ f ∷ \lcn{t_\nil}$,}
    \\
    \foldr(G, t'_n, t'_1\ldots t'_{n-1})
    &
    \tsm{if $Δ ⊢ f ∷ \rc$,}
    \\
    \foldl(G', t'_1, t'_2\ldots t'_{n})
    &
    \tsm{if $Δ ⊢ f ∷ \lc$,}
    \\
    \eapp{f}{{t_1'}\ldots{t_n'}}
    &
    \tsm{otherwise.}
    % \econs{f}{t_1}{t_2} & \text{otherwise.}
  \end{cases}
\end{array}
$$
where $G ≔ \glue{\vec ρ, f}[x,y]$,
and $G'(x,y) ≔ G(y,x)$,
and $t_i' ≔ \elab{Δ, \vec ρ}[t_i]$.
% for $1 ≤ i ≤ n$.

\end{definition}
%













% ----------------------------------------------------------


\end{document}


% \subsection{Terms, Types and Attributes}
% %
% The \emph{built-in constants} of Eunoia are given by a fixed
% subset of symbols and denoted by the following expressions:
% $\mtt{Type}$ (the kind of all types),
% $\mtt{->}$ (arrow-type constructor),
% $\mtt{\_}$ (binary function application),
% $\mtt{Bool}$ (the type of Boolean values),
% $\mtt{true}$ and $\mtt{false}$ (terms of type \mtt{Bool}).
