\documentclass[lualatex, compress, 12pt, handout]{beamer}
\usetheme{Arguelles}

% \usepackage{stix}
\usepackage{xcolor}
\usepackage{xcolor-material}
\usepackage{ebproof}
\usepackage{unicode-math-input}
\usepackage{MnSymbol}
% \usepackage{fdsymbol}
\usepackage{stmaryrd}
\usepackage{mathtools}
\usepackage{amssymb}

\usepackage{emoji}
\usepackage{listings}
\usepackage{array}
% \newfontfamily\Alegreya{Alegreya-Regular.otf}[
%   BoldFont=Alegreya-Bold.otf,
%   ItalicFont=Alegreya-Italic.otf,
%   BoldItalicFont=Alegreya-BoldItalic.otf]

\title{\texttt{eo2lp} \textit{--- from Eunoia to LambdaPi}}
\author{Ciarán Dunne and Guillaume Burel}
\institute{ENS Paris-Saclay, INRIA}

\documentclass[lualatex, 12pt, handout]{beamer}
\usetheme{Arguelles}

\usepackage{xcolor}
\usepackage{ebproof}
\usepackage{unicode-math-input}
\usepackage{MnSymbol}
\usepackage{stmaryrd}
\usepackage{mathtools}
\usepackage{amssymb}

\usepackage{emoji}
\usepackage{listings}
\usepackage{array}
% \newfontfamily\Alegreya{Alegreya-Regular.otf}[
%   BoldFont=Alegreya-Bold.otf,
%   ItalicFont=Alegreya-Italic.otf,
%   BoldItalicFont=Alegreya-BoldItalic.otf]

\title{Eunoia to LambdaPi}
\author{Ciarán Dunne and Guillaume Burel}
\institute{ENS Paris-Saclay, INRIA}

\documentclass[lualatex, 12pt, handout]{beamer}
\usetheme{Arguelles}

\usepackage{xcolor}
\usepackage{ebproof}
\usepackage{unicode-math-input}
\usepackage{MnSymbol}
\usepackage{stmaryrd}
\usepackage{mathtools}
\usepackage{amssymb}

\usepackage{emoji}
\usepackage{listings}
\usepackage{array}
% \newfontfamily\Alegreya{Alegreya-Regular.otf}[
%   BoldFont=Alegreya-Bold.otf,
%   ItalicFont=Alegreya-Italic.otf,
%   BoldItalicFont=Alegreya-BoldItalic.otf]

\title{Eunoia to LambdaPi}
\author{Ciarán Dunne and Guillaume Burel}
\institute{ENS Paris-Saclay, INRIA}

\documentclass[lualatex, 12pt, handout]{beamer}
\usetheme{Arguelles}

\usepackage{xcolor}
\usepackage{ebproof}
\usepackage{unicode-math-input}
\usepackage{MnSymbol}
\usepackage{stmaryrd}
\usepackage{mathtools}
\usepackage{amssymb}

\usepackage{emoji}
\usepackage{listings}
\usepackage{array}
% \newfontfamily\Alegreya{Alegreya-Regular.otf}[
%   BoldFont=Alegreya-Bold.otf,
%   ItalicFont=Alegreya-Italic.otf,
%   BoldItalicFont=Alegreya-BoldItalic.otf]

\title{Eunoia to LambdaPi}
\author{Ciarán Dunne and Guillaume Burel}
\institute{ENS Paris-Saclay, INRIA}

\input{talk.macros}


\begin{document}

\frame[plain]{\titlepage}

\section{Background}

\begin{frame}[fragile]{Background}
	\begin{itemize}
		\item<+-> \alert{Satisfiability modulo theories} (SMT) refers to the area
		      of automated reasoning aimed at solving (first-order) satisfiability
		      problems within various mathematical theories.
		      %
		\item<+-> \alert{SMT-LIB} (v2.7) is a formal language
		      used for satisfiability problems and specification
		      of theories.
	\end{itemize}
	%
	\begin{uncoverenv}<+->
		\exxample
		Find $x,y ∈ ℤ$ such that $(x + 2y) = 20$ and $(x - y) = 2$.
		\begin{lstlisting}
(set-logic QF_LIA)
(declare-const x Int)
(declare-const y Int)
(assert (= (+ x (* 2 y)) 20))
(assert (= (- x y) 2))
(check-sat)
(exit)\end{lstlisting}
	\end{uncoverenv}
\end{frame}

\begin{frame}[fragile]
	\begin{uncoverenv}<+->
		For satisfiable specifications,
		\texttt{cvc5} produces an \alert{assignment} for the
		free variables of the input problem.
	\end{uncoverenv}
	\\[3mm]
	\begin{uncoverenv}<+->
		\exxample
		\begin{lstlisting}
(set-logic QF_LIA)
(set-option :produce-models true)
(declare-const x Int)
(declare-const y Int)
(assert (= (+ x (* 2 y)) 20))
(assert (= (- x y) 2))
(check-sat)
(get-model)
(exit)\end{lstlisting}
		\hrule
		\begin{lstlisting}
sat
(
  (define-fun x () Int 8)
  (define-fun y () Int 6)
)\end{lstlisting}
	\end{uncoverenv}
\end{frame}

\begin{frame}[fragile]
	\begin{uncoverenv}<+->
		Some SMT solvers produce \alert{proofs} for unsatisfiable problems.
	\end{uncoverenv}
	\\[3mm]
	\begin{uncoverenv}<+->
		\exxample
		\begin{lstlisting}[basicstyle=\ttfamily\footnotesize]
(set-logic QF_UF)
(set-option :produce-proofs true)
(declare-const p Bool)
(assert (and p (not p)))
(check-sat)
(get-proof)
(exit)\end{lstlisting}
		\hrule
		\begin{lstlisting}[basicstyle=\ttfamily\footnotesize]
unsat
(declare-fun p () Bool)
(assume @p1 (and p (not p)))
(step @p2 :rule and_elim :premises (@p1) :args (1))
(step @p3 :rule and_elim :premises (@p1) :args (0))
(step @p4 false :rule contra :premises (@p3 @p2))\end{lstlisting}
	\end{uncoverenv}
\end{frame}

\begin{frame}[fragile]{Eunoia and Ethos}
	\begin{itemize}
		\item<+->
		      \alert{Eunoia} is an emerging \emph{logical framework} aimed at
		      formalizing the proof systems used by SMT solvers.
		      \begin{itemize}
			      \item[\emoji{flag-united-states}]<+-> development led by Andrew Reynolds, at University of Iowa.
			      \item[\emoji{ghost}]<+-> the `spiritual successor' of the \alert{Alethe} proof format.
			      \item[\emoji{scroll}]<+-> covers theory signatures \& proof scripts.
			      \item[\emoji{check-mark}]<+-> paired with the \alert{Ethos} checker.
		      \end{itemize}

		\item<+-> Extends SMT-LIB by adding:
		      \begin{itemize}
			      \item<+->(dependent) types, parametric polymorphism
			      \item<+->`programs' (i.e., constants with rewrite rules)
			      \item<+->inference rules, commands for building proofs
		      \end{itemize}

	\end{itemize}
\end{frame}

\begin{frame}[fragile]
	% tikz diagram here for the dir-tree of eo/cpc/...
	% tikz diagram for proof output by cvc5, checked by ethos.
	\begin{itemize}
		\item<+-> The \alert{co-operating proof calculus}
		      (i.e., CPC, the proof system of \texttt{cvc5})
		      is formalized as a Eunoia signature $Σ_{\cpc}$.

		\item<+-> Proofs produced by \texttt{cvc5} are
		      Eunoia proof scripts that exclusively use
		      the inference rules from $Σ_{\cpc}$.
	\end{itemize}
\end{frame}

\begin{frame}[fragile]{LambdaPi}
	\begin{itemize}
		\item<+-> Logical framework based
		      on the \emph{$λΠ$-calculus modulo rewriting}.
		      \begin{itemize}
			      \item[\emoji{flag-france}]<+-> development led by Frédéric Blanqui, INRIA Paris-Saclay
			      \item[\emoji{locked}]<+-> small and fast, trusted code base
			      \item[\emoji{robot}]<+-> interactive theorem proving via LSP!
		      \end{itemize}

		\item<+-> Primarily focused on proof assistant \alert{interoperability}.

		\item<+-> \emph{\color{oc-green-9}Goal:}
		      Design a translation procedure $T$ such that:
		      \begin{itemize}
			      \item<+-> if $Σ$ is a Eunoia signature implementing some \alert{logic} $L$,
			      \item<+-> then $T(Σ)$ is a LambdaPi signature also implementing $L$.
			            \\[2mm]
		      \end{itemize}

		\item<+-> Thus, if $Π$ is a valid Eunoia \alert{proof script} depending on $Σ$,
		      then $T(Π)$ should be \alert{well-typed} wrt. $T(Σ)$.
	\end{itemize}
\end{frame}

\begin{frame}[fragile]{Why Eunoia?}
	\begin{itemize}
		\item<+-> What about Alethe?
		      \begin{itemize}
			      \item<+-> inference rules only exist \emph{in papyro}.
			      \item<+-> missing details needed for reconstruction.
			      \item<+->
			            \textbf{Coltellacci and Merz}: extend \alert{Carcara}
			            to allow LambdaPi output.
		      \end{itemize}
	\end{itemize}
\end{frame}

\begin{frame}{}
	\begin{itemize}
		\item<+-> \emph{\color{oc-green-9}Goal:}
		      Design a translation procedure $T$ such that:
		      \begin{itemize}
			      \item<+-> if $Σ$ is a Eunoia signature implementing some \alert{logic} $L$,
			      \item<+-> then $T(Σ)$ is a LambdaPi signature also implementing $L$.
			            \\[2mm]
		      \end{itemize}

		\item<+-> Thus, if $Π$ is a valid Eunoia \alert{proof script} depending on $Σ$,
		      then $T(Π)$ should be \alert{well-typed} wrt. $T(Σ)$.
	\end{itemize}
\end{frame}

\section{Syntax}



\section{Semantics}


\section{Proofs \& Lambdapi}






% TODO.
% smt-lib 3 proposal.
% - dependent types, polymorphism
% - commands
% chad browns notes.
% Eunoia, Ethos.

% SMT-LIB 3 in CC
% proof-irrelevant semantics of CC

% dekuti encoding of CC.
% future proofing:
% translating smt3 concrete syntax into CC does the same for SMT 2.6 and for SMT 2.7.


% some argument about why even though we have a much more expressive logic,
% SMT solvers will only still support a small fragment of it.
% regardless, it is very important for many reasons to

\end{document}



% \begin{frame}[fragile]
% 	\exxample Proof script generated by \textsc{cvc5}.
% 	\begin{lstlisting}
% (declare-const =
%   (-> (! Type :var T :implicit) T T Bool))
% (declare-rule symm ((T Type) (t T) (s T))
%     :premises ((= t s))
%     :conclusion (= s t)
% )
% (declare-type Int ())
% (declare-const a Int)
% (declare-const b Int)
% (assume @p0 (= a b))
% (step @p1 (= b a) :rule symm :premises (@p0))\end{lstlisting}
% 	%
% 	Declaring rules requires adding $\PROP$-level functions, e.g.;
% 	%
% 	$$\msf{symm} :
% 		(\pii {T : \TYPE,\,t : T,\,s : T} {\lift{t = s} → \lift{s = t}})$$
% \end{frame}

% \begin{frame}
% 	\begin{itemize}
% 		\item<+-> \emph{\color{oc-green-9}Goal:}
% 		      Implement a translation from SMT-LIB v3 and Eunoia to
% 		      an OCaml representation of CC.
% 		\item<+-> Use axioms of Theory $𝕌$ for encoding CC in Lambdapi.
% 		      %
% 		      $$
% 			      \begin{array}{l@{\quad}l}
% 				      \eval{\TYPE}_{\lp} ≡ \tt{Set}
% 				       &
% 				      \eval{\lam {x : t_1} {t_2}}_{\lp} ≡ \lam {x' : \eval{t_1}} {\eval{t_2}}
% 				      \\
% 				      \eval{\PROP}_{\lp} ≡ \tt{Prop}
% 				       &
% 				      \eval{\pii {x : t_1}{t_2}}_{\lp} ≡ \eval{t_1}_{\lp} \leadsto^d \eval{t_2}_{\lp}
% 			      \end{array}
% 		      $$
% 	\end{itemize}
% \end{frame}



\begin{document}

\frame[plain]{\titlepage}

\section{Background}

\begin{frame}[fragile]{Background}
	\begin{itemize}
		\item<+-> \alert{Satisfiability modulo theories} (SMT) refers to the area
		      of automated reasoning aimed at solving (first-order) satisfiability
		      problems within various mathematical theories.
		      %
		\item<+-> \alert{SMT-LIB} (v2.7) is a formal language
		      used for satisfiability problems and specification
		      of theories.
	\end{itemize}
	%
	\begin{uncoverenv}<+->
		\exxample
		Find $x,y ∈ ℤ$ such that $(x + 2y) = 20$ and $(x - y) = 2$.
		\begin{lstlisting}
(set-logic QF_LIA)
(declare-const x Int)
(declare-const y Int)
(assert (= (+ x (* 2 y)) 20))
(assert (= (- x y) 2))
(check-sat)
(exit)\end{lstlisting}
	\end{uncoverenv}
\end{frame}

\begin{frame}[fragile]
	\begin{uncoverenv}<+->
		For satisfiable specifications,
		\texttt{cvc5} produces an \alert{assignment} for the
		free variables of the input problem.
	\end{uncoverenv}
	\\[3mm]
	\begin{uncoverenv}<+->
		\exxample
		\begin{lstlisting}
(set-logic QF_LIA)
(set-option :produce-models true)
(declare-const x Int)
(declare-const y Int)
(assert (= (+ x (* 2 y)) 20))
(assert (= (- x y) 2))
(check-sat)
(get-model)
(exit)\end{lstlisting}
		\hrule
		\begin{lstlisting}
sat
(
  (define-fun x () Int 8)
  (define-fun y () Int 6)
)\end{lstlisting}
	\end{uncoverenv}
\end{frame}

\begin{frame}[fragile]
	\begin{uncoverenv}<+->
		Some SMT solvers produce \alert{proofs} for unsatisfiable problems.
	\end{uncoverenv}
	\\[3mm]
	\begin{uncoverenv}<+->
		\exxample
		\begin{lstlisting}[basicstyle=\ttfamily\footnotesize]
(set-logic QF_UF)
(set-option :produce-proofs true)
(declare-const p Bool)
(assert (and p (not p)))
(check-sat)
(get-proof)
(exit)\end{lstlisting}
		\hrule
		\begin{lstlisting}[basicstyle=\ttfamily\footnotesize]
unsat
(declare-fun p () Bool)
(assume @p1 (and p (not p)))
(step @p2 :rule and_elim :premises (@p1) :args (1))
(step @p3 :rule and_elim :premises (@p1) :args (0))
(step @p4 false :rule contra :premises (@p3 @p2))\end{lstlisting}
	\end{uncoverenv}
\end{frame}

\begin{frame}[fragile]{Eunoia and Ethos}
	\begin{itemize}
		\item<+->
		      \alert{Eunoia} is an emerging \emph{logical framework} aimed at
		      formalizing the proof systems used by SMT solvers.
		      \begin{itemize}
			      \item[\emoji{flag-united-states}]<+-> development led by Andrew Reynolds, at University of Iowa.
			      \item[\emoji{ghost}]<+-> the `spiritual successor' of the \alert{Alethe} proof format.
			      \item[\emoji{scroll}]<+-> covers theory signatures \& proof scripts.
			      \item[\emoji{check-mark}]<+-> paired with the \alert{Ethos} checker.
		      \end{itemize}

		\item<+-> Extends SMT-LIB by adding:
		      \begin{itemize}
			      \item<+->(dependent) types, parametric polymorphism
			      \item<+->`programs' (i.e., constants with rewrite rules)
			      \item<+->inference rules, commands for building proofs
		      \end{itemize}

	\end{itemize}
\end{frame}

\begin{frame}[fragile]
	% tikz diagram here for the dir-tree of eo/cpc/...
	% tikz diagram for proof output by cvc5, checked by ethos.
	\begin{itemize}
		\item<+-> The \alert{co-operating proof calculus}
		      (i.e., CPC, the proof system of \texttt{cvc5})
		      is formalized as a Eunoia signature $Σ_{\cpc}$.

		\item<+-> Proofs produced by \texttt{cvc5} are
		      Eunoia proof scripts that exclusively use
		      the inference rules from $Σ_{\cpc}$.
	\end{itemize}
\end{frame}

\begin{frame}[fragile]{LambdaPi}
	\begin{itemize}
		\item<+-> Logical framework based
		      on the \emph{$λΠ$-calculus modulo rewriting}.
		      \begin{itemize}
			      \item[\emoji{flag-france}]<+-> development led by Frédéric Blanqui, INRIA Paris-Saclay
			      \item[\emoji{locked}]<+-> small and fast, trusted code base
			      \item[\emoji{robot}]<+-> interactive theorem proving via LSP!
		      \end{itemize}

		\item<+-> Primarily focused on proof assistant \alert{interoperability}.

		\item<+-> \emph{\color{oc-green-9}Goal:}
		      Design a translation procedure $T$ such that:
		      \begin{itemize}
			      \item<+-> if $Σ$ is a Eunoia signature implementing some \alert{logic} $L$,
			      \item<+-> then $T(Σ)$ is a LambdaPi signature also implementing $L$.
			            \\[2mm]
		      \end{itemize}

		\item<+-> Thus, if $Π$ is a valid Eunoia \alert{proof script} depending on $Σ$,
		      then $T(Π)$ should be \alert{well-typed} wrt. $T(Σ)$.
	\end{itemize}
\end{frame}

\begin{frame}[fragile]{Why Eunoia?}
	\begin{itemize}
		\item<+-> What about Alethe?
		      \begin{itemize}
			      \item<+-> inference rules only exist \emph{in papyro}.
			      \item<+-> missing details needed for reconstruction.
			      \item<+->
			            \textbf{Coltellacci and Merz}: extend \alert{Carcara}
			            to allow LambdaPi output.
		      \end{itemize}
	\end{itemize}
\end{frame}

\begin{frame}{}
	\begin{itemize}
		\item<+-> \emph{\color{oc-green-9}Goal:}
		      Design a translation procedure $T$ such that:
		      \begin{itemize}
			      \item<+-> if $Σ$ is a Eunoia signature implementing some \alert{logic} $L$,
			      \item<+-> then $T(Σ)$ is a LambdaPi signature also implementing $L$.
			            \\[2mm]
		      \end{itemize}

		\item<+-> Thus, if $Π$ is a valid Eunoia \alert{proof script} depending on $Σ$,
		      then $T(Π)$ should be \alert{well-typed} wrt. $T(Σ)$.
	\end{itemize}
\end{frame}

\section{Syntax}



\section{Semantics}


\section{Proofs \& Lambdapi}






% TODO.
% smt-lib 3 proposal.
% - dependent types, polymorphism
% - commands
% chad browns notes.
% Eunoia, Ethos.

% SMT-LIB 3 in CC
% proof-irrelevant semantics of CC

% dekuti encoding of CC.
% future proofing:
% translating smt3 concrete syntax into CC does the same for SMT 2.6 and for SMT 2.7.


% some argument about why even though we have a much more expressive logic,
% SMT solvers will only still support a small fragment of it.
% regardless, it is very important for many reasons to

\end{document}



% \begin{frame}[fragile]
% 	\exxample Proof script generated by \textsc{cvc5}.
% 	\begin{lstlisting}
% (declare-const =
%   (-> (! Type :var T :implicit) T T Bool))
% (declare-rule symm ((T Type) (t T) (s T))
%     :premises ((= t s))
%     :conclusion (= s t)
% )
% (declare-type Int ())
% (declare-const a Int)
% (declare-const b Int)
% (assume @p0 (= a b))
% (step @p1 (= b a) :rule symm :premises (@p0))\end{lstlisting}
% 	%
% 	Declaring rules requires adding $\PROP$-level functions, e.g.;
% 	%
% 	$$\msf{symm} :
% 		(\pii {T : \TYPE,\,t : T,\,s : T} {\lift{t = s} → \lift{s = t}})$$
% \end{frame}

% \begin{frame}
% 	\begin{itemize}
% 		\item<+-> \emph{\color{oc-green-9}Goal:}
% 		      Implement a translation from SMT-LIB v3 and Eunoia to
% 		      an OCaml representation of CC.
% 		\item<+-> Use axioms of Theory $𝕌$ for encoding CC in Lambdapi.
% 		      %
% 		      $$
% 			      \begin{array}{l@{\quad}l}
% 				      \eval{\TYPE}_{\lp} ≡ \tt{Set}
% 				       &
% 				      \eval{\lam {x : t_1} {t_2}}_{\lp} ≡ \lam {x' : \eval{t_1}} {\eval{t_2}}
% 				      \\
% 				      \eval{\PROP}_{\lp} ≡ \tt{Prop}
% 				       &
% 				      \eval{\pii {x : t_1}{t_2}}_{\lp} ≡ \eval{t_1}_{\lp} \leadsto^d \eval{t_2}_{\lp}
% 			      \end{array}
% 		      $$
% 	\end{itemize}
% \end{frame}



\begin{document}

\frame[plain]{\titlepage}

\section{Background}

\begin{frame}[fragile]{Background}
	\begin{itemize}
		\item<+-> \alert{Satisfiability modulo theories} (SMT) refers to the area
		      of automated reasoning aimed at solving (first-order) satisfiability
		      problems within various mathematical theories.
		      %
		\item<+-> \alert{SMT-LIB} (v2.7) is a formal language
		      used for satisfiability problems and specification
		      of theories.
	\end{itemize}
	%
	\begin{uncoverenv}<+->
		\exxample
		Find $x,y ∈ ℤ$ such that $(x + 2y) = 20$ and $(x - y) = 2$.
		\begin{lstlisting}
(set-logic QF_LIA)
(declare-const x Int)
(declare-const y Int)
(assert (= (+ x (* 2 y)) 20))
(assert (= (- x y) 2))
(check-sat)
(exit)\end{lstlisting}
	\end{uncoverenv}
\end{frame}

\begin{frame}[fragile]
	\begin{uncoverenv}<+->
		For satisfiable specifications,
		\texttt{cvc5} produces an \alert{assignment} for the
		free variables of the input problem.
	\end{uncoverenv}
	\\[3mm]
	\begin{uncoverenv}<+->
		\exxample
		\begin{lstlisting}
(set-logic QF_LIA)
(set-option :produce-models true)
(declare-const x Int)
(declare-const y Int)
(assert (= (+ x (* 2 y)) 20))
(assert (= (- x y) 2))
(check-sat)
(get-model)
(exit)\end{lstlisting}
		\hrule
		\begin{lstlisting}
sat
(
  (define-fun x () Int 8)
  (define-fun y () Int 6)
)\end{lstlisting}
	\end{uncoverenv}
\end{frame}

\begin{frame}[fragile]
	\begin{uncoverenv}<+->
		Some SMT solvers produce \alert{proofs} for unsatisfiable problems.
	\end{uncoverenv}
	\\[3mm]
	\begin{uncoverenv}<+->
		\exxample
		\begin{lstlisting}[basicstyle=\ttfamily\footnotesize]
(set-logic QF_UF)
(set-option :produce-proofs true)
(declare-const p Bool)
(assert (and p (not p)))
(check-sat)
(get-proof)
(exit)\end{lstlisting}
		\hrule
		\begin{lstlisting}[basicstyle=\ttfamily\footnotesize]
unsat
(declare-fun p () Bool)
(assume @p1 (and p (not p)))
(step @p2 :rule and_elim :premises (@p1) :args (1))
(step @p3 :rule and_elim :premises (@p1) :args (0))
(step @p4 false :rule contra :premises (@p3 @p2))\end{lstlisting}
	\end{uncoverenv}
\end{frame}

\begin{frame}[fragile]{Eunoia and Ethos}
	\begin{itemize}
		\item<+->
		      \alert{Eunoia} is an emerging \emph{logical framework} aimed at
		      formalizing the proof systems used by SMT solvers.
		      \begin{itemize}
			      \item[\emoji{flag-united-states}]<+-> development led by Andrew Reynolds, at University of Iowa.
			      \item[\emoji{ghost}]<+-> the `spiritual successor' of the \alert{Alethe} proof format.
			      \item[\emoji{scroll}]<+-> covers theory signatures \& proof scripts.
			      \item[\emoji{check-mark}]<+-> paired with the \alert{Ethos} checker.
		      \end{itemize}

		\item<+-> Extends SMT-LIB by adding:
		      \begin{itemize}
			      \item<+->(dependent) types, parametric polymorphism
			      \item<+->`programs' (i.e., constants with rewrite rules)
			      \item<+->inference rules, commands for building proofs
		      \end{itemize}

	\end{itemize}
\end{frame}

\begin{frame}[fragile]
	% tikz diagram here for the dir-tree of eo/cpc/...
	% tikz diagram for proof output by cvc5, checked by ethos.
	\begin{itemize}
		\item<+-> The \alert{co-operating proof calculus}
		      (i.e., CPC, the proof system of \texttt{cvc5})
		      is formalized as a Eunoia signature $Σ_{\cpc}$.

		\item<+-> Proofs produced by \texttt{cvc5} are
		      Eunoia proof scripts that exclusively use
		      the inference rules from $Σ_{\cpc}$.
	\end{itemize}
\end{frame}

\begin{frame}[fragile]{LambdaPi}
	\begin{itemize}
		\item<+-> Logical framework based
		      on the \emph{$λΠ$-calculus modulo rewriting}.
		      \begin{itemize}
			      \item[\emoji{flag-france}]<+-> development led by Frédéric Blanqui, INRIA Paris-Saclay
			      \item[\emoji{locked}]<+-> small and fast, trusted code base
			      \item[\emoji{robot}]<+-> interactive theorem proving via LSP!
		      \end{itemize}

		\item<+-> Primarily focused on proof assistant \alert{interoperability}.

		\item<+-> \emph{\color{oc-green-9}Goal:}
		      Design a translation procedure $T$ such that:
		      \begin{itemize}
			      \item<+-> if $Σ$ is a Eunoia signature implementing some \alert{logic} $L$,
			      \item<+-> then $T(Σ)$ is a LambdaPi signature also implementing $L$.
			            \\[2mm]
		      \end{itemize}

		\item<+-> Thus, if $Π$ is a valid Eunoia \alert{proof script} depending on $Σ$,
		      then $T(Π)$ should be \alert{well-typed} wrt. $T(Σ)$.
	\end{itemize}
\end{frame}

\begin{frame}[fragile]{Why Eunoia?}
	\begin{itemize}
		\item<+-> What about Alethe?
		      \begin{itemize}
			      \item<+-> inference rules only exist \emph{in papyro}.
			      \item<+-> missing details needed for reconstruction.
			      \item<+->
			            \textbf{Coltellacci and Merz}: extend \alert{Carcara}
			            to allow LambdaPi output.
		      \end{itemize}
	\end{itemize}
\end{frame}

\begin{frame}{}
	\begin{itemize}
		\item<+-> \emph{\color{oc-green-9}Goal:}
		      Design a translation procedure $T$ such that:
		      \begin{itemize}
			      \item<+-> if $Σ$ is a Eunoia signature implementing some \alert{logic} $L$,
			      \item<+-> then $T(Σ)$ is a LambdaPi signature also implementing $L$.
			            \\[2mm]
		      \end{itemize}

		\item<+-> Thus, if $Π$ is a valid Eunoia \alert{proof script} depending on $Σ$,
		      then $T(Π)$ should be \alert{well-typed} wrt. $T(Σ)$.
	\end{itemize}
\end{frame}

\section{Syntax}



\section{Semantics}


\section{Proofs \& Lambdapi}






% TODO.
% smt-lib 3 proposal.
% - dependent types, polymorphism
% - commands
% chad browns notes.
% Eunoia, Ethos.

% SMT-LIB 3 in CC
% proof-irrelevant semantics of CC

% dekuti encoding of CC.
% future proofing:
% translating smt3 concrete syntax into CC does the same for SMT 2.6 and for SMT 2.7.


% some argument about why even though we have a much more expressive logic,
% SMT solvers will only still support a small fragment of it.
% regardless, it is very important for many reasons to

\end{document}



% \begin{frame}[fragile]
% 	\exxample Proof script generated by \textsc{cvc5}.
% 	\begin{lstlisting}
% (declare-const =
%   (-> (! Type :var T :implicit) T T Bool))
% (declare-rule symm ((T Type) (t T) (s T))
%     :premises ((= t s))
%     :conclusion (= s t)
% )
% (declare-type Int ())
% (declare-const a Int)
% (declare-const b Int)
% (assume @p0 (= a b))
% (step @p1 (= b a) :rule symm :premises (@p0))\end{lstlisting}
% 	%
% 	Declaring rules requires adding $\PROP$-level functions, e.g.;
% 	%
% 	$$\msf{symm} :
% 		(\pii {T : \TYPE,\,t : T,\,s : T} {\lift{t = s} → \lift{s = t}})$$
% \end{frame}

% \begin{frame}
% 	\begin{itemize}
% 		\item<+-> \emph{\color{oc-green-9}Goal:}
% 		      Implement a translation from SMT-LIB v3 and Eunoia to
% 		      an OCaml representation of CC.
% 		\item<+-> Use axioms of Theory $𝕌$ for encoding CC in Lambdapi.
% 		      %
% 		      $$
% 			      \begin{array}{l@{\quad}l}
% 				      \eval{\TYPE}_{\lp} ≡ \tt{Set}
% 				       &
% 				      \eval{\lam {x : t_1} {t_2}}_{\lp} ≡ \lam {x' : \eval{t_1}} {\eval{t_2}}
% 				      \\
% 				      \eval{\PROP}_{\lp} ≡ \tt{Prop}
% 				       &
% 				      \eval{\pii {x : t_1}{t_2}}_{\lp} ≡ \eval{t_1}_{\lp} \leadsto^d \eval{t_2}_{\lp}
% 			      \end{array}
% 		      $$
% 	\end{itemize}
% \end{frame}



\begin{document}

\frame[plain]{\titlepage}

\section{Background}


% need to mention why Alethe precedes Eunoia.
%
% \begin{frame}[fragile, plain]{Why Eunoia?}
% 	\begin{itemize}
% 		\item<+-> What about Alethe?
% 		      \begin{itemize}
% 			      \item<+-> inference rules only exist \emph{in papyro}.
% 			      \item<+-> missing details needed for reconstruction.
% 			      \item<+->
% 			            \textbf{Coltellacci and Merz}: extend \alert{Carcara}
% 			            to support LambdaPi output.
% 		      \end{itemize}
% 	\end{itemize}
% \end{frame}
%


% Cite Reynolds talk.
% Also need to discuss LFSC.
%

\begin{frame}[fragile]{Eunoia and Ethos}
	\begin{itemize}
		\item<+->
		      \alert{Eunoia} is an emerging \emph{logical framework} aimed at
		      formalizing the proof systems used by SMT solvers.
		      \begin{itemize}
			      \item[\emoji{flag-united-states}]<+-> development led by Andrew Reynolds, at University of Iowa.
			      \item[\emoji{ghost}]<+-> the `spiritual successor' of the \alert{Alethe} proof format.
			      \item[\emoji{scroll}]<+-> covers theory signatures \& proof scripts.
			      \item[\emoji{check-mark}]<+-> paired with the \alert{Ethos} checker.
		      \end{itemize}

		\item<+-> Extends SMT-LIB by adding:
		      \begin{itemize}
			      \item<+->(dependent) types, parametric polymorphism,
			      \item<+->`programs' (i.e., constants with rewrite rules),
			      \item<+->inference rule declarations,
			      \item<+->commands for building proof scripts.
		      \end{itemize}
	\end{itemize}
\end{frame}




% How are ATPs currently connected to LambdaPi. `why3`?

% - 'formal semantics' for Eunoia.
% - typecheck the signature for Eunoia proof systems
% - proof check the proofs for Eunoia proof scripts.

% - verification of CPC itself wrt. some other logic.

% - using cvc5 as a ATP backend for LambdaPi logics.
\begin{frame}[fragile]{LambdaPi}
	\begin{itemize}
		\item<+-> Logical framework based
		      on the \emph{$λΠ$-calculus modulo rewriting}.
		      \begin{itemize}
			      \item[\emoji{flag-france}]<+-> development led by Frédéric Blanqui, INRIA Paris-Saclay
			      \item[\emoji{locked}]<+-> small code base, trusted foundations.
			      \item[\emoji{leopard}]<+-> fast typechecker.
			      \item[\emoji{robot}]<+-> interactive theorem proving via LSP!
		      \end{itemize}

		\item<+-> Primarily focused on proof assistant \alert{interoperability}.
	\end{itemize}
	\vspace{2mm}
	\begin{figure}
		\centering
		\begin{tikzpicture}[scale=0.8]
			\path (0,0) node (lp) {{\large\LambdaPi}}
			(5,0) node (isa) {\small\sysname{Isabelle}}
			(3,-2) node (agda) {\small\sysname{Agda}}
			% (0,2) node (ze) {Zenon \& ArchSAT}
			(3,2) node (hol) {\small\sysname{HOL}}
			(0,-2) node (pvs) {\small\sysname{PVS}}
			(-3,-2) node (k) {\small\sysname{$\mathbb{K}$-framework}}
			(-5,0) node (mat) {\small\sysname{Matita}}
			(0,2) node (coq) {\small\sysname{Rocq}}
			(-3,2) node (lean) {\small\sysname{Lean}}
			% (0,2) node [draw, dashed, purple] (smt) {\color{purple}\textbf{SMT}}
			;
			% \draw[->,red, thick] (smt) -- (lp) node[midway,sloped,above]
			% {\footnotesize{Carcara}};
			\draw[->] (lean) -- (lp) node[midway,sloped,above] {\scriptsize\toolname{lean2dk}};
			\draw[->] (isa) -- (lp) node[midway,sloped,above] {\scriptsize\toolname{isabelle\_dk}};
			\draw[->] (agda) -- (lp) node[midway,sloped,above] {\scriptsize\toolname{agda2dk}};
			% \draw[<->] (ze) -- (lp) node[midway,sloped,above] {};
			\draw[<->] (hol) -- (lp) node[midway,sloped,above] {\scriptsize\toolname{hol2dk}};
			\draw[<->] (mat) -- (lp) node[midway,sloped,above] {\scriptsize\toolname{Krajono}};
			\draw[->] (pvs) -- (lp);
			\draw[->] (k) -- (lp) node[midway,sloped,above] {\scriptsize\toolname{KaMeLo}};
			\draw[<->] (coq) -- (lp)  node[midway,sloped,above] {\scriptsize\toolname{vodk}};
		\end{tikzpicture}
		\label{fig:interop}
	\end{figure}
\end{frame}

\begin{frame}{The Co-operating Proof Calculus}
	% tikz diagram here for the dir-tree of eo/cpc/...
	% tikz diagram for proof output by cvc5, checked by ethos.
	\begin{itemize}
		\item<+-> The \alert{co-operating proof calculus} (CPC)
		      is \texttt{cvc5}'s proof system.
		      \begin{itemize}
			      \item[\emoji{sparkles}]<+-> formalized as a Eunoia signature $Σ_{\cpc}$.
			      \item[\emoji{elephant}]<+-> not small (${>}\ 600$ inference rules).
			      \item[\emoji{puzzle-piece}]<+-> some rules take arguments, some have side-conditions.
		      \end{itemize}

		\item<+-> Proofs produced by \texttt{cvc5} are
		      Eunoia \alert{proof scripts} that exclusively use
		      the rules from $Σ_{\cpc}$.
	\end{itemize}
	%
\end{frame}

\begin{frame}[plain, fragile]
	\begin{uncoverenv}<+->%
		\exxample A CPC rule for \alert{elimination on $n$-ary conjunctions},
		where $φ_1 \ldots φ_n$ are formulas and $i ∈ ℕ$.
		$$
			\begin{prooftree}
				\hypo{(φ_1 ∧ \ldots ∧ φ_n) ∣ i}
				\infer1{φ_i}
			\end{prooftree}
			\quad\mcomment{(and\_elim)}
		$$
	\end{uncoverenv}
	\\
	\begin{uncoverenv}<+->%
		The rule is \alert{formalized} in Eunoia thus:
		\begin{lstlisting}
(declare-rule and_elim ((Fs Bool) (i Int))
    :premises (Fs)
    :args (i)
    :conclusion (eo::list_nth and Fs i)
)\end{lstlisting}
	\end{uncoverenv}
\end{frame}

% \begin{frame}[fragile, plain]{Background}
% 	\begin{itemize}
% 		\item<+-> \alert{Satisfiability modulo theories} (SMT)
% 		      is an area of automated reasoning aimed at
% 		      developing tools for deciding the satisfiability
% 		      of formulas within a curated selection of
% 		      first-order mathematical theories.
% 		      %
% 		\item<+-> \alert{SMT-LIB} (v2.7) is a standardized
% 		      formal language for satisfiability problems
% 		      and theory documentation.
% 	\end{itemize}
% 	%
% 	\begin{uncoverenv}<+->
% 		\exxample
% 		Find $x,y ∈ ℤ$ such that $(x + 2y) = 20$ and $(x - y) = 2$.
% 		\begin{lstlisting}
% (set-logic QF_LIA)
% (declare-const x Int)
% (declare-const y Int)
% (assert (= (+ x (* 2 y)) 20))
% (assert (= (- x y) 2))
% (check-sat)
% (exit)\end{lstlisting}
% 	\end{uncoverenv}
% \end{frame}

% \begin{frame}[fragile, plain]
% 	\begin{uncoverenv}<+->
% 		\exxample
% 		The example problem is \alert{satisfiable}.
% 		Hence, \texttt{cvc5} can return a \alert{solution}
% 		assigning concrete values to free variables.
% 	\end{uncoverenv}
% 	%
% 	\begin{uncoverenv}<+->
% 		\begin{lstlisting}
% (set-logic QF_LIA)
% (set-option :produce-models true)
% (declare-const x Int)
% (declare-const y Int)
% (assert (= (+ x (* 2 y)) 20))
% (assert (= (- x y) 2))
% (check-sat)
% (exit)
% \end{lstlisting}
% 		\hrule
% 		\begin{lstlisting}
% sat
% (
%   (define-fun x () Int 8)
%   (define-fun y () Int 6)
% )\end{lstlisting}
% 	\end{uncoverenv}
% \end{frame}

\begin{frame}[fragile, plain]
	\begin{uncoverenv}<+->
		\exxample
		The following problem is \alert{unsatisfiable}.
		In this case, \texttt{cvc5} can provide a \alert{proof} demonstrating this.
	\end{uncoverenv}
	\begin{uncoverenv}<+->
		\begin{lstlisting}[basicstyle=\ttfamily\footnotesize]
(set-logic QF_UF)
(set-option :produce-proofs true)
(declare-const p Bool)
(assert (and p (not p)))
(check-sat)
(get-proof)
(exit)\end{lstlisting}
	\end{uncoverenv}
	\hrule
	\begin{uncoverenv}<+->
		\begin{lstlisting}[basicstyle=\ttfamily\footnotesize]
unsat
(declare-fun p () Bool)
(assume @p1 (and p (not p)))
(step @p2 :rule and_elim :premises (@p1) :args (1))
(step @p3 :rule and_elim :premises (@p1) :args (0))
(step @p4 false :rule contra :premises (@p3 @p2))\end{lstlisting}
	\end{uncoverenv}
	% \begin{uncoverenv}<+->
	% 	\textbf{What is the formal meaning of this proof?}
	% \end{uncoverenv}
	% \\
	% \begin{uncoverenv}<+->
	% 	\textbf{In general, what rules may be used in a proof?}
	% \end{uncoverenv}
\end{frame}


\begin{frame}{}
	\begin{itemize}
		\item<+-> \emph{\color{oc-green-9}Goal:}
		      Design a translation procedure $T$ such that;
		      \begin{itemize}
			      \item<+-> if $Σ$ is a Eunoia signature implementing some \alert{logic} $L$,
			      \item<+-> then $T(Σ)$ is a LambdaPi signature also implementing $L$.
			            \\[2mm]
		      \end{itemize}

		\item<+-> Thus, if $Π$ is a valid Eunoia \alert{proof script} depending on $Σ$,
		      then $T(Π)$ should be \alert{well-typed} wrt. $T(Σ)$.
	\end{itemize}
\end{frame}



\section{Eunoia}

\setbeamercolor*{structure}{fg=MaterialGreen400}
\setbeamercolor*{background canvas}{bg=MaterialGreen400}
\begin{frame}[standout]
	\centering\Huge
	\textbf{Eunoia}
\end{frame}

\begin{frame}
	\begin{itemize}
		\item<+-> Fix a set of \alert{symbols} $𝒮$, and let $s ∈ 𝒮$.
		\item<+-> Define $\mbf{eo}$ as the set of Eunoia \alert{expressions} thus:
		      $$
			      \begin{array}{l@{\ }c@{\ }l@{\quad}l}
				      e ∈ \mbf{eo} & ⩴ & s                             & \mcomment{(symbol)}      \\
				                   & ∣ & \paren{{s}\ e_1\ \ldots\ e_n} & \mcomment{(application)}
			      \end{array}
		      $$
		\item<+-> In general, expressions are either:
		      \begin{itemize}
			      \item<+-> \alert{terms} (e.g., \mtt{true}, \mtt{false})
			      \item<+-> \alert{types}
			            (e.g.,
			            \mtt{Bool},
			            \paren{\mtt{->}\ \mtt{Bool}\ \mtt{Bool}}
			            )
			      \item<+-> \alert{kinds}
			            (e.g., \mtt{Type}, \paren{\mtt{->}\ \mtt{Type}\ \mtt{Type}})
		      \end{itemize}
	\end{itemize}
\end{frame}


% Eunoia no longer has type definitions/declarations.
\begin{frame}[fragile]
	Eunoia has \alert{type declarations}.
	$$\paren{\mtt{declare-type}\ s\ \paren{\plur e n}}$$
	\\[2mm]
	\begin{uncoverenv}<+->
		\exxample The \mtt{Array} symbol declared a (binary) type constructor.
		\begin{lstlisting}
(declare-type Array (Type Type))\end{lstlisting}
	\end{uncoverenv}
\end{frame}

\begin{frame}[fragile]
	Eunoia has \alert{constant declarations} of the form:
	$$\paren{\mtt{declare-const}\ s\ e\ \maybe{α}}$$
	where $α$ is a \alert{constant attribute}. i.e.,
	$$
		\begin{array}{l@{\ }c@{\ }l}
			α ∈ \mbf{attr_c} & ⩴   & \mtt{:right-assoc} ∣ \mtt{:right-assoc-nil}\some{t}                        \\
			                 & {∣} & \mtt{:left-assoc} ∣ \mtt{:left-assoc-nil}\some{t}                          \\
			                 & {∣} & \mtt{:chainable}\some{s} ∣ \mtt{:pairwise}\some{s} ∣ \mtt{:binder}\some{s}
			\\[2mm]
		\end{array}
	$$

	\begin{uncoverenv}<+->
		\exxample Declare \mtt{and} right-associative,
		with \emph{nil terminator} \mtt{true}.
		\begin{lstlisting}
(declare-const and (-> Bool Bool Bool)
  :right-assoc-nil true
)\end{lstlisting}
	\end{uncoverenv}
\end{frame}

% TODO. precisely state elaboration algorithm, including :list.
\begin{frame}[fragile]
	\begin{uncoverenv}<+->
		\exxample Declare \mtt{and} right-associative,
		with \emph{nil terminator} \mtt{true}.
		\begin{lstlisting}
(declare-const and (-> Bool Bool Bool)
  :right-assoc-nil true
)\end{lstlisting}
	\end{uncoverenv}
	\begin{uncoverenv}<+->
		The following $n$-ary application of \mtt{and} is elaborated thus:\\
		$$\text{\smt{(and p q r)}}
			⟹ \text{\smt{(and p (and q (and r true)))}}$$
		% tikz diagram of
		%   `(and p q r)`
		% elaborating to
		%   `(and p (and q (and r true)))`
	\end{uncoverenv}
\end{frame}

% TODO. worry about :implicit variables.
\begin{frame}[fragile]
	We can also declare \alert{parameterized constants}:
	$$ \paren{\mtt{declare-parameterized-const}
		\ s
		\  \paren{\plur ρ n}
		\  {e}
		\ \maybe{α}
		} $$
	\begin{uncoverenv}<+->
		where $ρ$ is a \alert{(typed) parameter}. i.e.,
		$$
			\begin{array}{l@{\ }c@{\ }l}
				ρ ∈ \mbf{param}  & ⩴ & \paren{s\ t\ \maybe{ν}}       \\
				ν ∈ \mbf{attr_v} & ⩴ & \mtt{:implicit} ∣ \mtt{:list}
			\end{array}
		$$

	\end{uncoverenv}
	\begin{uncoverenv}<+->
		\exxample Implicit type parameter
		and \mtt{:chainable} attribute.
		\begin{lstlisting}
(declare-parameterized-const =
  ((A Type :implicit)) (-> A A Bool)
  :chainable and
)\end{lstlisting}
	\end{uncoverenv}
\end{frame}

\begin{frame}[fragile]
	\begin{uncoverenv}<+->
		\exxample Implicit type parameter
		and \mtt{:chainable} attribute.
		\begin{lstlisting}
(declare-parameterized-const =
  ((A Type :implicit)) (-> A A Bool)
  :chainable and
)\end{lstlisting}
	\end{uncoverenv}
	\begin{uncoverenv}<+->%
		The following $n$-ary application of \mtt{=} is elaborated thus:
		$$ \text{\smt{(= x y z)}} ⟹ \text{\smt{(and (= x y) (= y z))}}$$
	\end{uncoverenv}
\end{frame}


% app : sym → eo list → eo;
%
% $to_clause (app "or" [F1;F2]) ↪ ....
%
% $to_clause (or $F1 $F2) ↪ ....
%
%
% TODO. discuss inference of :type t
\begin{frame}[fragile]
	Eunoia can \alert{define} symbols (with an optional type annotation):
	$$\paren{\mtt{define}
			\ s
			\  \paren{\plur ρ n}
			\ e
			\ \maybe{\mtt{:type}\ t}
		}$$
	\\[2mm]
	\begin{uncoverenv}<+->%
		\exxample Some definition from \texttt{cpc/rules/Booleans.eo}.
		\begin{lstlisting}
(define $remove_maybe_self ((l Bool) (C Bool))
  (eo::ite (eo::eq l C) false (eo::list_erase or C l))
)\end{lstlisting}
	\end{uncoverenv}
\end{frame}

% \begin{frame}
% 	Eunoia has \alert{built-in programs}:
% 	$$\eo{list\_concat} : 𝒮 → \mbf{eo} → \mbf{eo} → \mbf{eo}$$
% \end{frame}

% Need to discuss programs that involve applications of pattern vars.
\begin{frame}[fragile]
	Eunoia has user-defined \alert{programs}.
	$$
		\left(\begin{array}{l}
				\mtt{program}\ s\ \paren{\plur ρ n}                   \\
				\quad \mtt{:signature}\ \paren{\plur t m}\ t'         \\
				\quad\paren{\paren{e_1\ e'_1}\ldots\paren{e_k\ e'_k}} \\
			\end{array}\right)
	$$
	\\[2mm]
	\begin{uncoverenv}<+->
		\exxample Some program from \texttt{cpc/rules/Booleans.eo}.
		\begin{lstlisting}
(program $to_clause
  ((F1 Bool) (F2 Bool :list))
  :signature (Bool) Bool
  (
    (($to_clause (or F1 F2)) (or F1 F2))
    (($to_clause false)      false)
    (($to_clause F1)         (or F1))
  )
)\end{lstlisting}
	\end{uncoverenv}
\end{frame}


% TODO. discuss future work for assume-push and step-pop.
% use LambdaPi proof blocks.
\begin{frame}[fragile]
	Eunoia has \alert{rule declarations}.
	$$
		\left(
		\begin{array}{l}
				\mtt{declare-rule}\ s\ \paren{\plur ρ n}\
				\\ \quad
				% \maybe{\mtt{:assumption}\ e}\
				% \\ \quad
				\maybe{\mtt{:premises}\ \paren{\plur φ m}}\
				\\ \quad
				\maybe{\mtt{:args}\ \paren{\plur e k}}\
				\\ \quad
				% \maybe{\mtt{:requires}\ {\many{e,e'}}}\
				% \\ \quad
				\mtt{:conclusion}\ {ψ}
			\end{array}
		\right)
	$$
	\\[2mm]
	\begin{uncoverenv}<+->
		\exxample Resolution rule from \texttt{cpc/rules/Booleans.eo}.
		\begin{lstlisting}
(declare-rule resolution
  ((C1 Bool) (C2 Bool) (pol Bool) (L Bool))
  :premises (C1 C2)
  :args (pol L)
  :conclusion ($resolve C1 C2 pol L)
)\end{lstlisting}
	\end{uncoverenv}
\end{frame}

\begin{frame}[fragile]
	For \alert{proof scripts}, we have two main commands:
	$$	\begin{array}[t]{r@{\ }l}
			π ∈ \mbf{prf}
			 &
			\begin{array}[t]{c@{\ }l}
				{⩴} & \left(\mtt{assume}\ s\ φ\right) \\[2mm]
				{∣} &
				\left(\begin{array}{@{}l@{}}
						      \mtt{step}\ s\ \maybe{ψ}\ \mtt{:rule}\ s'        \\
						      \quad \maybe{\mtt{:premises}\ \paren{\plur φ n}} \\
						      \quad \maybe{\mtt{:args}\ \paren{\plur e m}}     \\
					      \end{array}\right)
				\\
				{∣} & \ldots

				% {∣} & \paren{\mtt{assume-push}\ s\ φ}
				% \\
				% {∣} & \paren{\mtt{step-pop}\ s\ \maybe{e}\
				% \paren{\mtt{:rule}\ {s_r}}\
				% \maybe{\mtt{:premises}\ {\vec φ}}\
				% \maybe{\mtt{:args}\ {\vec e}}
				% }
			\end{array}
		\end{array}$$
	\\[2mm]
	% assume, step, assume-push, step-pop.
	\begin{uncoverenv}<+->
		\exxample
		\begin{lstlisting}
(assume @p1 (and p (not p)))
(step @p2 :rule and_elim :premises (@p1) :args (1))
(step @p3 :rule and_elim :premises (@p1) :args (0))
(step @p4 false :rule contra :premises (@p3 @p2))\end{lstlisting}
	\end{uncoverenv}
\end{frame}

% \begin{frame}
% 	In general, expressions are either \alert{terms}, \alert{types}, or \alert{kinds}.
% 	\begin{itemize}
% 		\item<+-> Fix a domain $𝒟$ of \alert{individuals}.
% 		      \begin{itemize}
% 			      \item<+-> each term denotes some $x ∈ 𝒟$, or some function $f ∈ 𝒟 ⇀ 𝒟$.
% 			      \item<+-> e.g., $\mtt{true}$ and $\mtt{false}$ are terms denoting $\mbf{1}$ and $\mbf{0}$ resp.
% 		      \end{itemize}


% 		      partitioned by $B_1 \ldots B_n$,
% 		      $$ 𝒟 = B_1 ∪ \ldots ∪ B_n $$
% 		      \begin{itemize}
% 			      \item<+-> each term denotes either an individual or a function between individuals,
% 			      \item<+-> e.g., $\mtt{Bool}$ is a base type denoting $\{\mbf{1},\mbf{0}\}$,
% 		      \end{itemize}
% 		\item<+-> Let $𝒯$ be the domain for \alert{types}, where:
% 		      $$ τ ∈ 𝒯 ⩴ B_i ∣ (τ_1 → τ_2)$$
% 		\item<+-> Let $𝒦$ be the domain for \alert{kinds}, where:
% 		      $$ κ ∈ 𝒦 ⩴ 𝒯 ∣ (κ_1 → κ_2) $$

% 	\end{itemize}


% \begin{itemize}
% 	\item<+-> $\paren{\mtt{->}\ \mtt{Bool}\ \mtt{Bool}}$ denotes the function space $\{\mbf{1},\mbf{0}\} → \{\mbf{1},\mbf{0}\}$.
% \end{itemize}
% \begin{itemize}
% 	\item<+-> $\mtt{Type}$ is a kind, denoting the set of all types.
% 	\item<+-> $\paren{\mtt{->}\ \mtt{Type}\ \mtt{Type}}$ is a kind, denoting the set of type constructors.
% \end{itemize}



% \end{uncoverenv}

\section{LambdaPi}

\setbeamercolor*{structure}{fg=MaterialPurple400}
\setbeamercolor*{background canvas}{bg=MaterialPurple400}
\begin{frame}[standout]
	\centering\Huge
	\textbf{LambdaPi}
\end{frame}

% TODO. explicit arguments in applications.
% implicit bindings for Lam and Pi.
\begin{frame}
	\begin{itemize}
		\item<+-> LambdaPi \alert{terms} are those of the $λΠ$-calculus.
		      $$
			      \begin{array}{l}
				      t ∈ \mbf{term_{lp}} ⩴\ x\ ∣\ t_1 ⋅ t_2\ ∣\ \binddot{λ}{x : t_1}{t_2}\ ∣\ \binddot{Π}{x : t_1}{t_2}
			      \end{array}
		      $$
		      \\[5mm]
		\item<+-> \alert{Symbols} are declared thus:
		      $$\mtt{symbol}\ s\ \any{ρ} : t;$$
		      where $ρ$ ranges over LambdaPi \alert{parameters}:
		      $$ρ ∈ \mbf{param_{lp}} ⩴ (x : t) ∣ [x : t]$$
	\end{itemize}
\end{frame}

% TODO. Discuss type preservation of rewrite rules.
% explain the 'sequential' modifier.
\begin{frame}
	\begin{itemize}
		\item<+-> Symbols can also be \alert{defined}:
		      $$\mtt{symbol}\ s\ \maybe{{:}\,t} ≔ t';$$
		      \begin{uncoverenv}<+->
			      Note that providing the type of $s$ is optional.
		      \end{uncoverenv}
		      \\[5mm]
		\item<+->
		      \alert{Rewrite rules} are declared as follows:
		      $$\mtt{rule}\ r\ \any{\mtt{with}\ r'};$$
		      Where $r$ ranges over $\mbf{rw} ⩴ (t ↪ t')$.
	\end{itemize}
\end{frame}

\begin{frame}
	\begin{uncoverenv}<+->%
		\alert{Type universes} \emph{a la Tarski}; closed under $({\leadsto})$.
		$$
			\begin{array}{c}
				\mtt{Set} : \mtt{TYPE};            \qquad
				\mtt{El} : \mtt{Set} → \mtt{TYPE}; \\[2mm]
				(\leadsto) : \mtt{Set} → \mtt{Set} → \mtt{Set};
			\end{array}
		$$
	\end{uncoverenv}
	\\
	\begin{uncoverenv}<+->%
		Proofs are encoded similarly:
		$$
			\mtt{Prop} : \mtt{TYPE}; \qquad \mtt{Prf} : \mtt{Prop} → \mtt{TYPE};
		$$
	\end{uncoverenv}
	\\
	\begin{uncoverenv}<+->
		\exxample%
		$$
			\begin{array}{l@{\ \ :\ \ }l}
				\mtt{symbol}\ \paren{{=}}\ [a : \mtt{Set}]                   & \mtt{El}\,\paren{a \leadsto a \leadsto \mtt{Bool}};
				\\
				\mtt{symbol}\ \msf{refl}\ [a : \mtt{Set}]\ [x : \mtt{El}\,a] &
				\mtt{Prf}\paren{x = x};
			\end{array}
		$$
	\end{uncoverenv}
\end{frame}

\section{Translation}

\setbeamercolor*{structure}{fg=MaterialRed400}
\setbeamercolor*{background canvas}{bg=MaterialRed400}
\begin{frame}[standout]
	\centering\Huge
	\textbf{Translation}
\end{frame}

\newcommand{\tra}[1]{\llbracket\,#1\,\rrbracket}
\newcommand{\ki}[1]{{\tra{#1}}_{\mbf{ki}}}
\newcommand{\ty}[1]{{\tra{#1}}_{\mbf{ty}}}
\newcommand{\tm}[1]{{\tra{#1}}_{\mbf{tm}}}
\newcommand{\rawapp}[2]{\paren{\mtt{\_}\ {#1}\ {#2}}}

\begin{frame}
	\textbf{Goal:}
	Given a Eunoia signature $Σ$,
	generate the corresponding LambdaPi signature $T(Σ)$.

	\begin{itemize}
		\item<+-> Process each command in $Σ$,
		      updating an environment $Θ$ as we go:
		      $$T_{Θ}(c\,{;}\,Σ) = c\,{;}\,T_{Θ'}(Σ)$$

		\item<+-> Our translation tool \texttt{eo2lp} is written in OCaml.
		\item<+-> The following is a high-level overview.
	\end{itemize}
\end{frame}

% \begin{frame}
% 	We begin with an encoding of Eunoia's built-in symbols:
% 	$$\mtt{symbol}\ \iden{Bool} : Set;$$
% \end{frame}

\begin{frame}
	Expressions are first elaborated with $\mbf{elab}_{γ} : \mbf{eo} → \mbf{eo}$.
	$$γ : 𝒮 ⇀ (\mbf{attr_c} ∪ \mbf{attr_v})$$
	Where $γ$ \alert{attributes} of symbols during translation.
	\begin{itemize}
		\item<+-> Eunoia has a built-in symbol $\mtt{\_}$ for (higher-order) application.

		\item<+-> The \alert{default} elaboration strategy is to left-fold:
		      $$
			      \begin{array}{r@{\ =\ }l}
				      \mbf{elab}_{γ}\paren{s\ e_1\ \ldots\ e_n}
				       & ((s ∗ e_1) ∗ \ldots ∗ e_n)
				      \\
				       & \paren{\mtt{\_}\ \paren{\ldots\paren{\mtt{\_}\ s\ e_1}\ldots}\ \ e_n}
			      \end{array}
		      $$

		\item<+-> In general, strategy depends on attributes, e.g.,
		      $$ \mbf{elab}_{γ}\paren{\mtt{and}\ p\ q \ r} =
			      \bapp{and}{p}{
				      \paren{
					      \bapp{and}{q}{
						      \paren{
							      \bapp{and}{r}{\mtt{false}}
						      }
					      }
				      }
			      }
		      $$

		      % $$
		      %  \begin{array}{l@{\ }c@{\ }l}
		      %   α ∈ \mbf{attr_c} & ⩴   & \mtt{:right-assoc} ∣ \mtt{:right-assoc-nil}\some{t}                        \\
		      %                    & {∣} & \mtt{:left-assoc} ∣ \mtt{:left-assoc-nil}\some{t}                          \\
		      %                    & {∣} & \mtt{:chainable}\some{s} ∣ \mtt{:pairwise}\some{s} ∣ \mtt{:binder}\some{s}
		      %   \\[2mm]
		      %   ν ∈ \mbf{attr_v} & ⩴   & \mtt{:implicit} ∣ \mtt{:list}
		      %  \end{array}
		      % $$
	\end{itemize}
\end{frame}

% \begin{frame}
% 	Track target \alert{universe} of symbols by $μ : 𝒮 ⇀ \{ \mbf{tm}, \mbf{ty}, \mbf{ki} \}$,
% 	e.g., $$ μ(\mtt{Bool}) = \mbf{ty} \qquad μ(\mtt{Type}) = \mbf{ki} $$
% 	%
% 	\begin{uncoverenv}<+->%
% 		Extend to Eunoia expressions thus:
% 		$$
% 			\begin{array}{r@{\ =\ }l}
% 				μ\paren{\mtt{->}\ {\plur e n}} &
% 				\begin{dcases}
% 					\ \mbf{ki} & \text{if $e_n = \mtt{Type}$,} \\
% 					\ \mbf{ty} & \text{otherwise}
% 				\end{dcases}
% 				\\[5mm]
% 				μ\paren{s\ \plur e n}          &
% 				μ(s)
% 			\end{array}
% 		$$
% 	\end{uncoverenv}
% \end{frame}

\begin{frame}
	\begin{uncoverenv}<+->
		Translate \alert{kinds} into LambdaPi types via
		${\tra\cdot}_{\mbf{ty}} : \mbf{eo} → \mbf{lp}$;
		$$
			\begin{array}{r@{\ =\ }l}
				\ty{\mtt{Type}}                        & \mtt{Set}
				\\[2mm]
				% \begin{dcases}
				% 	\ \mtt{Set} & \text{if}\ $s = $,
				% 	\\
				% 	\ \ty{s}    & \text{otherwise}
				% \end{dcases}
				\ty{\paren{\paren{\mtt{->}} ∗ e} ∗ e'} &
				\ty{e} → \ty{e'}
				\\[2mm]
				%
				\ty{e'}                                &
				\mtt{El}\ \tm{e'}
			\end{array}
		$$
		\begin{uncoverenv}<+->%
			\exxample Consider translating the following Eunoia kind.
			$$
				\begin{array}{r@{\ =\ }l}
					\ty{\paren{\mtt{->}\ \mtt{Int}\ \mtt{Type}}}
					 &
					\ty{\paren{\mtt{->} ∗ \mtt{Int}} ∗ \mtt{Type}}
					\\[2mm]
					 & \ty{\mtt{Int}} → \ty{\mtt{Type}}
					\\[2mm]
					 & \mtt{El}\,\tm{\mtt{Int}} → \mtt{Set}
				\end{array}
			$$
		\end{uncoverenv}
		% where $u_i ≔ μ(e_i)$ for each $i ∈ \{1 \ldots n\}$.
	\end{uncoverenv}
\end{frame}

\begin{frame}[fragile]
	\begin{uncoverenv}<+->%
		Now, we can easily translate type declarations:
		$$
			\begin{array}{c}
				\tra{\paren{\mtt{declare-type}\ t\ \paren{\plur e n}}}
				\\[2mm]
				⇓
				\\[2mm]
				\mtt{symbol}\ {\iden t} : \ty{\paren{\mtt{->}\ {\plur e n}\ \mtt{Type}}};
				\\[2mm]
				⇓
				\\[2mm]
				\mtt{symbol}\ {\iden t} : \ty{e_1} → \ldots → \ty{e_n} → \mtt{Set};
			\end{array}
		$$
	\end{uncoverenv}
	\\
	\begin{uncoverenv}<+->%
		\exxample
		\begin{lstlisting}
(declare-type Array (Type Type))\end{lstlisting}
		\begin{lstlisting}
symbol {|Array|} : Set → Set;\end{lstlisting}
	\end{uncoverenv}
\end{frame}


\begin{frame}
	\begin{uncoverenv}<+->
		Use $\tm{⋅} : \mbf{eo} → \mbf{lp}$ to translate \alert{terms/types}
		to LambdaPi terms.
		$$
			\begin{array}{r@{\ =\ }l}
				\tm{s}      &
				\begin{dcases}
					\ (\leadsto) & \text{if $s = \paren{\mtt{->}}$}, \\
					\ \iden s    & \text{otherwise}
				\end{dcases}
				\\[6mm]
				%
				\tm{e ∗ e'} & \tm{e} ⋅ \tm{e'}
			\end{array}
		$$
	\end{uncoverenv}
	\\
	\begin{uncoverenv}<+->%
		\exxample Consider translating the following type.
		$$\begin{array}{r@{\ = \ }l}
				\tm{\paren{\mtt{->}\ \mtt{Bool}\ \paren{\mtt{BitVec}\ \mtt{5}}}}
				 &
				\tm{\paren{\mtt{->} ∗ \mtt{Bool}} ∗ \paren{\mtt{BitVec} ∗ \mtt{5}}}
				\\[2mm]
				 &
				\tm{\mtt{Bool}} \leadsto \tm{\mtt{BitVec} ∗ \mtt{5}}
				\\[2mm]
				 &
				\iden{\mtt{Bool}} \leadsto \paren{\iden{\mtt{BitVec}} ⋅ \iden{\mtt{5}}}
			\end{array}$$
	\end{uncoverenv}
\end{frame}

\begin{frame}[fragile]
	\begin{uncoverenv}<+->%
		Now, we can translate \alert{constant declarations}, e.g.;
		$$
			\begin{array}{c}
				\paren{\mtt{declare-const}\ s\ \paren{\mtt{->}\ \plur e n}\ \maybe{α}}
				\\[2mm]
				⇓
				\\[2mm]
				\mtt{constant symbol}\ {\iden s} : \mtt{El}\ \tm{\paren{\mtt{->}\ \plur e n}};
				\\[2mm]
				⇓
				\\[2mm]
				\mtt{constant symbol}\ {\iden s} : \mtt{El}\ \paren{\tm{e_1} \leadsto \ldots \leadsto \tm{e_n}};
			\end{array}
		$$
	\end{uncoverenv}
	\\
	\begin{uncoverenv}<+->%
		Also, \alert{update} the attribute map $γ$ with $\paren{s ↦ α}$.
	\end{uncoverenv}
	% 	\\
	% 	\begin{uncoverenv}<+->%
	% 		\exxample
	% 		\begin{lstlisting}
	% (declare-type Array (Type Type))\end{lstlisting}
	% 		\begin{lstlisting}
	% symbol {Array} : Set → Set;\end{lstlisting}
	% 	\end{uncoverenv}
\end{frame}

\begin{frame}
	\begin{uncoverenv}<+->
		Translation of (implicit) parameters is easy.
		$$
			\begin{array}{r@{\ =\ }l}
				\tra{\paren{s\ e}}_{\mbf{param}}
				 & (\tm{s} : \ty{e})
				\\[2mm]
				\tra{\paren{s\ e\ \mtt{:implicit}}}_{\mbf{param}}
				 & [\tm{s} : \ty{e}]
			\end{array}
		$$
	\end{uncoverenv}
\end{frame}

\begin{frame}[fragile]
	\begin{uncoverenv}<+->%
		Translate \alert{parameterized constant declarations} thus:
		$$
			\begin{array}{c}
				\paren{\mtt{declare-parameterized-const}\ s\ \paren{\plur ρ n}\ e}
				\\[2mm]
				⇓
				\\[2mm]
				\mtt{constant symbol}\ {\iden s}\ \tra{ρ_1}\ldots\tra{ρ_n}
				: \mtt{El}\ \tm{e};
			\end{array}
		$$
	\end{uncoverenv}
	\\
	\begin{uncoverenv}<+->%
		\exxample Consider translating the following declaration.
		\begin{lstlisting}
(declare-parameterized-const =
  ((A Type :implicit)) (-> A A Bool)
  :chainable and
)\end{lstlisting}
		\begin{lstlisting}
constant symbol {|=|} [A : Set] : El (A ~> A ~> Bool)\end{lstlisting}
	\end{uncoverenv}
\end{frame}

% \begin{frame}
% 	%
% 	Translating parameters with \mtt{:list} produces an attribute map.
% 	$$
% 		\begin{array}{r@{\ }c@{\ }l}
% 			\tra{\paren{s\ e\ \mtt{:list}}}^γ_{\mbf{param}}
% 			 & {=} & \paren{(\tm{s} : \ty{e}),\ \paren{s ↦ \mtt{:list}}} \\
% 		\end{array}
% 	$$
% \end{frame}

\begin{frame}[fragile]
	%
	\alert{Definitions} are translated thus:
	$$
		\begin{array}{c}
			\paren{\mtt{define}
				\ s
				\  \paren{\plur ρ n}
				\ e
				\ \maybe{\mtt{:type}\ e'}
			}
			\\[2mm]
			⇓
			\\[2mm]
			\mtt{symbol}\ \iden{s}\ \tra{\plur ρ n}\ \maybe{{:}\ \tm{e'}}\ {{≔}\ \tm{e}};
		\end{array}
	$$
\end{frame}

\begin{frame}[fragile]
	%
	Programs are translated.\\[2mm]
	\begin{uncoverenv}<+->%
		\exxample Translation of \texttt{\$from\_clause}.
		\begin{lstlisting}
sequential symbol
  {|$from_clause|} : (El Bool → El Bool);

rule {|$from_clause|} (or $F1 $F2) |->
  {|eo::ite|} [Bool]
    ({|eo::is_eq|} [Bool] $F2 false)
    $F1 (or $F1 $F2)

with {|$from_clause|} $F1 |-> $F1;\end{lstlisting}
	\end{uncoverenv}
\end{frame}

\begin{frame}[fragile]
	%
	Rule declarations are translated.\\[2mm]
	\begin{uncoverenv}<+->%
		\exxample Translation of \texttt{\$from\_clause}.
		\begin{lstlisting}
sequential symbol
  cnf_implies_pos_aux : (El Bool → El Bool);

rule cnf_implies_pos_aux (=> $F1 $F2)
  |-> or (not (=> $F1 $F2))
        (or (not $F1) (or $F2 false));

constant symbol cnf_implies_pos : Π (x0 : El Bool),
  El (Proof (cnf_implies_pos_aux x0));\end{lstlisting}
	\end{uncoverenv}
\end{frame}


\begin{frame}[fragile]
	%
	Proof scripts are translated:\\[2mm]
	\begin{uncoverenv}<+->%
		\exxample Translation of \texttt{\$from\_clause}.
		\begin{lstlisting}[basicstyle=\ttfamily\tiny]
constant symbol Z : Set;
constant symbol input : El Bool;
constant symbol reg : El Bool;
constant symbol nf : El Z;
constant symbol flash : El Z;
constant symbol circuit : El Bool;
symbol {|@t1|} : El Bool ≔ not input;
symbol {|@t2|} : El Bool ≔ not reg;
symbol {|@t3|} : El Bool ≔ and input (and {|@t2|} true);
constant symbol {|@p1|} : El (Proof circuit);
constant symbol {|@p2|} : El (Proof (= nf flash));
constant symbol {|@p3|} : El (Proof (not (or {|@t3|} (or {|@t1|} (or reg false)))));
symbol {|@p4|} : El (Proof (not {|@t3|})) ≔ not_or_elim [or {|@t3|} (or {|@t1|} (or reg false))] {|@p3|} {|eo::0|};
symbol {|@p5|} : El (Proof {|@t2|}) ≔ not_or_elim [or {|@t3|} (or {|@t1|} (or reg false))] {|@p3|} ({|eo::succ|} ({|eo::succ|} {|eo::0|}));
symbol {|@p6|} : El (Proof (not {|@t1|})) ≔ not_or_elim [or {|@t3|} (or {|@t1|} (or reg false))] {|@p3|} ({|eo::succ|} {|eo::0|});
symbol {|@p7|} : El (Proof input) ≔ not_not_elim [input] {|@p6|};
symbol {|@p8_aux|} : El (Proof (and input (and {|@t2|} true))) ≔ and_cons {|@p7|} (and_cons {|@p5|} trueI);
symbol {|@p8|} : El (Proof {|@t3|}) ≔ and_intro [and input (and {|@t2|} true)] {|@p8_aux|};
symbol {|@p9|} : El (Proof false) ≔ contra [{|@t3|}] {|@p8|} {|@p4|};\end{lstlisting}
	\end{uncoverenv}
\end{frame}

\section{Results \& Future Work}


\setbeamercolor*{structure}{fg=MaterialOrange400}
\setbeamercolor*{background canvas}{bg=MaterialOrange400}
\begin{frame}[standout]
	\centering\Huge
	\textbf{Results \& Future Work}
\end{frame}

\begin{frame}
	%
	Carve out the portion of CPC supporting \alert{QFUF}.
	\begin{itemize}
		\item<+-> \alert{Rodin} SMT-LIB benchmark, 30 unsat problems.
		\item<+-> Run \texttt{cvc5} with \texttt{--proof-format=cpc}, dump proofs.
		\item<+-> Check which CPC rules were used, calculate dependencies.
		\item<+-> Make some minor modifications, call this fork \alert{CPC-mini}.
	\end{itemize}
\end{frame}

\begin{frame}
	Translate CPC-mini to LambdaPi using \texttt{eo2lp}.
	% \begin{itemize}
	% 	\item<+-> total translation time: ${<}\ 10\text{ms}$.
	% 	\item<+-> $N$ inference rules, $M$ programs.
	% 	\item<+-> total typechecking time: ${<}\ 5\text{ms}$.
	% \end{itemize}
\end{frame}

\begin{frame}
	Translate all of our Rodin proofs.
	% \begin{itemize}
	% 	\item<+-> max translation time: ${<}\ 10\text{ms}$.
	% 	\item<+-> max number proof steps: $70$
	% 	\item<+-> max checking time: ${<}\ 5\text{ms}$.
	% \end{itemize}
\end{frame}

\begin{frame}
	Lots of potential for \alert{future work}:
	\begin{itemize}
		\item[\emoji{globe-showing-europe-africa}]<+-> Support full CPC: arithmetic, strings, bit-vectors, etc.
		\item[\emoji{chart-increasing}]<+-> Scale up to bigger proofs.
		\item[\emoji{broom}]<+-> Tidy translation: perform elaboration in LambdaPi?
		\item[\emoji{flag-brazil}]<+-> Do all of this in Brazil, Nov 2025?
	\end{itemize}
\end{frame}




% TODO.
% smt-lib 3 proposal.
% - dependent types, polymorphism
% - commands
% chad browns notes.
% Eunoia, Ethos.

% SMT-LIB 3 in CC
% proof-irrelevant semantics of CC

% dekuti encoding of CC.
% future proofing:
% translating smt3 concrete syntax into CC does the same for SMT 2.6 and for SMT 2.7.


% some argument about why even though we have a much more expressive logic,
% SMT solvers will only still support a small fragment of it.
% regardless, it is very important for many reasons to

\end{document}



% \begin{frame}[fragile, plain]
% 	\exxample Proof script generated by \textsc{cvc5}.
% 	\begin{lstlisting}
% (declare-const =
%   (-> (! Type :var T :implicit) T T Bool))
% (declare-rule symm ((T Type) (t T) (s T))
%     :premises ((= t s))
%     :conclusion (= s t)
% )
% (declare-type Int ())
% (declare-const a Int)
% (declare-const b Int)
% (assume @p0 (= a b))
% (step @p1 (= b a) :rule symm :premises (@p0))\end{lstlisting}
% 	%
% 	Declaring rules requires adding $\PROP$-level functions, e.g.;
% 	%
% 	$$\msf{symm} :
% 		(\pii {T : \TYPE,\,t : T,\,s : T} {\lift{t = s} → \lift{s = t}})$$
% \end{frame}

% \begin{frame}
% 	\begin{itemize}
% 		\item<+-> \emph{\color{oc-green-9}Goal:}
% 		      Implement a translation from SMT-LIB v3 and Eunoia to
% 		      an OCaml representation of CC.
% 		\item<+-> Use axioms of Theory $𝕌$ for encoding CC in Lambdapi.
% 		      %
% 		      $$
% 			      \begin{array}{l@{\quad}l}
% 				      \eval{\TYPE}_{\lp} ≡ \tt{Set}
% 				       &
% 				      \eval{\lam {x : t_1} {t_2}}_{\lp} ≡ \lam {x' : \eval{t_1}} {\eval{t_2}}
% 				      \\
% 				      \eval{\PROP}_{\lp} ≡ \tt{Prop}
% 				       &
% 				      \eval{\pii {x : t_1}{t_2}}_{\lp} ≡ \eval{t_1}_{\lp} \leadsto^d \eval{t_2}_{\lp}
% 			      \end{array}
% 		      $$
% 	\end{itemize}
% \end{frame}
