\documentclass[lualatex, compress, 12pt]{beamer}
\usetheme{Arguelles}

\usepackage{xcolor}
\usepackage{ebproof}
\usepackage{unicode-math-input}
\usepackage{MnSymbol}
\usepackage{stmaryrd}
\usepackage{mathtools}
\usepackage{amssymb}

\usepackage{listings}
\usepackage{array}
% \newfontfamily\Alegreya{Alegreya-Regular.otf}[
%   BoldFont=Alegreya-Bold.otf,
%   ItalicFont=Alegreya-Italic.otf,
%   BoldItalicFont=Alegreya-BoldItalic.otf]

\title{Eunoia to LambdaPi}
\author{Ciarán Dunne and Guillaume Burel}
\institute{ENS Paris-Saclay, INRIA}

\lstdefinelanguage{smtlib}{
    morekeywords={
        assert, check-sat, check-sat-assuming, echo, exit,
        get-assertions, get-assignment, get-info, get-model, get-option,
        get-proof, get-unsat-assumptions, get-unsat-core, get-value,
        pop, push, reset, reset-assertions, set-info, set-logic, set-option,
        declare-const, define-const,
        declare-fun, define-fun, define-fun-rec,
        declare-sort, define-sort,
        declare-type, define-type,
        declare-rule, program,
        assume, step, par, unknown
    },
    alsoletter={-, +,-,*,/,=,>,<,>=,<=,=>},
    extendedchars=true,
    sensitive=true,
    morecomment=[l]{;},  % Line comments start with ;
    morestring=[b]",      % Strings are enclosed in "
    morestring=[b]|,      % Symbols are enclosed in |
    morekeywords=[2]{ite, distinct, true, false, not, or, and, xor, forall, exists, choose, lambda, let, as}, % logic keywords
    morekeywords=[3]{Type, Bool, Int, Real, List, BitVec, FloatingPoint},  % types
    morekeywords=[4]{+,-,*,/,=,>,<,>=,<=,=>},  % infix operators
    morekeywords=[5]{sat, unsat},
    commentstyle=\color{gray},        % Comment color
    keywordstyle=\color{blue},        % Keyword color
    keywordstyle=[2]\color{purple},   % Control keywords color
    keywordstyle=[3]\color{teal},     % Data types color
    keywordstyle=[4]\color{red},      % Logical operators color
    keywordstyle=[5]\color{magenta},
    stringstyle=\color{violet},       % String color
    numberstyle=\color{violet},       % Numeric constants color
    basicstyle=\ttfamily\footnotesize,             % Use a monospaced font
    moredelim=[is][\itshape]{|}{|},   % Enclose symbols in vertical bars in italics
    numbers=left,                     % Line numbers on the left
    numberstyle=\tiny\color{gray},    % Line numbers style
    stepnumber=1,                     % Line numbers step
    escapeinside={(*@}{@*)},          % Allow LaTeX inside listings
    showstringspaces=false,           % Do not display spaces as underscores
    breaklines=false,                   % Automatically break long lines
    keepspaces=true,
    showlines=false,
    mathescape=true,
    numberblanklines=false
}

% For inline SMT-LIB code
\newcommand{\smt}[1]{\lstinline[language=smtlib]{#1}}
\newcommand{\exxample}{{\mediumfont\itshape{\textcolor{oc-green-9}{Example.} }}}
\setbeamercolor*{structure}{fg=oc-indigo-5}
\lstset{language=smtlib}

\newcommand{\binddot}[3]{#1\,#2.\,#3}
\newcommand{\plur}[2]{{#1}_{1}, \ldots, {#1}_{#2}}

\newcommand{\msf}{\mathsf}
\newcommand{\mbf}{\mathbf}

\newcommand{\ar}{\msf{ar}}
\newcommand{\Sort}{\msf{Sort}}
\newcommand{\Rank}{\msf{Rank}}
\newcommand{\Var}{\msf{Var}}

\newcommand{\Bool}{\mbf{Bool}}
\newcommand{\true}{\msf{true}}
\newcommand{\false}{\msf{false}}

\newcommand{\mtt}[1]{{\color{magenta}\texttt{#1}}}
\newcommand{\sat}{\mtt{sat}}
\newcommand{\unsat}{\mtt{unsat}}

\newcommand{\all}{\binddot∀}
\newcommand{\exi}{\binddot∃}
\newcommand{\nexi}{\binddot∄}

\newcommand{\TYPE}{\mtt{type}}
\newcommand{\PROP}{\mtt{prop}}
\newcommand{\KIND}{\mtt{kind}}
\newcommand{\lam}{\binddot{λ}}
\newcommand{\pii}{\binddot{Π}}
\newcommand{\lift}[1]{⦉#1⦊}

\newcommand{\Term}{\msf{Term}}
\newcommand{\lett}[2]{\mbf{let}\ #1\ \mbf{in}\ #2}

\newcommand{\norm}[1]{}

\newcommand{\TermGrammar}{
  \begin{array}{l@{\ }r@{\ \ }l@{\quad}l}
    t ∈ \Term(Σ)
    & ⩴ & x & \textit{variables}
  \\
    & ∣ & f(\plur t n) & \textit{function app.}
  \\
    & ∣ & f^σ(\plur t n) & \textit{sorted function app.}
  \\
    & ∣ & \all {x : σ} t & \textit{universal quantification}
  \\
    & ∣ & \exi {x : σ} t & \textit{existential quantification}
  \\
    & ∣ & \lett {x ≔ t_1} {t_2} & \textit{local definition}
  \end{array}
}

\newcommand{\eval}[1]{\llbracket\,#1\,\rrbracket }
\newcommand{\gtext}[1]{\text{{\color{gray}#1}}}
\newcommand{\lp}{\msf{lp}}

\begin{document}

\frame[plain]{\titlepage}

\section{Background}

\begin{frame}[fragile]{Background}
	\begin{itemize}
		\item<+-> \alert{Satisfiability modulo theories} (SMT) refers to the area
		      of automated reasoning aimed at solving (first-order) satisfiability
		      problems within various mathematical theories.
		      %
		\item<+-> \alert{SMT-LIB} (v2.7) is a formal language
		      used for satisfiability problems and specification
		      of theories.
	\end{itemize}
	%
	\begin{uncoverenv}<+->
		\exxample
		Find $x,y ∈ ℤ$ such that $(x + 2y) = 20$ and $(x - y) = 2$.
		\begin{lstlisting}
(set-logic QF_LIA)
(declare-const x Int)
(declare-const y Int)
(assert (= (+ x (* 2 y)) 20))
(assert (= (- x y) 2))
(check-sat)
(exit)\end{lstlisting}\end{uncoverenv}
\end{frame}


\begin{frame}[fragile]
	\begin{itemize}
		\item[]<+->{For satisfiable specifications,
		      \texttt{cvc5} produces an \alert{assignment} for the
		      uninterpreted symbols of the input problem.}
	\end{itemize}
	\begin{uncoverenv}<+->
		\exxample
		\begin{lstlisting}
(set-logic QF_LIA)
(set-option :produce-models true)
(declare-const x Int)
(declare-const y Int)
(assert (= (+ x (* 2 y)) 20))
(assert (= (- x y) 2))
(check-sat)
(get-model)
(exit)\end{lstlisting}
		\hrule
		\begin{lstlisting}
sat
(
  (define-fun x () Int 8)
  (define-fun y () Int 6)
)\end{lstlisting}

	\end{uncoverenv}
\end{frame}

\begin{frame}[fragile]
	\begin{itemize}
		\item[]<+-> \hspace{-5mm}Some SMT solvers produce \alert{proofs}
		      for unsatisfiable problems.
	\end{itemize}
	\begin{uncoverenv}<+->
		\exxample
		\begin{lstlisting}[basicstyle=\ttfamily\footnotesize]
(set-logic QF_UF)
(set-option :produce-proofs true)
(declare-const p Bool)
(assert (and p (not p)))
(check-sat)
(get-proof)
(exit)\end{lstlisting}
		\hrule
		\begin{lstlisting}[basicstyle=\ttfamily\footnotesize]
unsat
(declare-fun p () Bool)
(assume @p1 (and p (not p)))
(step @p2 :rule and_elim :premises (@p1) :args (1))
(step @p3 :rule and_elim :premises (@p1) :args (0))
(step @p4 false :rule contra :premises (@p3 @p2))\end{lstlisting}
	\end{uncoverenv}
\end{frame}

\newcommand{\cpc}{\mathsf{CPC}}

\begin{frame}[fragile]
	\begin{itemize}
		\item<+->
		      \alert{Eunoia} is a \emph{logical framework} for
		      formalizing the proof systems of SMT solvers.

		\item<+-> The \alert{co-operating proof calculus}
		      (i.e., CPC, the proof system of \texttt{cvc5})
		      is formalized as a Eunoia signature $\Pi_{\cpc}$.

		\item<+-> Proofs produced by \texttt{cvc5} are
		      Alethe-style, but exclusively use the rules
		      declared in $\Pi_{\cpc}$.
	\end{itemize}

	\begin{uncoverenv}<+->
		\exxample Introducing a rule for symmetry of equality.
		\begin{lstlisting}
(declare-rule symm ((T Type) (t T) (s T))
    :premises ((= t s))
    :conclusion (= s t)
)\end{lstlisting}

	\end{uncoverenv}
\end{frame}

\begin{frame}[fragile]
	\exxample Proof script generated by \textsc{cvc5}.
	\begin{lstlisting}
(declare-const =
  (-> (! Type :var T :implicit) T T Bool))
(declare-rule symm ((T Type) (t T) (s T))
    :premises ((= t s))
    :conclusion (= s t)
)
(declare-type Int ())
(declare-const a Int)
(declare-const b Int)
(assume @p0 (= a b))
(step @p1 (= b a) :rule symm :premises (@p0))\end{lstlisting}
	%
	Declaring rules requires adding $\PROP$-level functions, e.g.;
	%
	$$\msf{symm} :
		(\pii {T : \TYPE,\,t : T,\,s : T} {\lift{t = s} → \lift{s = t}})$$
\end{frame}

\begin{frame}
	\begin{itemize}
		\item<+-> \emph{\color{oc-green-9}Goal:}
		      Implement a translation from SMT-LIB v3 and Eunoia to
		      an OCaml representation of CC.
		\item<+-> Use axioms of Theory $𝕌$ for encoding CC in Lambdapi.
		      %
		      $$
			      \begin{array}{l@{\quad}l}
				      \eval{\TYPE}_{\lp} ≡ \tt{Set}
				       &
				      \eval{\lam {x : t_1} {t_2}}_{\lp} ≡ \lam {x' : \eval{t_1}} {\eval{t_2}}
				      \\
				      \eval{\PROP}_{\lp} ≡ \tt{Prop}
				       &
				      \eval{\pii {x : t_1}{t_2}}_{\lp} ≡ \eval{t_1}_{\lp} \leadsto^d \eval{t_2}_{\lp}
			      \end{array}
		      $$
	\end{itemize}
\end{frame}



\section{Syntax}



\section{Semantics}


\section{Proofs \& Lambdapi}






% TODO.
% smt-lib 3 proposal.
% - dependent types, polymorphism
% - commands
% chad browns notes.
% Eunoia, Ethos.

% SMT-LIB 3 in CC
% proof-irrelevant semantics of CC

% dekuti encoding of CC.
% future proofing:
% translating smt3 concrete syntax into CC does the same for SMT 2.6 and for SMT 2.7.


% some argument about why even though we have a much more expressive logic,
% SMT solvers will only still support a small fragment of it.
% regardless, it is very important for many reasons to

\end{document}


\end{frame}
